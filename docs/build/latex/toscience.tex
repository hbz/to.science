%% Generated by Sphinx.
\def\sphinxdocclass{report}
\documentclass[letterpaper,10pt,english]{sphinxmanual}
\ifdefined\pdfpxdimen
   \let\sphinxpxdimen\pdfpxdimen\else\newdimen\sphinxpxdimen
\fi \sphinxpxdimen=.75bp\relax
\ifdefined\pdfimageresolution
    \pdfimageresolution= \numexpr \dimexpr1in\relax/\sphinxpxdimen\relax
\fi
%% let collapsible pdf bookmarks panel have high depth per default
\PassOptionsToPackage{bookmarksdepth=5}{hyperref}

\PassOptionsToPackage{warn}{textcomp}
\usepackage[utf8]{inputenc}
\ifdefined\DeclareUnicodeCharacter
% support both utf8 and utf8x syntaxes
  \ifdefined\DeclareUnicodeCharacterAsOptional
    \def\sphinxDUC#1{\DeclareUnicodeCharacter{"#1}}
  \else
    \let\sphinxDUC\DeclareUnicodeCharacter
  \fi
  \sphinxDUC{00A0}{\nobreakspace}
  \sphinxDUC{2500}{\sphinxunichar{2500}}
  \sphinxDUC{2502}{\sphinxunichar{2502}}
  \sphinxDUC{2514}{\sphinxunichar{2514}}
  \sphinxDUC{251C}{\sphinxunichar{251C}}
  \sphinxDUC{2572}{\textbackslash}
\fi
\usepackage{cmap}
\usepackage[T1]{fontenc}
\usepackage{amsmath,amssymb,amstext}
\usepackage{babel}



\usepackage{tgtermes}
\usepackage{tgheros}
\renewcommand{\ttdefault}{txtt}



\usepackage[Bjarne]{fncychap}
\usepackage{sphinx}

\fvset{fontsize=auto}
\usepackage{geometry}


% Include hyperref last.
\usepackage{hyperref}
% Fix anchor placement for figures with captions.
\usepackage{hypcap}% it must be loaded after hyperref.
% Set up styles of URL: it should be placed after hyperref.
\urlstyle{same}

\addto\captionsenglish{\renewcommand{\contentsname}{Contents:}}

\usepackage{sphinxmessages}
\setcounter{tocdepth}{1}



\title{to.science}
\date{Feb 23, 2022}
\release{}
\author{Ingolf Kuss, Peter Reimer, Andres Quast, Jan Schnasse}
\newcommand{\sphinxlogo}{\vbox{}}
\renewcommand{\releasename}{}
\makeindex
\begin{document}

\pagestyle{empty}
\sphinxmaketitle
\pagestyle{plain}
\sphinxtableofcontents
\pagestyle{normal}
\phantomsection\label{\detokenize{index::doc}}


\sphinxAtStartPar
Über dieses Dokument

\sphinxAtStartPar
Dieses Dokument kommt zusammen mit einem
\sphinxhref{https://github.com/hbz/to.science/tree/master/vagrant/ubuntu-14.04}{Vagrantfile}
und beschreibt eine beispielhafte Installation von Regal. Unter
{\hyperref[\detokenize{index:_vagrant}]{\emph{Vagrant}}} findet sich eine Anleitung zur Installation in
einer Virtualbox.

\sphinxAtStartPar
Eine Kurzaufstellung der wichtigsten API\sphinxhyphen{}Calls findet sich unter
\sphinxhref{./api.html}{Regal\sphinxhyphen{}Api}

\sphinxAtStartPar
Dieses Dokument ist im Format rst geschrieben und kann mit dem
Werkzeug sphinx in HTML übersetzt werden. Mehr dazu im Abschnitt
{\hyperref[\detokenize{colophon:dokumentation}]{\sphinxcrossref{\DUrole{std,std-ref}{Dokumentation}}}}.


\chapter{Einleitung}
\label{\detokenize{toscience:einleitung}}\label{\detokenize{toscience:id1}}\label{\detokenize{toscience::doc}}
\sphinxAtStartPar
to.science (ehemals “Regal”) ist eine Content Repository zur Verwaltung
und Veröffentlichung elektronischer Publikationen. Es wird seit 2013 am
\sphinxhref{https://hbz-nrw.de}{Hochschulbibliothekszentrum NRW (hbz)}
entwickelt.

\sphinxAtStartPar
to.science basiert auf den folgenden Kerntechnologien:
\begin{itemize}
\item {} 
\sphinxAtStartPar
Fedora Commons 3

\item {} 
\sphinxAtStartPar
Elasticsearch 1.1

\item {} 
\sphinxAtStartPar
Drupal 7

\item {} 
\sphinxAtStartPar
Playframework 2.4

\item {} 
\sphinxAtStartPar
MySQL 5

\item {} 
\sphinxAtStartPar
Java 8

\item {} 
\sphinxAtStartPar
PHP 5

\end{itemize}

\sphinxAtStartPar
Für die Webarchivierung kommen außerdem Openwayback, Heritrix und WPull
zum Einsatz.
\begin{itemize}
\item {} 
\sphinxAtStartPar
openwayback hbz\sphinxhyphen{}2.3.2

\item {} 
\sphinxAtStartPar
pywb

\item {} 
\sphinxAtStartPar
heritrix 3.2.0

\item {} 
\sphinxAtStartPar
wpull

\end{itemize}

\sphinxAtStartPar
Regal ist ein mehrkomponentiges System. Einzelne Komponenten sind als
Webservices realisiert und kommunizieren über HTTP\sphinxhyphen{}APIs miteinander.
Derzeit sind folgende Komponenten im Einsatz
\begin{itemize}
\item {} 
\sphinxAtStartPar
\sphinxhref{https://github.com/hbz/to.science.api}{to.science.api}

\item {} 
\sphinxAtStartPar
\sphinxhref{https://github.com/hbz/to.science.labels}{to.science.labels}

\item {} 
\sphinxAtStartPar
\sphinxhref{https://github.com/hbz/to.science.forms}{to.science.forms}

\item {} 
\sphinxAtStartPar
\sphinxhref{https://github.com/hbz/skos-lookup}{skos\sphinxhyphen{}lookup}

\item {} 
\sphinxAtStartPar
\sphinxhref{https://github.com/hbz/thumby}{thumby}

\item {} 
\sphinxAtStartPar
\sphinxhref{https://github.com/hbz/DeepZoomService}{deepzoomer}

\item {} 
\sphinxAtStartPar
\sphinxhref{https://github.com/hbz/to.science.drupal}{to.science.drupal}

\end{itemize}

\sphinxAtStartPar
Drupal Themes
\begin{itemize}
\item {} 
\sphinxAtStartPar
\sphinxhref{https://github.com/edoweb/zbmed-drupal-theme}{zbmed\sphinxhyphen{}drupal\sphinxhyphen{}theme}

\item {} 
\sphinxAtStartPar
\sphinxhref{https://github.com/edoweb/edoweb-drupal-theme}{edoweb\sphinxhyphen{}drupal\sphinxhyphen{}theme}

\end{itemize}

\sphinxAtStartPar
Über die Systemschnittstellen können eine ganze Reihe von Drittsystemen
angesprochen werden. Die folgende Abbildung verschafft einen groben
Überblick über eine typische Regal\sphinxhyphen{}Installation und die angebundenen
Drittsysteme.

\begin{figure}[htbp]
\centering
\capstart

\noindent\sphinxincludegraphics{{regal-arch-4}.jpeg}
\caption{Typische Regal\sphinxhyphen{}Installation mit Drupal Frontend, Backendkomponenten und angebundenen Drittsytemen}\label{\detokenize{toscience:id84}}\end{figure}


\chapter{Konzepte}
\label{\detokenize{toscience:konzepte}}\label{\detokenize{toscience:id2}}

\section{Objektmodell}
\label{\detokenize{toscience:objektmodell}}\label{\detokenize{toscience:id3}}
\sphinxAtStartPar
Regal realisiert ein einheitliches Objektmodell in dem sich eine
Vielzahl von Publikationstypen speichern lassen. Die Speicherschicht
wird über {\hyperref[\detokenize{toscience:_fedora_commons_3}]{\emph{Fedora Commons 3}}} realisiert.

\sphinxAtStartPar
Eine einzelne Publikation besteht i.d.R. aus mehreren {\hyperref[\detokenize{toscience:_fedora_commons_3}]{\emph{Fedora Commons
3}}}\sphinxhyphen{}Objekten, die in einer hierarchischen
Beziehung zueinander stehen.


\begin{savenotes}\sphinxattablestart
\centering
\sphinxcapstartof{table}
\sphinxthecaptionisattop
\sphinxcaption{Fedora Object}\label{\detokenize{toscience:id85}}
\sphinxaftertopcaption
\begin{tabulary}{\linewidth}[t]{|T|T|T|}
\hline
\sphinxstyletheadfamily 
\sphinxAtStartPar
Datenstrom
&\sphinxstyletheadfamily 
\sphinxAtStartPar
Pflicht
&\sphinxstyletheadfamily 
\sphinxAtStartPar
Beschreibung
\\
\hline
\sphinxAtStartPar
DC
&
\sphinxAtStartPar
Ja
&
\sphinxAtStartPar
Von Fedora
vorgeschrieben. Wird
für die fedorainterne
Suche verwendet
\\
\hline
\sphinxAtStartPar
RELS\sphinxhyphen{}EXT
&
\sphinxAtStartPar
Ja
&
\sphinxAtStartPar
Von Fedora
vorgeschrieben. Wird
für viele Sachen
verwendet \sphinxhyphen{} (1)
Hierarchien \sphinxhyphen{} (2)
Steuerung der
Sichtbarkeiten \sphinxhyphen{} (2)
OAI\sphinxhyphen{}Providing
\\
\hline
\sphinxAtStartPar
data
&
\sphinxAtStartPar
Nein
&
\sphinxAtStartPar
Die eigentlichen
Daten der
Publikation. Oft ein
PDF.
\\
\hline
\sphinxAtStartPar
metadata oder
metadata2
&
\sphinxAtStartPar
Nein
&
\sphinxAtStartPar
Bibliografische
Metadaten. Metadata2
wurde mit dem Umstieg
auf die Lobid\sphinxhyphen{}API v2
eingeführt.
\\
\hline
\sphinxAtStartPar
objectTimestamp
&
\sphinxAtStartPar
Nein
&
\sphinxAtStartPar
Eine Datei mit einem
Zeitstempel. Der
Zeitstempel wird bei
bestimmten Aktionen
gesetzt.
\\
\hline
\sphinxAtStartPar
seq
&
\sphinxAtStartPar
Nein
&
\sphinxAtStartPar
Eine Hilfsdatei mit
einem JSON\sphinxhyphen{}Array. Das
Array zeigt an, in
welcher Reihenfolge
Kindobjekte
anzuzeigen sind.
Dieses Hilfskonstrukt
existiert, da in der
RELS\sphinxhyphen{}EXT keine
RDF\sphinxhyphen{}Listen abgelegt
werden können.
\\
\hline
\sphinxAtStartPar
conf
&
\sphinxAtStartPar
Nein
&
\sphinxAtStartPar
Websites und
Webschnitte speichern
in einem
conf\sphinxhyphen{}Datenstrom alle
Parameter mit denen
die zugehörige
Webseite geharvested
wurde.
\\
\hline
\end{tabulary}
\par
\sphinxattableend\end{savenotes}

\sphinxAtStartPar
Die Metadaten werden als ASCII\sphinxhyphen{}Kodierte N\sphinxhyphen{}Triple abgelegt. Da alle
Fedora\sphinxhyphen{}Daten als Dateien im Dateisystem abgelegt werden, ist diese
Veriante besonders robust gegen Speicherfehler. N\sphinxhyphen{}Triple ist ein Format,
dass sich Zeilenweise lesen lässt. ASCII ist die einfachste Form der
Textkodierung.

\sphinxAtStartPar
Die Daten werden als “managed”\sphinxhyphen{}Datastream in den Objektspeicher der
Fedora abgelegt. Eine Ausnahme bilden Webseiten. Die als WARC
gespeicherten Inhalte werden “unmanaged” lediglich verlinkt. Im Fedora
Objektspeicher wird nur eine Datei mit der ensprechenden Referenz
abgelegt.


\section{Namespaces und Identifier}
\label{\detokenize{toscience:namespaces-und-identifier}}\label{\detokenize{toscience:id4}}
\sphinxAtStartPar
Jede Regal\sphinxhyphen{}Installation arbeitet auf einem festgelegten Namespace. Wenn
über die {\hyperref[\detokenize{toscience:_regal_api_2}]{\emph{regal\sphinxhyphen{}api}}} Objekte angelegt werden, finden
sich diese immer in dem entsprechenden Namespace wieder. Hinter dem
Namespace findet sich, abgetrennt mit einem Dopplepunkt eine
hochlaufende Zahl, die i.d.R. über {\hyperref[\detokenize{toscience:_fedora_commons_3}]{\emph{Fedora Commons
3}}} bezogen wird.

\sphinxAtStartPar
Der so zusammengesetzte Identifier kommt in allen Systemkomponenten zum
Einsatz.


\begin{savenotes}\sphinxattablestart
\centering
\sphinxcapstartof{table}
\sphinxthecaptionisattop
\sphinxcaption{Beispiel Regal Identifier}\label{\detokenize{toscience:id86}}
\sphinxaftertopcaption
\begin{tabulary}{\linewidth}[t]{|T|T|T|}
\hline
\sphinxstyletheadfamily 
\sphinxAtStartPar
ID
&\sphinxstyletheadfamily 
\sphinxAtStartPar
Komponente
&\sphinxstyletheadfamily 
\sphinxAtStartPar
URL
\\
\hline
\sphinxAtStartPar
regal:1
&
\sphinxAtStartPar
drupal
&
\sphinxAtStartPar
\sphinxurl{http://local}
host/resource/regal:1
\\
\hline
\sphinxAtStartPar
regal:1
&
\sphinxAtStartPar
regal\sphinxhyphen{}api
&
\sphinxAtStartPar
\sphinxurl{http://api.local}
host/resource/regal:1
\\
\hline
\sphinxAtStartPar
regal:1
&
\sphinxAtStartPar
fedora
&
\sphinxAtStartPar
ht
tp://localhost:8080/f
edora/objects/regal:1
\\
\hline
\sphinxAtStartPar
regal:1
&
\sphinxAtStartPar
elasticsearch
&
\sphinxAtStartPar
\sphinxurl{http://localhost:92}
00/regal/\_all/regal:1
\\
\hline
\end{tabulary}
\par
\sphinxattableend\end{savenotes}


\section{Deskriptive Metadaten}
\label{\detokenize{toscience:deskriptive-metadaten}}\label{\detokenize{toscience:id5}}
\sphinxAtStartPar
Regal unterstützt eine große Anzahl von Metadatenfeldern zur
Beschreibung von bibliografischen Ressourcen. Jedes in Regal
verspeicherte Objekt kann mit Hilfe von RDF\sphinxhyphen{}Metadaten beschrieben
werden. Das System verspeichert grundsätzlich alle Metadaten, solange
Sie im richtigen Format an die Schnittstelle gespielt werden.

\sphinxAtStartPar
Darüber hinaus können über bestimmte Angaben, bestimmte weitergehende
Funktionen angesteuert werden. Dies betrifft u.A.:
\begin{itemize}
\item {} 
\sphinxAtStartPar
Anzeige und Darstellung

\item {} 
\sphinxAtStartPar
Metadatenkonvertierungen

\item {} 
\sphinxAtStartPar
OAI\sphinxhyphen{}Providing

\item {} 
\sphinxAtStartPar
Suche

\end{itemize}

\sphinxAtStartPar
Alle bekannten Metadateneinträge werden in der Komponente
{\hyperref[\detokenize{toscience:_etikett}]{\emph{Etikett}}} verwaltet. In {\hyperref[\detokenize{toscience:_etikett}]{\emph{Etikett}}} kann
konfiguriert werden, welche URIs aus den RDF\sphinxhyphen{}Daten in das JSON\sphinxhyphen{}LD\sphinxhyphen{}Format
von {\hyperref[\detokenize{toscience:_regal_api_2}]{\emph{regal\sphinxhyphen{}api}}} überführt werden. Außerdem kann die
Reihenfolge der Darstellung, und das Label zur Anzeige gesetzt werden.


\begin{savenotes}\sphinxattablestart
\centering
\sphinxcapstartof{table}
\sphinxthecaptionisattop
\sphinxcaption{Etikett\sphinxhyphen{}Eintrag für dc:title}\label{\detokenize{toscience:id87}}
\sphinxaftertopcaption
\begin{tabulary}{\linewidth}[t]{|T|T|T|T|T|T|T|}
\hline
\sphinxstyletheadfamily 
\sphinxAtStartPar
Label
&\sphinxstyletheadfamily 
\sphinxAtStartPar
Pi
ctogram
&\sphinxstyletheadfamily 
\sphinxAtStartPar
Name
(json)
&\sphinxstyletheadfamily 
\sphinxAtStartPar
URI
&\sphinxstyletheadfamily 
\sphinxAtStartPar
Type
&\sphinxstyletheadfamily 
\sphinxAtStartPar
Co
ntainer
&\sphinxstyletheadfamily 
\sphinxAtStartPar
Comment
\\
\hline
\sphinxAtStartPar
Titel
&
\sphinxAtStartPar
keine
Angabe
&
\sphinxAtStartPar
title
&
\sphinxAtStartPar
ht
tp://pu
rl.org/
dc/term
s/title
&
\sphinxAtStartPar
String
&
\sphinxAtStartPar
keine
Angabe
&
\sphinxAtStartPar
keine
Angabe
\\
\hline
\end{tabulary}
\par
\sphinxattableend\end{savenotes}

\sphinxAtStartPar
\sphinxstylestrong{Etikett\sphinxhyphen{}Eintrag als Json.}

\sphinxAtStartPar
“title”:\{ “@id”=”\sphinxurl{http://purl.org/dc/terms/title}”, “label”=”Titel” \}

\sphinxAtStartPar
Die etikett Datenbank wird beim Neustart jeder
{\hyperref[\detokenize{toscience:_regal_api_2}]{\emph{regal\sphinxhyphen{}api}}}\sphinxhyphen{}Instanz eingelesen. Außerdem wird die
HTTP\sphinxhyphen{}Schnittstelle von Etikett immer wieder angesprochen um zur Anzeige
geeignete Labels in das System zu holen und anstatt der rohen URIs
einzublenden. Das {\hyperref[\detokenize{toscience:_regal_api_2}]{\emph{regal\sphinxhyphen{}api}}}\sphinxhyphen{}Modul läuft dabei auch
ohne den Etikett\sphinxhyphen{}Services, allerdings nur mit eingeschränkter
Funktionalität; beispielsweise fallen Anzeigen von verlinkten Ressourcen
(und das ist in Regal fast alles) weniger schön aus.


\subsection{Wie kommen bibliografische Metadaten ins System?}
\label{\detokenize{toscience:wie-kommen-bibliografische-metadaten-ins-system}}\label{\detokenize{toscience:id6}}
\sphinxAtStartPar
In Regal können bibliografische Metadaten aus dem hbz\sphinxhyphen{}Verbundkatalog an
Ressourcen “angelinkt” werden. Dies erfolgt über Angabe der ID des
entsprechenden Titelsatzes (z.b. HT017766754). Mit Hilfe dieser ID kann
Regal einen Titelimport durchführen. Dabei wird auf die Schnittstellen
der \sphinxhref{https://lobid.org}{Lobid\sphinxhyphen{}API} zugegriffen.

\sphinxAtStartPar
Regal bietet außerdem die Möglichkeit, Metadaten über Erfassungsmasken
zu erzeugen und zu speichern. Dies erfolgt mit Hilfe des Moduls
{\hyperref[\detokenize{toscience:_zettel}]{\emph{Zettel}}}. {\hyperref[\detokenize{toscience:_zettel}]{\emph{Zettel}}} ist ein Webservice, der
verschiedene HTML\sphinxhyphen{}Formulare bereitstellt. Die Formulare können
RDF\sphinxhyphen{}Metadaten einlesen und ausgeben. Zettel\sphinxhyphen{}Formulare werden über
Javascript mit Hilfe eines IFrame in die eigentliche Anwendung
angebunden. Über Zettel werden Konzepte aus dem Bereich Linked Data
umgesetzt. So können Feldinhalte über entsprechende Eingabeelemente in
Drittsystemen recherchiert und verlinkt werden. Die Darstellung von
Links erfolgt in Zettel mit Hilfe von {\hyperref[\detokenize{toscience:_etikett}]{\emph{Etikett}}}.
Umfangreichere Notationssysteme wie Agrovoc oder DDC werden über einen
eigenen Index aus dem Modul {\hyperref[\detokenize{toscience:_skos_lookup}]{\emph{skos\sphinxhyphen{}lookup}}} eingebunden.
Zettel unterstützt zur Zeit folgende Linked\sphinxhyphen{}Data\sphinxhyphen{}Quellen:
\begin{itemize}
\item {} 
\sphinxAtStartPar
\sphinxhref{https://lobid.org/gnd}{Lobid (GND)}

\item {} 
\sphinxAtStartPar
\sphinxhref{https://lobid.org/resources}{Lobid (Ressource)}

\item {} 
\sphinxAtStartPar
\sphinxhref{http://aims.fao.org/vest-registry/vocabularies/agrovoc}{Agrovoc}

\item {} 
\sphinxAtStartPar
\sphinxhref{https://www.oclc.org/en/dewey.html}{DDC}

\item {} 
\sphinxAtStartPar
\sphinxhref{https://www.crossref.org/services/funder-registry/}{CrossRef (Funder
Registry)}

\item {} 
\sphinxAtStartPar
\sphinxhref{https://orcid.org/}{Orcid}

\item {} 
\sphinxAtStartPar
\sphinxhref{https://www.geonames.org/}{Geonames}

\item {} 
\sphinxAtStartPar
\sphinxhref{https://www.openstreetmap.org}{Open Street Maps Koordinaten}

\end{itemize}


\section{Anzeige und Darstellung}
\label{\detokenize{toscience:anzeige-und-darstellung}}\label{\detokenize{toscience:id7}}
\sphinxAtStartPar
Über die Schnittstellen der {\hyperref[\detokenize{toscience:_regal_api_2}]{\emph{regal\sphinxhyphen{}api}}} können
unterschiedliche Darstellungen einer Publikation bezogen werden. Über
\sphinxhref{https://de.wikipedia.org/wiki/Content\_Negotiation}{Content
Negotiation}
können Darstellungen per HTTP\sphinxhyphen{}Header angefragt werden. Um
unterschiedliche Darstellungen im Browser anzeigen zu lassen, kann
außerden, über das Setzen von entsprechenden Endungen, auf
unterschiedliche Representationen eine Resource zugegriffen werden.

\sphinxAtStartPar
\sphinxstylestrong{Auswahl von Pfaden zu unterschiedlichen Representationen einer
Ressource.}

\sphinxAtStartPar
/resource/regal:1 /resource/regal:1.json /resource/regal:1.rdf
/resource/regal:1.epicur /resource/regal:1.mets

\sphinxAtStartPar
In der HTML\sphinxhyphen{}Darstellung greift {\hyperref[\detokenize{toscience:_regal_api_2}]{\emph{regal\sphinxhyphen{}api}}} auf den
Hilfsdienst {\hyperref[\detokenize{toscience:_thumby}]{\emph{Thumby}}} zu um darüber Thumbnail\sphinxhyphen{}Darstellungen
von PDFs oder Bilder zu kreieren. Bei großen Bildern wird außerdem der
{\hyperref[\detokenize{toscience:_deepzoomer}]{\emph{Deepzoomer}}} angelinkt, der eine Darstellung von
hochauflösenden Bildern über das Tool
\sphinxhref{https://openseadragon.github.io/}{OpenSeadragon} erlaubt. Video\sphinxhyphen{} und
Audio\sphinxhyphen{}Dateien werden über die entsprechenden HTML5 Elemente gerendert.


\section{Der hbz\sphinxhyphen{}Verbundkatalog}
\label{\detokenize{toscience:der-hbz-verbundkatalog}}\label{\detokenize{toscience:id8}}
\sphinxAtStartPar
Metadaten, die über den Verbundkatalog importiert wurden, können über
einen Cronjob regelmäßig aktualisiert werden. Außerdem können diese
Daten über OAI\sphinxhyphen{}PMH an den Verbundkatalog zurückgeliefert werden, so dass
dieser, Links auf die Volltexte erhält.


\section{Metadatenkonvertierung}
\label{\detokenize{toscience:metadatenkonvertierung}}\label{\detokenize{toscience:id9}}
\sphinxAtStartPar
Für die Metadatenkonvertierung gibt es kein festes Vorgehensmodell oder
Werkzeug. I.d.R. gibt es für jede Representation eine oder eine Reihe
von Javaklassen, die für eine On\sphinxhyphen{}the\sphinxhyphen{}fly\sphinxhyphen{}Konvertierung sorgen. Die
HTML\sphinxhyphen{}Darstellung basiert grundlegend auf denselben Daten, die auch im
\sphinxhref{https://www.elastic.co/guide/index.html}{Elasticsearch}\sphinxhyphen{}Index liegen
und ist im wesentlichen eine JSON\sphinxhyphen{}LD\sphinxhyphen{}Darstellung, die mit Hilfe der in
{\hyperref[\detokenize{toscience:_etikett}]{\emph{Etikett}}} hinterlegten Konfiguration aus den
bibliografischen Metadaten gewonnen wurde.


\section{OAI\sphinxhyphen{}Providing}
\label{\detokenize{toscience:oai-providing}}\label{\detokenize{toscience:id10}}
\sphinxAtStartPar
Öffentlich zugängliche Publikationen sind auch über die
OAI\sphinxhyphen{}Schnittstelle verfügbar. Dabei wird jede Publikation einer Reihe von
OAI\sphinxhyphen{}Sets zugeordnet und in unterschiedlichen Formaten angeboten.


\begin{savenotes}\sphinxattablestart
\centering
\sphinxcapstartof{table}
\sphinxthecaptionisattop
\sphinxcaption{OAI Set}\label{\detokenize{toscience:id88}}
\sphinxaftertopcaption
\begin{tabulary}{\linewidth}[t]{|T|T|}
\hline
\sphinxstyletheadfamily 
\sphinxAtStartPar
Set
&\sphinxstyletheadfamily 
\sphinxAtStartPar
Kriterium
\\
\hline
\sphinxAtStartPar
ddc:*
&
\sphinxAtStartPar
Wenn ein dc:subject mit dem
String “http://dewey.info/class/”
beginnt, wird ein Set mit der
entsprechenden DDC\sphinxhyphen{}Nummer
gebildet und die Publikation wird
zugeordnet
\\
\hline
\sphinxAtStartPar
contentType
&
\sphinxAtStartPar
Der “contentType” weist darauf
hin, in welcher Weise die
Publikation in Regal. Abgelegt
ist.
\\
\hline
\sphinxAtStartPar
open\_access
&
\sphinxAtStartPar
All Publikationen, die als
Sichtbarkeit “public” haben
\\
\hline
\sphinxAtStartPar
urn\sphinxhyphen{}set\sphinxhyphen{}1
&
\sphinxAtStartPar
Publikationen mit einer URN, die
mit \sphinxurl{urn:nbn:de:hbz:929:01} beginnt
\\
\hline
\sphinxAtStartPar
urn\sphinxhyphen{}set\sphinxhyphen{}2
&
\sphinxAtStartPar
Publikationen mit einer URN, die
mit \sphinxurl{urn:nbn:de:hbz:929:02} beginnt
\\
\hline
\sphinxAtStartPar
epicur
&
\sphinxAtStartPar
Publikationen, die in einem
URN\sphinxhyphen{}Set sind
\\
\hline
\sphinxAtStartPar
aleph
&
\sphinxAtStartPar
Publikationen , die mit einer
Aleph\sphinxhyphen{}Id verknüpft sind
\\
\hline
\sphinxAtStartPar
edoweb01
&
\sphinxAtStartPar
spezielles, pro
{\hyperref[\detokenize{toscience:_regal_api_2}]{\emph{reg
al\sphinxhyphen{}api}}}\sphinxhyphen{}Instanz
konfigurierbares Set für alle
Publikationen, die im aleph\sphinxhyphen{}Set
sind
\\
\hline
\sphinxAtStartPar
ellinet01
&
\sphinxAtStartPar
spezielles, pro
{\hyperref[\detokenize{toscience:_regal_api_2}]{\emph{reg
al\sphinxhyphen{}api}}}\sphinxhyphen{}Instanz
konfigurierbares Set für alle
Publikationen, die im aleph\sphinxhyphen{}Set
sind
\\
\hline
\end{tabulary}
\par
\sphinxattableend\end{savenotes}


\begin{savenotes}\sphinxattablestart
\centering
\sphinxcapstartof{table}
\sphinxthecaptionisattop
\sphinxcaption{OAI Metadatenformat}\label{\detokenize{toscience:id89}}
\sphinxaftertopcaption
\begin{tabulary}{\linewidth}[t]{|T|T|}
\hline
\sphinxstyletheadfamily 
\sphinxAtStartPar
Format
&\sphinxstyletheadfamily 
\sphinxAtStartPar
Kriterium
\\
\hline
\sphinxAtStartPar
oai\_dc
&
\sphinxAtStartPar
Alle öffentlich sichtbaren
Objekte, die als bestimmte
ContentTypes angelegt wurden.
\\
\hline
\sphinxAtStartPar
epicur
&
\sphinxAtStartPar
Alle Objekte, die eine URN haben
\\
\hline
\sphinxAtStartPar
aleph
&
\sphinxAtStartPar
Alle Objekte, die einen
persistenten Identifier haben
\\
\hline
\sphinxAtStartPar
mets
&
\sphinxAtStartPar
Wie oai\_dc
\\
\hline
\sphinxAtStartPar
rdf
&
\sphinxAtStartPar
Wie oai\_dc
\\
\hline
\sphinxAtStartPar
wgl
&
\sphinxAtStartPar
Format für LeibnizOpen. Alle
Objekte die über das Feld
“collectionOne” einer Institution
zugeordnet wurden und über den
ContentType “article”
eingeliefert wurden.
\\
\hline
\end{tabulary}
\par
\sphinxattableend\end{savenotes}


\section{Suche}
\label{\detokenize{toscience:suche}}\label{\detokenize{toscience:id11}}
\sphinxAtStartPar
Der Elasticsearch\sphinxhyphen{}Index wird mit Hilfe einer JSON\sphinxhyphen{}LD Konvertierung
befüllt. Die Konvertierung basiert im wesentlichen auf den
bibliografischen Metadaten der einzelnen Ressourcen und wir mit Hilfe
der in {\hyperref[\detokenize{toscience:_etikett}]{\emph{Etikett}}} hinterlegten Konfiguration erzeugt.


\section{Zugriffsberechtigungen und Sichtbarkeiten}
\label{\detokenize{toscience:zugriffsberechtigungen-und-sichtbarkeiten}}\label{\detokenize{toscience:id12}}
\sphinxAtStartPar
Regal setzt ein rollenbasiertes Konzept zur Steuerung von
Zugriffsberechtigungen um. Eine besondere Bedeutung kommt dem lesenden
Zugriff auf Ressourcen zu. Einzelne Ressourcen können in ihrer
Sichtbarkeit so eingeschränkt werden, dass nur mit den Rechten einer
bestimmten Rolle lesend zugegriffen werden kann. Dabei kann der Zugriff
auf Metadaten und Daten separat gesteuert werden.

\begin{figure}[htbp]
\centering
\capstart

\noindent\sphinxincludegraphics{{accessControl}.png}
\caption{Screenshot zur Verdeutlichung von Sichtbarkeiten in Regal}\label{\detokenize{toscience:id90}}\end{figure}

\sphinxAtStartPar
Die Konfiguration hat Auswirkungen auf die Sichtbarkeit einer
Publikation in den unterschiedlichen Systemteilen. Die folgende Tabelle
veranschaulicht den derzeitigen Stand der Implementierung.


\subsection{Sichtbarkeiten, Operationen, Rollen}
\label{\detokenize{toscience:sichtbarkeiten-operationen-rollen}}\label{\detokenize{toscience:id13}}

\begin{savenotes}\sphinxattablestart
\centering
\sphinxcapstartof{table}
\sphinxthecaptionisattop
\sphinxcaption{\sphinxstylestrong{Schreibender} Zugriff auf Daten und Metadaten}\label{\detokenize{toscience:id91}}
\sphinxaftertopcaption
\begin{tabulary}{\linewidth}[t]{|T|T|}
\hline

\sphinxAtStartPar
Rolle
&
\sphinxAtStartPar
Art der Aktion
\\
\hline
\sphinxAtStartPar
ADMIN
&
\sphinxAtStartPar
Darf alle Aktionen durchführen.
Auch Bulk\sphinxhyphen{}Aktionen und “Purges”
\\
\hline
\sphinxAtStartPar
EDITOR
&
\sphinxAtStartPar
Darf Objekte anlegen, löschen,
Sichtbarkeiten ändern, etc.
\\
\hline
\end{tabulary}
\par
\sphinxattableend\end{savenotes}


\begin{savenotes}\sphinxattablestart
\centering
\sphinxcapstartof{table}
\sphinxthecaptionisattop
\sphinxcaption{\sphinxstylestrong{Lesender} Zugriff auf Metadaten}\label{\detokenize{toscience:id92}}
\sphinxaftertopcaption
\begin{tabulary}{\linewidth}[t]{|T|T|}
\hline
\sphinxstyletheadfamily 
\sphinxAtStartPar
Sichtbarkeit
&\sphinxstyletheadfamily 
\sphinxAtStartPar
Rolle
\\
\hline
\sphinxAtStartPar
public
&
\sphinxAtStartPar
GUEST,READ
ER,SUBSCRIBER,REMOTE,ADMIN,EDITOR
\\
\hline
\sphinxAtStartPar
private
&
\sphinxAtStartPar
ADMIN,EDITOR
\\
\hline
\end{tabulary}
\par
\sphinxattableend\end{savenotes}


\begin{savenotes}\sphinxattablestart
\centering
\sphinxcapstartof{table}
\sphinxthecaptionisattop
\sphinxcaption{\sphinxstylestrong{Lesender} Zugriff auf Daten}\label{\detokenize{toscience:id93}}
\sphinxaftertopcaption
\begin{tabulary}{\linewidth}[t]{|T|T|}
\hline
\sphinxstyletheadfamily 
\sphinxAtStartPar
Sichtbarkeit
&\sphinxstyletheadfamily 
\sphinxAtStartPar
Rolle
\\
\hline
\sphinxAtStartPar
public
&
\sphinxAtStartPar
GUEST,READ
ER,SUBSCRIBER,REMOTE,ADMIN,EDITOR
\\
\hline
\sphinxAtStartPar
restricted
&
\sphinxAtStartPar
READ
ER,SUBSCRIBER,REMOTE,ADMIN,EDITOR
\\
\hline
\sphinxAtStartPar
remote
&
\sphinxAtStartPar
READ
ER,SUBSCRIBER,REMOTE,ADMIN,EDITOR
\\
\hline
\sphinxAtStartPar
single
&
\sphinxAtStartPar
SUBSCRIBER,ADMIN,EDITOR
\\
\hline
\sphinxAtStartPar
private
&
\sphinxAtStartPar
ADMIN,EDITOR
\\
\hline
\end{tabulary}
\par
\sphinxattableend\end{savenotes}


\section{Benutzerverwaltung}
\label{\detokenize{toscience:benutzerverwaltung}}\label{\detokenize{toscience:id14}}
\sphinxAtStartPar
Die Benutzerverwaltung von Regal findet innerhalb von Drupal statt. Zwar
können auch in der {\hyperref[\detokenize{toscience:_regal_api_2}]{\emph{regal\sphinxhyphen{}api}}} Benutzer angelegt
werden, jedoch ist die Implementierung in diesem Bereich erst
rudimentär.


\subsection{Drupal}
\label{\detokenize{toscience:drupal}}\label{\detokenize{toscience:id15}}
\sphinxAtStartPar
Benutzer in Drupal können über das Modul
{\hyperref[\detokenize{toscience:_regal_drupal}]{\emph{regal\sphinxhyphen{}drupal}}} unterschiedlichen Rollen zugewiesen
werden. Die Authorisierung erfolgt passwortbasiert. Alle Drupal\sphinxhyphen{}Benutzer
greifen über einen vorkonfigurierten Accessor auf die
{\hyperref[\detokenize{toscience:_regal_api_2}]{\emph{regal\sphinxhyphen{}api}}} zu. Alle Zugriffe erfolgen verschlüsselt
unter Angabe eines Passwortes. Die Rolle mit deren Berechtigungen
zugegriffen wird, wird dabei in {\hyperref[\detokenize{toscience:_regal_drupal}]{\emph{regal\sphinxhyphen{}drupal}}}
gesetzt. Die Drupal\sphinxhyphen{}BenutzerId wird als Metadatum in Form eines
proprietären HTTP\sphinxhyphen{}Headers mit an {\hyperref[\detokenize{toscience:_regal_api_2}]{\emph{regal\sphinxhyphen{}api}}}
geliefert.


\subsection{Regal\sphinxhyphen{}Api}
\label{\detokenize{toscience:regal-api}}\label{\detokenize{toscience:id16}}
\sphinxAtStartPar
Auch in regal\sphinxhyphen{}api können Api\sphinxhyphen{}Benutzer angelegt werden. Zur
Benutzerverwaltung wird eine MySQL\sphinxhyphen{}Datenbank eingesetzt, in der die
Passworte der Nutzer abgelegt sind.


\section{Ansichten}
\label{\detokenize{toscience:ansichten}}\label{\detokenize{toscience:id17}}
\sphinxAtStartPar
Um Daten, die in {\hyperref[\detokenize{toscience:_regal_api_2}]{\emph{regal\sphinxhyphen{}api}}} abgelegt wurden zur
Anzeige zu bringen sind i.d.R. mehrere Schritte nötig. Die genaue
Vorgehensweise ist davon abhängig, wo die Daten abgelegt werden (in
welchem Fedora Datenstrom). Grundsätzlich basiert die HTML\sphinxhyphen{}Darstellung
auf den Daten, die unter dem Format \sphinxcode{\sphinxupquote{.json2}} einer Ressource abrufbar
sind und einen Eintrag in context.json haben.

\begin{sphinxVerbatim}[commandchars=\\\{\}]
\PYG{o}{*}\PYG{o}{*}\PYG{n}{Daten} \PYG{n}{zur} \PYG{n}{Ansicht} \PYG{n}{bringen}\PYG{o}{*}\PYG{o}{*}
\end{sphinxVerbatim}
\begin{enumerate}
\sphinxsetlistlabels{\arabic}{enumi}{enumii}{}{.}%
\item {} 
\sphinxAtStartPar
Eintrag des zugehörigen RDF\sphinxhyphen{}Properties in die entsprechende
{\hyperref[\detokenize{toscience:_etikett}]{\emph{Etikett}}}\sphinxhyphen{}Instanz, bzw. in die \sphinxcode{\sphinxupquote{/conf/labels.json}}.
Der Eintrag muss einen Namen, ein Label und einen Datentyp haben.
{\hyperref[\detokenize{toscience:_regal_api_2}]{\emph{regal\sphinxhyphen{}api}}} neu starten, bzw mit
\sphinxcode{\sphinxupquote{POST /context.json}} das neu Laden der Contexteinträge erzwingen.

\item {} 
\sphinxAtStartPar
Dies müsste reichen, um eine Standardanzeige in der HTML\sphinxhyphen{}Ausgabe zu
erreichen

\item {} 
\sphinxAtStartPar
Wenn die Daten nicht erscheinen, sollte man überprüfen, ob sie unter
dem Format \sphinxcode{\sphinxupquote{.json2}} erscheinen. Wenn nicht, stellt sich die Frage,
wo die Daten abgelegt werden. Komplett werden nur die Daten aus dem
Fedora Datenstrom /metadata2 prozessiert. Befindet sich das Datum in
z.B. im /RELS\sphinxhyphen{}EXT Datenstrom so muss es zunächst manuell unter
\sphinxcode{\sphinxupquote{helper.JsonMapper\#getLd2()}} in das JSON\sphinxhyphen{}Objekt eingefügt werden.

\item {} 
\sphinxAtStartPar
Einige Felder werden auch ausgeblendet. Dies geschieht in
{\hyperref[\detokenize{toscience:_regal_api_2}]{\emph{regal\sphinxhyphen{}api}}} unter \sphinxcode{\sphinxupquote{/public/stylesheets/main.css}}
und in Drupal innerhalb der entsprechenden themes.

\item {} 
\sphinxAtStartPar
Um spezielle Anzeigen zu realisieren muss schließlich im
HTML\sphinxhyphen{}Template angefasst werden, unter
\sphinxcode{\sphinxupquote{/app/views/tags/resourceView.scala.html}} .

\end{enumerate}

\sphinxAtStartPar
Insgesamt läuft es also so: Alles was in {\hyperref[\detokenize{toscience:_etikett}]{\emph{Etikett}}}
konfiguriert ist, wird auch ins JSON und damit ins HTML und in den
Suchindex übernommen. Dinge, die im HTML nicht benötigt werden, werden
über CSS wieder ausgeblendet.


\chapter{Software}
\label{\detokenize{toscience:software}}\label{\detokenize{toscience:id18}}
\sphinxAtStartPar
Die technische Dokumentation der HTTP\sphinxhyphen{}Schnittstelle findet sich unter
{\hyperref[\detokenize{api:api-documentation}]{\sphinxcrossref{\DUrole{std,std-ref}{API\sphinxhyphen{}documentation}}}}:

\sphinxAtStartPar
Nachfolgend sei eine Innenansicht der einzelnen Module aufgestellt. Die
Integration der Module erfolgt i.d.R. über HTTPs. Die Module werden über
entsprechende Einträge in der Apache\sphinxhyphen{}Konfiguration sichtbar gemacht. Es
handelt sich also um eine Webservice\sphinxhyphen{}Architektur, in der alle
Webservices über einen Apache\sphinxhyphen{}Webserver und entsprechende Einträge in
ihren Konfigurationsdateien miteinander verbunden werden.

\begin{figure}[htbp]
\centering
\capstart

\noindent\sphinxincludegraphics{{regal-dependencies}.jpeg}
\caption{Regal Abhängigkeiten}\label{\detokenize{toscience:id94}}\end{figure}


\section{regal\sphinxhyphen{}api}
\label{\detokenize{toscience:regal-api-2}}\label{\detokenize{toscience:id19}}

\begin{savenotes}\sphinxattablestart
\centering
\sphinxcapstartof{table}
\sphinxthecaptionisattop
\sphinxcaption{Überblick}\label{\detokenize{toscience:id95}}
\sphinxaftertopcaption
\begin{tabulary}{\linewidth}[t]{|T|T|}
\hline

\sphinxAtStartPar
Source
&
\sphinxAtStartPar
\sphinxhref{https://github.com/edoweb/regal-api}{regal\sphinxhyphen{}api}
\\
\hline
\sphinxAtStartPar
Technik
&
\sphinxAtStartPar
\sphinxhref{https://www.playframework.com/documentation/2.4.x/JavaHome}{Play
2.4
.2}
\\
\hline
\sphinxAtStartPar
Ports
&
\sphinxAtStartPar
9000 / 9100
\\
\hline
\sphinxAtStartPar
Verzeichnis
&
\sphinxAtStartPar
/opt/regal/apps/regal ,
/opr/regal/src/regal
\\
\hline
\sphinxAtStartPar
HTTP Pfad
&
\sphinxAtStartPar
/
\\
\hline
\end{tabulary}
\par
\sphinxattableend\end{savenotes}

\sphinxAtStartPar
Mit regal\sphinxhyphen{}api werden alle grundlegenden Funktionen von Regal
bereitgestellt. Dies umfasst:
\begin{itemize}
\item {} 
\sphinxAtStartPar
HTTP Schnittstelle

\item {} 
\sphinxAtStartPar
Sichtbarkeiten, Zugriffskontrolle, Rollen

\item {} 
\sphinxAtStartPar
Speicherung, Datenhaltung

\item {} 
\sphinxAtStartPar
Konvertierungen

\item {} 
\sphinxAtStartPar
Ansichten

\item {} 
\sphinxAtStartPar
Suche

\item {} 
\sphinxAtStartPar
Webarchivierung

\end{itemize}

\sphinxAtStartPar
Der Webservice ist auf Basis von \sphinxhref{https://www.playframework.com/documentation/2.4.x/JavaHome}{Play
2.4.2}
realisiert und bietet eine reichhaltig HTTP\sphinxhyphen{}API zur Verwaltung von
elektronischen Publikationen an. Die {\hyperref[\detokenize{toscience:_regal_api_2}]{\emph{regal\sphinxhyphen{}api}}}
operiert auf {\hyperref[\detokenize{toscience:_fedora_commons_3}]{\emph{Fedora Commons 3}}},
{\hyperref[\detokenize{toscience:_mysql}]{\emph{MySql}}} und {\hyperref[\detokenize{toscience:_elasticsearch_1_1}]{\emph{Elasticsearch 1.1}}}.
Über die API werden auch Funktionalitäten von {\hyperref[\detokenize{toscience:_etikett}]{\emph{Etikett}}},
{\hyperref[\detokenize{toscience:_thumby}]{\emph{Thumby}}}, {\hyperref[\detokenize{toscience:_zettel}]{\emph{Zettel}}} und
{\hyperref[\detokenize{toscience:_deepzoomer}]{\emph{Deepzoomer}}} angesprochen. Für die Webarchivierung
werden {\hyperref[\detokenize{toscience:_heritrix}]{\emph{heritrix}}}, {\hyperref[\detokenize{toscience:_wpull}]{\emph{wpull}}} und
{\hyperref[\detokenize{toscience:_openwayback}]{\emph{openwayback}}} angebunden.


\subsection{Konfiguration}
\label{\detokenize{toscience:konfiguration}}\label{\detokenize{toscience:id20}}

\begin{savenotes}\sphinxattablestart
\centering
\sphinxcapstartof{table}
\sphinxthecaptionisattop
\sphinxcaption{Dateien im /conf Verzeichnis}\label{\detokenize{toscience:id96}}
\sphinxaftertopcaption
\begin{tabulary}{\linewidth}[t]{|T|T|}
\hline
\sphinxstyletheadfamily 
\sphinxAtStartPar
Datei
&\sphinxstyletheadfamily 
\sphinxAtStartPar
Beschreibung
\\
\hline
\sphinxAtStartPar
\sphinxstylestrong{aggregations.conf}
&
\sphinxAtStartPar
Diese Datei wird verwendet um die
Schnittstelle \sphinxcode{\sphinxupquote{/browse}} zu
konfigurieren. Die Einträg im
Object “aggs” können direkt über
die \sphinxcode{\sphinxupquote{/browse}} Schnittstelle
angesprochen werden. Mit Hilfe
des Elasticsearch\sphinxhyphen{}Indexes wird
dann eine entsprechende Antwort
generiert. Beispiel:
\sphinxcode{\sphinxupquote{/browse/rdftype}} liefert eine
Liste mit allen
Publikationstypen, die im Index
vorhanden sind.
\\
\hline
\sphinxAtStartPar
\sphinxstylestrong{application.conf.tmpl}
&
\sphinxAtStartPar
Eine template Datei für die
Hauptkonfiguration von
{\hyperref[\detokenize{toscience:_regal_api_2}]{\emph{regal\sphinxhyphen{}api}}}.
Diese Datei sollte zur lokalen
Verwendung einmal nach
application.conf kopiert werden.
In der Datei sind alle Passwörter
auf \sphinxstyleemphasis{admin} gesetzt.
\\
\hline
\sphinxAtStartPar
crawler\sphinxhyphen{}beans.cxml
&
\sphinxAtStartPar
Die Datei wird verwendet, wenn im
Webarchivierungsmodul eine neue
Konfiguration für eine Webseite
angelegt wird.
\\
\hline
\sphinxAtStartPar
ehcache.xml
&
\sphinxAtStartPar
die Konfiguration der Ehcache
Komponente
\\
\hline
\sphinxAtStartPar
fedora\sphinxhyphen{}users.xml
&
\sphinxAtStartPar
deprecated \sphinxhyphen{} Zur Löschung
vorgeschlagen
\\
\hline
\sphinxAtStartPar
hbz\_edoweb\_url.txt
&
\sphinxAtStartPar
deprecated \sphinxhyphen{} Zur Löschung
vorgeschlagen
\\
\hline
\sphinxAtStartPar
html.html
&
\sphinxAtStartPar
deprecated \sphinxhyphen{} Zur Löschung
vorgeschlagen
\\
\hline
\sphinxAtStartPar
install.properties
&
\sphinxAtStartPar
deprecated \sphinxhyphen{} Zur Löschung
vorgeschlagen
\\
\hline
\sphinxAtStartPar
labels\sphinxhyphen{}edoweb.de
&
\sphinxAtStartPar
Labels für eine bestimmt
Regal\sphinxhyphen{}Instanz
\\
\hline
\sphinxAtStartPar
labels\sphinxhyphen{}for
\sphinxhyphen{}proceeding\sphinxhyphen{}and\sphinxhyphen{}researchData.json
&
\sphinxAtStartPar
deprecated \sphinxhyphen{} Zur Löschung
vorgeschlagen
\\
\hline
\sphinxAtStartPar
labels\sphinxhyphen{}lobid.json
&
\sphinxAtStartPar
deprecated \sphinxhyphen{} Zur Löschung
vorgeschlagen
\\
\hline
\sphinxAtStartPar
labels\sphinxhyphen{}publisso.de
&
\sphinxAtStartPar
Labels für eine bestimmte
Regal\sphinxhyphen{}Instanz
\\
\hline
\sphinxAtStartPar
\sphinxstylestrong{labels.json}
&
\sphinxAtStartPar
Eine sinnvolle
Startkonfiguration. Die Datei
wurde mit {\hyperref[\detokenize{toscience:_etikett}]{\emph{Etikett}}}
erzeugt. Beim Start von
{\hyperref[\detokenize{toscience:_regal_api_2}]{\emph{regal\sphinxhyphen{}api}}}
wird zunächst versucht eine
ähnliche Konfiguration direkt von
einer laufenden
{\hyperref[\detokenize{toscience:_etikett}]{\emph{Etikett}}}\sphinxhyphen{}Instanz
zu holen. Wenn dies nicht klappt,
wird auf die labels.json
zurückgegriffen.
\\
\hline
\sphinxAtStartPar
list.html
&
\sphinxAtStartPar
deprecated \sphinxhyphen{} Zur Löschung
vorgeschlagen
\\
\hline
\sphinxAtStartPar
logback.developer.xml
&
\sphinxAtStartPar
Eine logging Konfiguration. Ich
kopiere die immer nach
logback.developer.js.xml (in
.gitignore) und trage sie dann in
die application.conf ein. Auf
diese Weise kann ich an Loglevels
herumkonfigurieren ohne das in
diese Änderungen in die
Versionsverwaltung spielen zu
müssen.
\\
\hline
\sphinxAtStartPar
logback.xml
&
\sphinxAtStartPar
Konfiguration des Loggers. Diese
Datei ist in application.conf
eingetragen.
\\
\hline
\sphinxAtStartPar
mab
xml\sphinxhyphen{}string\sphinxhyphen{}template\sphinxhyphen{}on\sphinxhyphen{}record.xml
&
\sphinxAtStartPar
Eine template\sphinxhyphen{}Datei zur
Generierung von MAB\sphinxhyphen{}Ausgaben.
\\
\hline
\sphinxAtStartPar
mail.properties
&
\sphinxAtStartPar
Konfiguration zur Versendung von
Mails. Standardmäßig schickt die
Applikation eine Mail, sobald sie
im Production\sphinxhyphen{}Mode neu gestartet
wurde. Auch der Umzugsservice im
Webarchivierungsmodul verschickt
Mails.
\\
\hline
\sphinxAtStartPar
nwbib\sphinxhyphen{}spatial.ttl
&
\sphinxAtStartPar
deprecated \sphinxhyphen{} Zur Löschung
vorgeschlagen
\\
\hline
\sphinxAtStartPar
nwbib.ttl
&
\sphinxAtStartPar
deprecated \sphinxhyphen{} Zur Löschung
vorgeschlagen
\\
\hline
\sphinxAtStartPar
\sphinxstylestrong{public\sphinxhyphen{}index\sphinxhyphen{}config.json}
&
\sphinxAtStartPar
Konfiguration des
Elasticsearch\sphinxhyphen{}Indexes. Da in dem
Index vorallem Metadaten liegen,
soll fast nicht tokenisiert
werden.
\\
\hline
\sphinxAtStartPar
\sphinxstylestrong{routes}
&
\sphinxAtStartPar
Hier sind alle HTTP\sphinxhyphen{}Pfade
übersichtlich aufgeführt.
\\
\hline
\sphinxAtStartPar
scm\sphinxhyphen{}info.sh
&
\sphinxAtStartPar
Diese Datei kann man unter Linux
in die profile\sphinxhyphen{}Konfiguration
seines Benutzers einbinden. Dann
erhält man im Terminal farbige
Angabgen zu Git\sphinxhyphen{}Branches,etc.
\\
\hline
\sphinxAtStartPar
start\sphinxhyphen{}regal.sh
&
\sphinxAtStartPar
deprecated \sphinxhyphen{} Zur Löschung
vorgeschlagen
\\
\hline
\sphinxAtStartPar
tomcat\sphinxhyphen{}users.xml
&
\sphinxAtStartPar
deprecated \sphinxhyphen{} Zur Löschung
vorgeschlagen
\\
\hline
\sphinxAtStartPar
unescothes.ttl
&
\sphinxAtStartPar
deprecated \sphinxhyphen{} Zur Löschung
vorgeschlagen
\\
\hline
\sphinxAtStartPar
wglcontributor.csv
&
\sphinxAtStartPar
deprecated \sphinxhyphen{} Zur Löschung
vorgeschlagen
\\
\hline
\end{tabulary}
\par
\sphinxattableend\end{savenotes}


\subsection{Die Applikation}
\label{\detokenize{toscience:die-applikation}}\label{\detokenize{toscience:id21}}

\begin{savenotes}\sphinxattablestart
\centering
\sphinxcapstartof{table}
\sphinxthecaptionisattop
\sphinxcaption{Das /app Verzeichnis}\label{\detokenize{toscience:id97}}
\sphinxaftertopcaption
\begin{tabulary}{\linewidth}[t]{|T|T|}
\hline
\sphinxstyletheadfamily 
\sphinxAtStartPar
Package
&\sphinxstyletheadfamily 
\sphinxAtStartPar
Beschreibung
\\
\hline
\sphinxAtStartPar
default package
&
\sphinxAtStartPar
Hier befindet sich die Datei
Global, die in \sphinxhref{https://www.playframework.com/documentation/2.4.x/JavaHome}{Play
2
.4}
noch eine große Rolle spielt. In
der Datei können zum Beispiel
Aktionen vor dem Start der
Applikation erfolgen, auch können
hier HTTP\sphinxhyphen{}Requests mit geloggt
werden. Bestimmte Aktionen werden
nur im Production\sphinxhyphen{}Mode
ausgeführt, d.h. nur wenn die
Applikation mit \sphinxcode{\sphinxupquote{start}}
gestartet wurde oder über
\sphinxcode{\sphinxupquote{dist}} ein entsprechendes
Binary erzeugt wurde.
\\
\hline
\sphinxAtStartPar
actions
&
\sphinxAtStartPar
Hier sind Funktionen versammelt,
die meist unmittelbar aus den
Controller\sphinxhyphen{}Klassen aufgerufen
werden.
\\
\hline
\sphinxAtStartPar
archive.fedora
&
\sphinxAtStartPar
Ein Reihe von Dateien, über die
Zugriffe auf {\hyperref[\detokenize{toscience:_fedora_commons_3}]{\emph{Fedora Commons
3}}}
organisiert werden. Hier finden
sich auch einige Hilfsklassen
(\sphinxcode{\sphinxupquote{Utils}}). Das FedoraInterface
zeigt an, welche Aktionen auf der
Fedora ausgeführt werden. Der
Code in diesem Paket gehört mit
zu dem ältesten Code im gesamten
Regal\sphinxhyphen{}Projekt.
\\
\hline
\sphinxAtStartPar
archive.search
&
\sphinxAtStartPar
Zugriff auf die Elasticsearch
\\
\hline
\sphinxAtStartPar
authenticate
&
\sphinxAtStartPar
Regal verwendet Basic\sphinxhyphen{}Auth zur
Authentifizierung. Um die
entsprechenden Aufrufe in den
Controllern zu Schützen wird eine
Annotation \sphinxcode{\sphinxupquote{@BasicAuth}}
verwendet. Diese findet sich
hier. Die Annotation selbst
bewirkt, dass jeder
Controller\sphinxhyphen{}Aufruf durch die
Methode \sphinxcode{\sphinxupquote{basicAuth}} der Klasse
\sphinxcode{\sphinxupquote{BasicAuthAction.java}} läuft.
Ziel dieser Prozedur ist es, dem
aktuellen Zugriff die
Berechtigungen einer bestimmten
Rolle zuzuordnen.
\\
\hline
\sphinxAtStartPar
controllers
&
\sphinxAtStartPar
Der Code, der in diesen Klassen
organisiert ist, wird bei den
entsprechenden HTTP\sphinxhyphen{}Aufrufen
ausgeführt. In der
\sphinxcode{\sphinxupquote{/conf/routes}} Datei kann man
sehen, welcher HTTP\sphinxhyphen{}Aufruf,
welchen Methoden\sphinxhyphen{}Aufruf zur Folge
hat. Die Controller\sphinxhyphen{}Klassen sind
i.d.R. von der Klasse
MyController abgeleitet, die
Hilfsfunktionen bereitstellt,
aber auch Funktionen zur
Überprüfung von Zugriffsrechten.
Die Überprüfung von
Zugriffsrechten erfolgt durch
eingebettet Calls und wird über
die internen Klassen von
MyController realisiert.
Beispiel: Die Funktion
\sphinxcode{\sphinxupquote{listNodes}} in der Klasse
\sphinxcode{\sphinxupquote{controllers.Resource}} ruft
ihre Prozeduren eingebettet in
eine Funktion der Klasse
\sphinxcode{\sphinxupquote{ListAction}} auf. Die Klasse
\sphinxcode{\sphinxupquote{ListAction}} ist in
\sphinxcode{\sphinxupquote{MyController}} implementiert
und überprüft, ob der Aufruf mit
der nötigen Berechtigung
erfolgte. Vgl.
{\hyperref[\detokenize{toscience:_zugriffsberechtigungen_und_sichtbarkeiten}]{\emph{Zugriffsberechtigungen und
Sichtbarkeiten}}}
\\
\hline
\sphinxAtStartPar
converter.mab
&
\sphinxAtStartPar
Diese Datei realisiert das
OAI\sphinxhyphen{}Providing von MAB\sphinxhyphen{}Daten.
Ursprünglich war geplant,
wesentlich umfangreichere
MAB\sphinxhyphen{}Datensätze an den
Verbundkatalog zu liefern. Daher
wird hier mit einer eigenen
Template\sphinxhyphen{}Engine gearbeitet, etc.
Ein lustiges Produkt in diesem
Kontext ist auch die Klasse
\sphinxcode{\sphinxupquote{models.MabRecord}}.
\\
\hline
\sphinxAtStartPar
de.hbz.lobid.helper
&
\sphinxAtStartPar
Der hier befindliche Code kommt
ursprünglich aus einem anderen
Paket, wurde dann aber beim
Neuaufbau des Lobid 2
Datendienstes gemeinsam mit den
Kollegen weiterentwickelt und ist
schließlich wieder hier gelandet.
Mittlerweile ist die offizielle
JSON\sphinxhyphen{}LD\sphinxhyphen{}Library soweit
entwickelt, dass man die
Konvertierung auch darüber machen
kann. Achja, denn dafür ist der
Code: Lobid N\sphinxhyphen{}Triples in schönes
JSON umzuformen, das dann auch in
den Elasticsearch\sphinxhyphen{}Index kann.
\\
\hline
\sphinxAtStartPar
helper
&
\sphinxAtStartPar
Die mit Abstand wichtigste Klasse
in diesem Package heißt
\sphinxcode{\sphinxupquote{JsonMapper}}. Hier wird das
JSON für Index und Ansichten
erzeugt.
\\
\hline
\sphinxAtStartPar
helper.mail
&
\sphinxAtStartPar
Emails verschicken.
\\
\hline
\sphinxAtStartPar
helper.oai
&
\sphinxAtStartPar
Einige Klassen zur Regelung des
OAI\sphinxhyphen{}Providings. Der
\sphinxcode{\sphinxupquote{OAIDispatcher}} analysiert, ob
und wie ein \sphinxcode{\sphinxupquote{Node}} an die
OAI\sphinxhyphen{}Schnittstelle gelangt.
\\
\hline
\sphinxAtStartPar
models
&
\sphinxAtStartPar
Die wichtigste Klasse hier ist
\sphinxcode{\sphinxupquote{Node}} über diese Klasse läuft
der Großteil des
Datentransportes.
\\
\hline
\sphinxAtStartPar
views
&
\sphinxAtStartPar
Templates in der Sprache
\sphinxcode{\sphinxupquote{Twirl}} und einige
Java\sphinxhyphen{}Hilfsklassen.
\\
\hline
\sphinxAtStartPar
views.mediaViewers
&
\sphinxAtStartPar
Ein paar Viewer, die über die
Hilfsklasse \sphinxcode{\sphinxupquote{ViewerInfo}} in
\sphinxcode{\sphinxupquote{tags.resourceView}} eingebunden
werden können.
\\
\hline
\sphinxAtStartPar
views.oai
&
\sphinxAtStartPar
Mit \sphinxcode{\sphinxupquote{Twirl}} XML zu generieren
war keine gute Idee.
\\
\hline
\sphinxAtStartPar
views.tags
&
\sphinxAtStartPar
Hilfstemplates.
\\
\hline
\end{tabulary}
\par
\sphinxattableend\end{savenotes}


\section{Etikett}
\label{\detokenize{toscience:etikett}}\label{\detokenize{toscience:id22}}

\begin{savenotes}\sphinxattablestart
\centering
\sphinxcapstartof{table}
\sphinxthecaptionisattop
\sphinxcaption{Überblick}\label{\detokenize{toscience:id98}}
\sphinxaftertopcaption
\begin{tabulary}{\linewidth}[t]{|T|T|}
\hline

\sphinxAtStartPar
Source
&
\sphinxAtStartPar
\sphinxhref{https://github.com/hbz/etikett}{etikett}
\\
\hline
\sphinxAtStartPar
Technik
&
\sphinxAtStartPar
\sphinxhref{https://www.playframework.com/documentation/2.2.x/JavaHome}{Play Play
2.2
.2}
\\
\hline
\sphinxAtStartPar
Ports
&
\sphinxAtStartPar
9002 / 9102
\\
\hline
\sphinxAtStartPar
Verzeichnis
&
\sphinxAtStartPar
/opt/regal/apps/etikett ,
/opr/regal/src/etikett
\\
\hline
\sphinxAtStartPar
HTTP Pfad
&
\sphinxAtStartPar
/tools/etikett
\\
\hline
\end{tabulary}
\par
\sphinxattableend\end{savenotes}

\sphinxAtStartPar
Etikett ist eine einfache Datenbankanwendung, die es erlaubt
\begin{enumerate}
\sphinxsetlistlabels{\arabic}{enumi}{enumii}{}{.}%
\item {} 
\sphinxAtStartPar
Menschenlesbare Labels für URIs abzulegen. Über eine
HTTP\sphinxhyphen{}Schnittstelle kann dann nach dem Label gefragt werden.

\item {} 
\sphinxAtStartPar
Auch Konfigurationen zur Erzeugung eines JSON\sphinxhyphen{}LD Kontextes können
abgelegt werden.

\item {} 
\sphinxAtStartPar
Die Etikett\sphinxhyphen{}Datenbank erweitert sich dynamisch. Wird in einem
authentifizierten Zugriff nach einer noch nicht bekannten URI
gefragt, so versucht die Applikation ein Label für die URI zu finden.

\end{enumerate}

\sphinxAtStartPar
In Etikett sind verschiedene Lookups realisiert, die dynamisch Labels
für URIs finden können. Beispiele:
\begin{itemize}
\item {} 
\sphinxAtStartPar
Crossref

\item {} 
\sphinxAtStartPar
Geonames

\item {} 
\sphinxAtStartPar
GND

\item {} 
\sphinxAtStartPar
Openstreetmap

\item {} 
\sphinxAtStartPar
Orcid

\item {} 
\sphinxAtStartPar
RDF, Skos, etc.

\end{itemize}

\sphinxAtStartPar
Fragt man Etikett nach einem Label, so antwortet Etikett mit dem
Ergebnis des Lookups. Wenn Etikett nicht in der Lage ist, ein Label zu
finden, wird die URI, mit angefragt wurde, zurückgegeben.

\sphinxAtStartPar
Etikett kann auch als Cache verwendet werden. So werden authentifizierte
Anfragen in einer Datenbank persistiert. Erneute Anfragen werden dann
aus der Datenbank beantwortet, ein erneuter Lookup wird eingespart.
Einmal persistierte Labels werden nicht invalidiert. Die Invalidierung
kann von außerhalb über authentifizierte HTTP\sphinxhyphen{}Zugriffe realisiert
werden, stellt aber insgesamt noch ein Desiderat dar.

\sphinxAtStartPar
Etikett kann auch mit Labels vorkonfiguriert werden. Dabei können
zusätzliche Informationen zu jeder URIs mit abgelegt werden. Folgende
Informationen können in etikett abgelegt werden:
\begin{itemize}
\item {} 
\sphinxAtStartPar
URI

\item {} 
\sphinxAtStartPar
Label

\item {} 
\sphinxAtStartPar
Weight \sphinxhyphen{} Zur Definition von Anzeigereihenfolgen.

\item {} 
\sphinxAtStartPar
Pictogram Iconfont\sphinxhyphen{}ID \sphinxhyphen{} Kann anstatt oder zusätzlich zum Label
angezeigt werden.

\item {} 
\sphinxAtStartPar
ReferenceType \sphinxhyphen{} JSON\sphinxhyphen{}LD Typ

\item {} 
\sphinxAtStartPar
Container \sphinxhyphen{} JSON\sphinxhyphen{}LD Container

\item {} 
\sphinxAtStartPar
Beschreibung \sphinxhyphen{} Kommentar als Markdown

\end{itemize}

\begin{figure}[htbp]
\centering
\capstart

\noindent\sphinxincludegraphics{{etikett-screen}.png}
\caption{Etikett Oberfläche}\label{\detokenize{toscience:id99}}\end{figure}

\sphinxAtStartPar
Mit Hilfe dieser Angaben kann Etikett auch einen “JSON\sphinxhyphen{}LD Context”
bereitstellen. Insgesamt wird über Etikett eine Art “Application
Profile” realisiert. Das Profil gibt Auskunft, welche Metadatenfelder
(definiert als URIs) in welcher Weise (Typ, Container) Verwendung finden
und wie sie angezeigt werden sollen (Label, Weight, Pictogram).

\sphinxAtStartPar
Im Regal\sphinxhyphen{}Kontext wird {\hyperref[\detokenize{toscience:_etikett}]{\emph{Etikett}}} an vielen Stellen
verwendet.
\begin{itemize}
\item {} 
\sphinxAtStartPar
Zur Wandlung von RDF nach JSON\sphinxhyphen{}LD

\item {} 
\sphinxAtStartPar
Zur Anreicherung von RDF Importen

\item {} 
\sphinxAtStartPar
Zur menschenlesbaren Darstellung von RDF

\item {} 
\sphinxAtStartPar
Zur Konfiguration von Labels, Anzeigereihenfolgen und Pictogrammen

\item {} 
\sphinxAtStartPar
Als Cache

\end{itemize}


\subsection{Konfiguration}
\label{\detokenize{toscience:konfiguration-2}}\label{\detokenize{toscience:id23}}

\begin{savenotes}\sphinxattablestart
\centering
\sphinxcapstartof{table}
\sphinxthecaptionisattop
\sphinxcaption{Dateien im /conf Verzeichnis}\label{\detokenize{toscience:id100}}
\sphinxaftertopcaption
\begin{tabulary}{\linewidth}[t]{|T|T|}
\hline
\sphinxstyletheadfamily 
\sphinxAtStartPar
Datei
&\sphinxstyletheadfamily 
\sphinxAtStartPar
Beschreibung
\\
\hline
\sphinxAtStartPar
\sphinxstylestrong{evolutions}
&
\sphinxAtStartPar
Dieses Verzeichnis enthält
SQL\sphinxhyphen{}Skripte, die bei Änderungen
des Datenbankschemas automatisch
über EBean angelegt werden. Beim
nächsten Deployment einer neuen
Etikett\sphinxhyphen{}Version werden die
Skripte automatische angewendet.
Die Skripte enthalten immer einen
mit “Up” markierten Part, und
einen mit “Down” markierten Part
(für rollbacks).
\\
\hline
\sphinxAtStartPar
\sphinxstylestrong{application.conf}
&
\sphinxAtStartPar
Hier kann ein Benutzer
eingestellt werden. Alle Klassen
im Verzeichnis \sphinxcode{\sphinxupquote{models.*}}
erhalten eine SQL\sphinxhyphen{}Tabelle.
\\
\hline
\sphinxAtStartPar
ddc.turtle
&
\sphinxAtStartPar
Eine DDC Datei. Die Datei bietet
Labels für DDC\sphinxhyphen{}URIs an.
\\
\hline
\sphinxAtStartPar
labels.json
&
\sphinxAtStartPar
Eine Labels\sphinxhyphen{}Datei, die zur
initialen Befüllung verwendet
werden kann.
\\
\hline
\sphinxAtStartPar
regal.turtle
&
\sphinxAtStartPar
Eine Labels\sphinxhyphen{}Datei, die zur
initialen Befüllung verwendet
werden kann.
\\
\hline
\sphinxAtStartPar
\sphinxstylestrong{routes}
&
\sphinxAtStartPar
Alle HTTP\sphinxhyphen{}Schnittstellen
übersichtlich in einer Datei
\\
\hline
\sphinxAtStartPar
rpb.turtle
&
\sphinxAtStartPar
Eine Labels\sphinxhyphen{}Datei, die zur
initialen Befüllung verwendet
werden kann.
\\
\hline
\sphinxAtStartPar
rpb2.turtle
&
\sphinxAtStartPar
Eine Labels\sphinxhyphen{}Datei, die zur
initialen Befüllung verwendet
werden kann.
\\
\hline
\end{tabulary}
\par
\sphinxattableend\end{savenotes}


\subsection{Die Applikation}
\label{\detokenize{toscience:die-applikation-2}}\label{\detokenize{toscience:id24}}

\begin{savenotes}\sphinxattablestart
\centering
\sphinxcapstartof{table}
\sphinxthecaptionisattop
\sphinxcaption{Das /app Verzeichnis}\label{\detokenize{toscience:id101}}
\sphinxaftertopcaption
\begin{tabulary}{\linewidth}[t]{|T|T|}
\hline
\sphinxstyletheadfamily 
\sphinxAtStartPar
Package
&\sphinxstyletheadfamily 
\sphinxAtStartPar
Beschreibung
\\
\hline
\sphinxAtStartPar
default
&
\sphinxAtStartPar
In \sphinxcode{\sphinxupquote{Global}} werden die Requests
mit geloggt.
\\
\hline
\sphinxAtStartPar
controllers
&
\sphinxAtStartPar
In \sphinxcode{\sphinxupquote{Application}} werden alle
HTTP\sphinxhyphen{}Operationen implementiert.
Unterstützt wird BasicAuth.
\\
\hline
\sphinxAtStartPar
helper
&
\sphinxAtStartPar
Verschiedene Klassen, die eine
URI verfolgen und versuchen ein
Label aus den zurückgelieferten
Daten zu kreieren.
\\
\hline
\sphinxAtStartPar
models
&
\sphinxAtStartPar
Das Model \sphinxcode{\sphinxupquote{Etikett}} ist
persistierbar.
\\
\hline
\sphinxAtStartPar
views
&
\sphinxAtStartPar
Die meisten HTTP\sphinxhyphen{}Operationen
lassen sich auch über eine
Weboberfläche im Browser
aufrufen.
\\
\hline
\end{tabulary}
\par
\sphinxattableend\end{savenotes}


\section{Zettel}
\label{\detokenize{toscience:zettel}}\label{\detokenize{toscience:id25}}

\begin{savenotes}\sphinxattablestart
\centering
\sphinxcapstartof{table}
\sphinxthecaptionisattop
\sphinxcaption{Überblick}\label{\detokenize{toscience:id102}}
\sphinxaftertopcaption
\begin{tabulary}{\linewidth}[t]{|T|T|}
\hline

\sphinxAtStartPar
Source
&
\sphinxAtStartPar
{\color{red}\bfseries{}\textasciigrave{}zettel \textless{}
https://github.com/hbz/zettel\textgreater{}\textasciigrave{}\_\_}
\\
\hline
\sphinxAtStartPar
Technik
&
\sphinxAtStartPar
\sphinxhref{https://www.playframework.com/documentation/2.5.x/JavaHome}{Play Play
2.5
.4}
\\
\hline
\sphinxAtStartPar
Ports
&
\sphinxAtStartPar
9003 / 9103
\\
\hline
\sphinxAtStartPar
Verzeichnis
&
\sphinxAtStartPar
/opt/regal/apps/zettel,
/opr/regal/src/zettel
\\
\hline
\sphinxAtStartPar
HTTP Pfad
&
\sphinxAtStartPar
/tools/zettel
\\
\hline
\end{tabulary}
\par
\sphinxattableend\end{savenotes}

\sphinxAtStartPar
Zettel ist ein Webservice zur Bereitstellung von Webformularen. Die
Webformulare können über ein HTTP\sphinxhyphen{}GET geladen werden. Sollen
existierende Daten in ein Formular geladen werden, so können diese Daten
(1) als Form\sphinxhyphen{}encoded, (2) als JSON, oder (3) als RDF\sphinxhyphen{}XML über ein
\sphinxcode{\sphinxupquote{HTTP POST}} in das Formular geladen werden. Gleichzeitig kann
spezifiziert werden, in welchem Format das Formular Daten zurückliefern
soll.

\begin{figure}[htbp]
\centering
\capstart

\noindent\sphinxincludegraphics{{zettel-screen}.png}
\caption{Zettel Oberfläche}\label{\detokenize{toscience:id103}}\end{figure}

\sphinxAtStartPar
Zettel verfügt über keine eigene Speicherschicht. Daten die über ein
Formular erzeugt wurden, werden in der HTTP\sphinxhyphen{}Response zurückgeliefert.
Zur Integration von Zettel in andere Applikationen wurde ein
Kommunikationspattern entwickelt, das auf Javascript beruht. Das
Zettel\sphinxhyphen{}Formular wird hierzu in einem IFrame in die Applikation
eingebunden. Die Applikation muss außerdem ein Javascript einbinden, das
auf bestimmte Nachrichten aus dem IFrame lauscht. Bei bestimmte Aktionen
sendet das Zettel\sphinxhyphen{}Formular dann Nachrichten an die Applikation und
erlaubt dieser darauf zu reagieren. Um Daten von Zettel in die
Applikation zu bekommen, werden diese im HTML\sphinxhyphen{}DOM gespeichert und können
von dort durch die Applikation entgegengenommen werden.

\begin{figure}[htbp]
\centering
\capstart

\noindent\sphinxincludegraphics{{zettel-flos}.png}
\caption{Zettel Datenfluss}\label{\detokenize{toscience:id104}}\end{figure}


\subsection{Konfiguration}
\label{\detokenize{toscience:konfiguration-3}}\label{\detokenize{toscience:id26}}

\begin{savenotes}\sphinxattablestart
\centering
\sphinxcapstartof{table}
\sphinxthecaptionisattop
\sphinxcaption{Dateien im /conf Verzeichnis}\label{\detokenize{toscience:id105}}
\sphinxaftertopcaption
\begin{tabulary}{\linewidth}[t]{|T|T|}
\hline
\sphinxstyletheadfamily 
\sphinxAtStartPar
Datei
&\sphinxstyletheadfamily 
\sphinxAtStartPar
Beschreibung
\\
\hline
\sphinxAtStartPar
\sphinxstylestrong{application.conf}
&
\sphinxAtStartPar
Die Datei enthält einen Eintrag
zur Konfiguration von
{\hyperref[\detokenize{toscience:_etikett}]{\emph{Etikett}}}. Über
einen weiteren Eintrag können
“Hilfetexte” angelinkt werden.
Die Hilfetexte müssen in einer
statischen HTML abgelegt sein. Am
Ende der Datei werden einige
Limits deutlich über den Standard
erhöht, damit die großen
RDF\sphinxhyphen{}Posts auch funktionieren.
\\
\hline
\sphinxAtStartPar
\sphinxstylestrong{collectionOne.csv}
&
\sphinxAtStartPar
Die Datei regelt den Inhalt eines
Combo\sphinxhyphen{}Box widgets mit id
collectionOne.
\\
\hline
\sphinxAtStartPar
\sphinxstylestrong{ddc.csv}
&
\sphinxAtStartPar
Die Datei regelt den Inhalt eines
Combo\sphinxhyphen{}Box widgets mit id ddc.
\\
\hline
\sphinxAtStartPar
labels.json
&
\sphinxAtStartPar
Ein paar labels, falls keine
Instanz von
{\hyperref[\detokenize{toscience:_etikett}]{\emph{Etikett}}}
erreichbar ist.
\\
\hline
\sphinxAtStartPar
logback.xml
&
\sphinxAtStartPar
Logger Konfiguration.
\\
\hline
\sphinxAtStartPar
\sphinxstylestrong{professionalGroup.csv}
&
\sphinxAtStartPar
Die Datei regelt den Inhalt eines
Combo\sphinxhyphen{}Box widgets mit id
professionalGroup.
\\
\hline
\sphinxAtStartPar
routes
&
\sphinxAtStartPar
Alle HTTP\sphinxhyphen{}Pfade übersichtlich in
einer Datei
\\
\hline
\end{tabulary}
\par
\sphinxattableend\end{savenotes}


\subsection{Die Applikation}
\label{\detokenize{toscience:die-applikation-3}}\label{\detokenize{toscience:id27}}

\begin{savenotes}\sphinxattablestart
\centering
\sphinxcapstartof{table}
\sphinxthecaptionisattop
\sphinxcaption{Das /app Verzeichnis}\label{\detokenize{toscience:id106}}
\sphinxaftertopcaption
\begin{tabulary}{\linewidth}[t]{|T|T|}
\hline
\sphinxstyletheadfamily 
\sphinxAtStartPar
Package
&\sphinxstyletheadfamily 
\sphinxAtStartPar
Beschreibung
\\
\hline
\sphinxAtStartPar
controllers
&
\sphinxAtStartPar
Es gibt nur einen Controller.
Hier ist sowohl die
Basisfunktionalität
implementiert, als auch die
Autocompletion\sphinxhyphen{}Endpunkte für die
unterschiedlichen Widgets. Die
Schnittstelle zu Abhandlung von
Formulardaten ist recht generisch
gehalten. Über eine ID wird das
entsprechende Formular aus dem
services.ZettelRegister geholt
und das zugehörige Formular wird
gerendert. Die Formular erhalten
dabei unterschiedliche Templates
(z.B. \sphinxcode{\sphinxupquote{views.Article}}) und
unterschiedliche Modelklassen
(z.B. models.Article).
\\
\hline
\sphinxAtStartPar
models
&
\sphinxAtStartPar
Das Model “Article” heißt aus
historischen Gründen so.
Tatsächlich können mittlerweile
auch Kongressschriften und
Buchkapitel darüber abgebildet
werden (vermutlich wird sich der
Name nochmal ändern). Das Model
“Catalog” dient zum Import von
Daten aus dem Aleph\sphinxhyphen{}Katalog (über
Lobid). Mit ResearchData steht
ein prototypisches Model zur
Verarbeitung von Daten über
Forschungsdaten zur Verfügung.
Alle Models basieren auf einem
einzigen “fetten” ZettelModel.
Das ZettelModel enthält auch
Funktionen zur De/Serialisierung
in RDF und Json.
\\
\hline
\sphinxAtStartPar
services
&
\sphinxAtStartPar
Hier werden verschiedene
Hilfsklassen versammelt. Die
Klasse ZettelFields enthält ein
Mapping zur RDF\sphinxhyphen{}Deserialisierung.
\\
\hline
\sphinxAtStartPar
views
&
\sphinxAtStartPar
Alle HTML\sphinxhyphen{}Sichten und die
eigentlichen Formulare.
\\
\hline
\end{tabulary}
\par
\sphinxattableend\end{savenotes}


\section{skos\sphinxhyphen{}lookup}
\label{\detokenize{toscience:skos-lookup}}\label{\detokenize{toscience:id28}}

\begin{savenotes}\sphinxattablestart
\centering
\sphinxcapstartof{table}
\sphinxthecaptionisattop
\sphinxcaption{Überblick}\label{\detokenize{toscience:id107}}
\sphinxaftertopcaption
\begin{tabulary}{\linewidth}[t]{|T|T|}
\hline

\sphinxAtStartPar
Source
&
\sphinxAtStartPar
\sphinxhref{https://github.com/hbz/skos-lookup}{skos\sphinxhyphen{}lookup}
\\
\hline
\sphinxAtStartPar
Technik
&
\sphinxAtStartPar
\sphinxhref{https://www.playframework.com/documentation/2.5.x/JavaHome}{Play Play
2.5
.8}
\\
\hline
\sphinxAtStartPar
Ports
&
\sphinxAtStartPar
9004 / 9104
\\
\hline
\sphinxAtStartPar
Verzeichnis
&
\sphinxAtStartPar
/opt/regal/apps/skos\sphinxhyphen{}lookup,
/opr/regal/src/skos\sphinxhyphen{}lookup
\\
\hline
\sphinxAtStartPar
HTTP Pfad
&
\sphinxAtStartPar
/tools/skos\sphinxhyphen{}lookup
\\
\hline
\end{tabulary}
\par
\sphinxattableend\end{savenotes}

\sphinxAtStartPar
{\hyperref[\detokenize{toscience:_skos_lookup}]{\emph{skos\sphinxhyphen{}lookup}}} dient zur Unterstützung von
{\hyperref[\detokenize{toscience:_zettel}]{\emph{Zettel}}}. Der Webservice startet eine eingebettete
Elasticsearch\sphinxhyphen{}Instanz und verfügt über eine Schnittstelle um SKOS\sphinxhyphen{}Daten
in separate Indexe zu importieren und Schnittstellen zur Unterstützung
von jQuery\sphinxhyphen{}Autocomplete\sphinxhyphen{} und Select2\sphinxhyphen{}Widgets aufzubauen. Auf diese Weise
können auch umfangreichere Thesauri und Notationssysteme in den
Formularen von {\hyperref[\detokenize{toscience:_zettel}]{\emph{Zettel}}} fachgerecht angelinkt werden.
{\hyperref[\detokenize{toscience:_skos_lookup}]{\emph{skos\sphinxhyphen{}lookup}}} unterstützt auch mehrsprachige Thesauri.

\begin{figure}[htbp]
\centering
\capstart

\noindent\sphinxincludegraphics{{skos-lookup-autocomplete}.png}
\caption{SKOS\sphinxhyphen{}Lookup Beispiel 1}\label{\detokenize{toscience:id108}}\end{figure}

\begin{figure}[htbp]
\centering
\capstart

\noindent\sphinxincludegraphics{{example-select2}.png}
\caption{SKOS\sphinxhyphen{}Lookup Beispiel 2}\label{\detokenize{toscience:id109}}\end{figure}


\subsection{Konfiguration}
\label{\detokenize{toscience:konfiguration-4}}\label{\detokenize{toscience:id29}}

\begin{savenotes}\sphinxattablestart
\centering
\sphinxcapstartof{table}
\sphinxthecaptionisattop
\sphinxcaption{Dateien im /conf Verzeichnis}\label{\detokenize{toscience:id110}}
\sphinxaftertopcaption
\begin{tabulary}{\linewidth}[t]{|T|T|}
\hline
\sphinxstyletheadfamily 
\sphinxAtStartPar
Datei
&\sphinxstyletheadfamily 
\sphinxAtStartPar
Beschreibung
\\
\hline
\sphinxAtStartPar
\sphinxstylestrong{application.conf}
&
\sphinxAtStartPar
Hier wird der interne
Elasticsearch\sphinxhyphen{}Index konfiguriert.
Auch werden einige
Speichereinstellungen erhöht.
Damit auch große SKOS\sphinxhyphen{}Dateien
geladen werden können, sollten
auch die Java\sphinxhyphen{}Opts erhöht werden.
\\
\hline
\sphinxAtStartPar
logback.xml
&
\sphinxAtStartPar
Logger Konfiguration
\\
\hline
\sphinxAtStartPar
routes
&
\sphinxAtStartPar
Alle HTTP\sphinxhyphen{}Pfade übersichtlich in
einer Datei
\\
\hline
\sphinxAtStartPar
skos\sphinxhyphen{}context.json
&
\sphinxAtStartPar
Ein JSON\sphinxhyphen{}LD\sphinxhyphen{}Kontext zur
Umwandlung von RDF nach JSON.
(Origianl von: Jakob Voss)
\\
\hline
\sphinxAtStartPar
skos\sphinxhyphen{}setting.json
&
\sphinxAtStartPar
Settings zur Konfiguration des
Elasticsearchindexse. (Original
von: Jörg Prante)
\\
\hline
\end{tabulary}
\par
\sphinxattableend\end{savenotes}


\subsection{Die Applikation}
\label{\detokenize{toscience:die-applikation-4}}\label{\detokenize{toscience:id30}}

\begin{savenotes}\sphinxattablestart
\centering
\sphinxcapstartof{table}
\sphinxthecaptionisattop
\sphinxcaption{Das /app Verzeichnis}\label{\detokenize{toscience:id111}}
\sphinxaftertopcaption
\begin{tabulary}{\linewidth}[t]{|T|T|}
\hline
\sphinxstyletheadfamily 
\sphinxAtStartPar
Package
&\sphinxstyletheadfamily 
\sphinxAtStartPar
Beschreibung
\\
\hline
\sphinxAtStartPar
controllers
&
\sphinxAtStartPar
Alles in einem Controller. Die
API\sphinxhyphen{}Methoden liefern HTML und
JSON, so dass man sie aus dem
Browser, aber auch über andere
Tools ansprechen kann.
\\
\hline
\sphinxAtStartPar
elasticsearch
&
\sphinxAtStartPar
Eine embedded Elasticsearch. Dies
hat den Vorteil, dass man eine
aktuellere Version nutzen kann,
als z.B. die
{\hyperref[\detokenize{toscience:_regal_api_2}]{\emph{regal\sphinxhyphen{}api}}}.
\\
\hline
\sphinxAtStartPar
services
&
\sphinxAtStartPar
Hilfsklassen zum Laden der Daten.
\\
\hline
\sphinxAtStartPar
views
&
\sphinxAtStartPar
Ein Formular um neue Daten in die
Applikation zu laden. Und ein
Beispielformular zur
Demonstration der Nutzung.
\\
\hline
\end{tabulary}
\par
\sphinxattableend\end{savenotes}


\section{Thumby}
\label{\detokenize{toscience:thumby}}\label{\detokenize{toscience:id31}}

\begin{savenotes}\sphinxattablestart
\centering
\sphinxcapstartof{table}
\sphinxthecaptionisattop
\sphinxcaption{Überblick}\label{\detokenize{toscience:id112}}
\sphinxaftertopcaption
\begin{tabulary}{\linewidth}[t]{|T|T|}
\hline

\sphinxAtStartPar
Source
&
\sphinxAtStartPar
{\color{red}\bfseries{}\textasciigrave{}thumby \textless{}
https://github.com/hbz/thumby\textgreater{}\textasciigrave{}\_\_}
\\
\hline
\sphinxAtStartPar
Technik
&
\sphinxAtStartPar
\sphinxhref{https://www.playframework.com/documentation/2.2.x/JavaHome}{Play Play
2.2
.2}
\\
\hline
\sphinxAtStartPar
Ports
&
\sphinxAtStartPar
9001 / 9101
\\
\hline
\sphinxAtStartPar
Verzeichnis
&
\sphinxAtStartPar
/opt/regal/apps/thumby,
/opr/regal/src/thumby
\\
\hline
\sphinxAtStartPar
HTTP Pfad
&
\sphinxAtStartPar
/tools/thumby
\\
\hline
\end{tabulary}
\par
\sphinxattableend\end{savenotes}

\sphinxAtStartPar
{\hyperref[\detokenize{toscience:_thumby}]{\emph{Thumby}}} realisiert einen Thumbnail\sphinxhyphen{}Generator. Über ein
HTTP\sphinxhyphen{}GET wird {\hyperref[\detokenize{toscience:_thumby}]{\emph{Thumby}}} die URL eines PDFs, oder eines
Bildes übergeben. Sofern die {\hyperref[\detokenize{toscience:_thumby}]{\emph{Thumby}}} den Server kennt,
wird es versuchen ein Thumbnail der zurückgelieferten Daten zu
erstellen. Die Daten werden dauerhaft auf der Platte abgelegt und
zukünftige Requests, die auf dasselbe Bild verweisen werden direkt aus
dem Speicher von {\hyperref[\detokenize{toscience:_thumby}]{\emph{Thumby}}} beantwortet.


\subsection{Konfiguration}
\label{\detokenize{toscience:konfiguration-5}}\label{\detokenize{toscience:id32}}

\begin{savenotes}\sphinxattablestart
\centering
\sphinxcapstartof{table}
\sphinxthecaptionisattop
\sphinxcaption{Dateien im /conf Verzeichnis}\label{\detokenize{toscience:id113}}
\sphinxaftertopcaption
\begin{tabulary}{\linewidth}[t]{|T|T|}
\hline
\sphinxstyletheadfamily 
\sphinxAtStartPar
Datei
&\sphinxstyletheadfamily 
\sphinxAtStartPar
Beschreibung
\\
\hline
\sphinxAtStartPar
\sphinxstylestrong{application.conf}
&
\sphinxAtStartPar
Hier wird eine Whitelist gesetzt.
Thumby verarbeitet nur URLs von
den hier angegebenen Quellen.
Hier wird auch der Pfad auf der
Platte gesetzt, unter dem Thumby
thumbnail\sphinxhyphen{}Daten ablegt.
\\
\hline
\sphinxAtStartPar
routes
&
\sphinxAtStartPar
Alle HTTP\sphinxhyphen{}Pfade übersichtlich in
einer Datei
\\
\hline
\end{tabulary}
\par
\sphinxattableend\end{savenotes}


\subsection{Die Applikation}
\label{\detokenize{toscience:die-applikation-5}}\label{\detokenize{toscience:id33}}

\begin{savenotes}\sphinxattablestart
\centering
\sphinxcapstartof{table}
\sphinxthecaptionisattop
\sphinxcaption{Das /app Verzeichnis}\label{\detokenize{toscience:id114}}
\sphinxaftertopcaption
\begin{tabulary}{\linewidth}[t]{|T|T|}
\hline
\sphinxstyletheadfamily 
\sphinxAtStartPar
Package
&\sphinxstyletheadfamily 
\sphinxAtStartPar
Beschreibung
\\
\hline
\sphinxAtStartPar
controllers
&
\sphinxAtStartPar
Der Controller realisiert eine
GET\sphinxhyphen{}Methode, über die Thumbnails
erzeugt und zurückgegeben werden.
\\
\hline
\sphinxAtStartPar
helper
&
\sphinxAtStartPar
Klassen zur Organisation des
Speichers und zur
Thumbnailgenerierung.
\\
\hline
\sphinxAtStartPar
views
&
\sphinxAtStartPar
Es gibt eine Oberfläche mit einem
Upload\sphinxhyphen{}Formular.
\\
\hline
\end{tabulary}
\par
\sphinxattableend\end{savenotes}


\section{Deepzoomer}
\label{\detokenize{toscience:deepzoomer}}\label{\detokenize{toscience:id34}}

\begin{savenotes}\sphinxattablestart
\centering
\sphinxcapstartof{table}
\sphinxthecaptionisattop
\sphinxcaption{Überblick}\label{\detokenize{toscience:id115}}
\sphinxaftertopcaption
\begin{tabulary}{\linewidth}[t]{|T|T|}
\hline

\sphinxAtStartPar
Source
&
\sphinxAtStartPar
\sphinxhref{https://github.com/hbz/DeepZoomService}{DeepZoomService}
\\
\hline
\sphinxAtStartPar
Technik
&
\sphinxAtStartPar
{\color{red}\bfseries{}\textasciigrave{}Servlet
2.3 \textless{}
https://download.oracle.com/otn\sphinxhyphen{}p
ub/jcp/7840\sphinxhyphen{}servlet\sphinxhyphen{}2.3\sphinxhyphen{}spec\sphinxhyphen{}oth\sphinxhyphen{}
JSpec/servlet\sphinxhyphen{}2\_3\sphinxhyphen{}fcs\sphinxhyphen{}spec.ps\textgreater{}\textasciigrave{}\_\_}
\\
\hline
\sphinxAtStartPar
Ports
&
\sphinxAtStartPar
9091 / 9191
\\
\hline
\sphinxAtStartPar
Verzeichnis
&
\sphinxAtStartPar
/opt/regal/tomcat\sphinxhyphen{}for\sphinxhyphen{}deepzoom/,
/opr/regal/src/DeepZoomService
\\
\hline
\end{tabulary}
\par
\sphinxattableend\end{savenotes}

\sphinxAtStartPar
Der {[}DeepZoomService{]} kann als WAR in einem Application\sphinxhyphen{}Server deployed
werden. Mit dem Deepzoomer können pyramidale Bilder erzeugt, gespeichert
und über eine OpenSeadragon\sphinxhyphen{}konforme Schnittstelle abgerufen werden. Auf
diese Weise kann in Regal eine Viewer\sphinxhyphen{}Komponente realisiert werden, die
die Anzeige sehr großer, hochaufgelöster Bilder im Webbrowser
unterstützt.


\subsection{Konfiguration}
\label{\detokenize{toscience:konfiguration-6}}\label{\detokenize{toscience:id35}}

\begin{savenotes}\sphinxattablestart
\centering
\sphinxcapstartof{table}
\sphinxthecaptionisattop
\sphinxcaption{Dateien im /conf Verzeichnis}\label{\detokenize{toscience:id116}}
\sphinxaftertopcaption
\begin{tabulary}{\linewidth}[t]{|T|T|}
\hline
\sphinxstyletheadfamily 
\sphinxAtStartPar
Datei
&\sphinxstyletheadfamily 
\sphinxAtStartPar
Beschreibung
\\
\hline
\sphinxAtStartPar
\sphinxstylestrong{deepzoomer.cfgf}
&
\sphinxAtStartPar
Hier werden lokale Verzeichnisse,
aber auch die Server\sphinxhyphen{}URLs, unter
denen der Service öffentlich
abrufbar ist, gesetzt.
\\
\hline
\end{tabulary}
\par
\sphinxattableend\end{savenotes}


\section{regal\sphinxhyphen{}drupal}
\label{\detokenize{toscience:regal-drupal}}\label{\detokenize{toscience:id36}}

\begin{savenotes}\sphinxattablestart
\centering
\sphinxcapstartof{table}
\sphinxthecaptionisattop
\sphinxcaption{Überblick}\label{\detokenize{toscience:id117}}
\sphinxaftertopcaption
\begin{tabulary}{\linewidth}[t]{|T|T|}
\hline

\sphinxAtStartPar
Source
&
\sphinxAtStartPar
\sphinxhref{https://github.com/edoweb/regal-drupal}{regal\sphinxhyphen{}drupal}
\\
\hline
\sphinxAtStartPar
Technik
&
\sphinxAtStartPar
\sphinxhref{https://www.php.net/manual/en/}{PHP
5}
\\
\hline
\sphinxAtStartPar
Ports
&
\sphinxAtStartPar
80 / 443
\\
\hline
\sphinxAtStartPar
Verzeichnis
&
\sphinxAtStartPar
/opt/re
gal/var/drupal/sites/all/modules/
\\
\hline
\end{tabulary}
\par
\sphinxattableend\end{savenotes}

\sphinxAtStartPar
Ein Drupal 7 Modul über das Funktionalitäten der
{\hyperref[\detokenize{toscience:_regal_api_2}]{\emph{regal\sphinxhyphen{}api}}} angesprochen werden können. Das Modul
bietet Oberflächen zur Konfiguration, zur Suche und zur Verwaltung von
Objekthierarchien.


\subsection{Die Applikation}
\label{\detokenize{toscience:die-applikation-6}}\label{\detokenize{toscience:id37}}

\begin{savenotes}\sphinxattablestart
\centering
\sphinxcapstartof{table}
\sphinxthecaptionisattop
\sphinxcaption{Verzeichnisstruktur}\label{\detokenize{toscience:id118}}
\sphinxaftertopcaption
\begin{tabulary}{\linewidth}[t]{|T|T|}
\hline
\sphinxstyletheadfamily 
\sphinxAtStartPar
Verzeichnis
&\sphinxstyletheadfamily 
\sphinxAtStartPar
Beschreibung
\\
\hline
\sphinxAtStartPar
edoweb
&
\sphinxAtStartPar
Hier ist der Code für die
Oberflächen.
\\
\hline
\sphinxAtStartPar
edoweb\sphinxhyphen{}field
&
\sphinxAtStartPar
Hier werden Felder für
unterschiedliche RDF\sphinxhyphen{}Properties
in der Drupal\sphinxhyphen{}Datenbank
konfiguriert. Der Code ist
größtenteils obsolet, da die
Feldlogik nicht mehr benutzt
wird.
\\
\hline
\sphinxAtStartPar
edoweb\_storage
&
\sphinxAtStartPar
Hier sind die Zugriffe auf
{\hyperref[\detokenize{toscience:_regal_api_2}]{\emph{regal\sphinxhyphen{}api}}} und
{\hyperref[\detokenize{toscience:_elasticsearch}]{\emph{???}}} zu
finden.
\\
\hline
\end{tabulary}
\par
\sphinxattableend\end{savenotes}


\section{edoweb\sphinxhyphen{}drupal\sphinxhyphen{}theme}
\label{\detokenize{toscience:edoweb-drupal-theme}}\label{\detokenize{toscience:id38}}

\begin{savenotes}\sphinxattablestart
\centering
\sphinxcapstartof{table}
\sphinxthecaptionisattop
\sphinxcaption{Überblick}\label{\detokenize{toscience:id119}}
\sphinxaftertopcaption
\begin{tabulary}{\linewidth}[t]{|T|T|}
\hline

\sphinxAtStartPar
Source
&
\sphinxAtStartPar
\sphinxhref{https://github.com/edoweb/edoweb-drupal-theme}{edow
eb\sphinxhyphen{}drupal\sphinxhyphen{}theme}
\\
\hline
\sphinxAtStartPar
Technik
&
\sphinxAtStartPar
\sphinxhref{https://www.php.net/manual/en/}{PHP
5}
\\
\hline
\sphinxAtStartPar
Ports
&
\sphinxAtStartPar
80 / 443
\\
\hline
\sphinxAtStartPar
Verzeichnis
&
\sphinxAtStartPar
/opt/r
egal/var/drupal/sites/all/themes/
\\
\hline
\end{tabulary}
\par
\sphinxattableend\end{savenotes}

\sphinxAtStartPar
Eine Reihe von Stylsheets, CSS, Icons zur Gestaltung einer Oberfläche
für den Server \sphinxurl{https://edoweb-rlp.de}


\section{zbmed\sphinxhyphen{}drupal\sphinxhyphen{}theme}
\label{\detokenize{toscience:zbmed-drupal-theme}}\label{\detokenize{toscience:id39}}

\begin{savenotes}\sphinxattablestart
\centering
\sphinxcapstartof{table}
\sphinxthecaptionisattop
\sphinxcaption{Überblick}\label{\detokenize{toscience:id120}}
\sphinxaftertopcaption
\begin{tabulary}{\linewidth}[t]{|T|T|}
\hline

\sphinxAtStartPar
Source
&
\sphinxAtStartPar
\sphinxhref{https://github.com/edoweb/zbmed-drupal-theme}{zb
med\sphinxhyphen{}drupal\sphinxhyphen{}theme}
\\
\hline
\sphinxAtStartPar
Technik
&
\sphinxAtStartPar
\sphinxhref{https://www.php.net/manual/en/}{PHP
5}
\\
\hline
\sphinxAtStartPar
Ports
&
\sphinxAtStartPar
80 / 443
\\
\hline
\sphinxAtStartPar
Verzeichnis
&
\sphinxAtStartPar
/opt/r
egal/var/drupal/sites/all/themes/
\\
\hline
\end{tabulary}
\par
\sphinxattableend\end{savenotes}

\sphinxAtStartPar
Eine Reihe von Stylsheets, CSS, Icons zur Gestaltung einer Oberfläche
für den Server \sphinxurl{https://repository.publisso.de}


\section{openwayback}
\label{\detokenize{toscience:openwayback}}\label{\detokenize{toscience:id40}}
\sphinxAtStartPar
Repo: \sphinxurl{https://github.com/iipc/openwayback} Servlet 2.5 .Überblick


\begin{savenotes}\sphinxattablestart
\centering
\begin{tabulary}{\linewidth}[t]{|T|T|}
\hline

\sphinxAtStartPar
Source
&
\sphinxAtStartPar
\sphinxhref{https://github.com/iipc/openwayback}{openwayback}
\\
\hline
\sphinxAtStartPar
Technik
&
\sphinxAtStartPar
\sphinxhref{https://download.oracle.com/otn-pub/jcp/servlet-2.5-mr5-oth-JSpec/servlet-2.5-mr5-spec.pdf}{Servlet
2.5}
\\
\hline
\sphinxAtStartPar
Ports
&
\sphinxAtStartPar
8091 / 8191
\\
\hline
\sphinxAtStartPar
Verzeichnis
&
\sphinxAtStartPar
/o
pt/regal/tomcat\sphinxhyphen{}for\sphinxhyphen{}openwayback/,
/opr/regal/src/openwayback
\\
\hline
\end{tabulary}
\par
\sphinxattableend\end{savenotes}

\sphinxAtStartPar
\sphinxstylestrong{Achtung}: Es gibt einen am hbz entwickelten Branch. Dieser ist nicht
auf Github.

\sphinxAtStartPar
Openwayback ist eine Webapplikation die im ROOT Bereich eines Tomcats
installiert werden will. Sie kann Verzeichnisse mit WARC\sphinxhyphen{}Dateien
indexieren und darauf eine Oberfläche zur Recherche und zur Navigation
aufbauen.


\section{heritrix}
\label{\detokenize{toscience:heritrix}}\label{\detokenize{toscience:id41}}
\sphinxAtStartPar
Heritrix ist ein Werkzeug zur Sammlung von Webseiten. Heritrix startet
standalone als Webapplikation und bietet eine Weboberfläche zur
Verwaltung von Sammelvorgängen an. Eingesammelte Webseiten werden als
WARC\sphinxhyphen{}Dateien in einem bestimmten Bereich der Platte abgelegt.


\section{wpull}
\label{\detokenize{toscience:wpull}}\label{\detokenize{toscience:id42}}
\sphinxAtStartPar
Wpull ist ein Kommandozeilen\sphinxhyphen{}Wermzeug zur Sammlung von Webseiten. Mit
WPull können WARC\sphinxhyphen{}Dateien erzeugt werden.


\section{Fedora Commons 3}
\label{\detokenize{toscience:fedora-commons-3}}\label{\detokenize{toscience:id43}}
\sphinxAtStartPar
Fedora Commons 3 ist ein Repository\sphinxhyphen{}Framework. Für Regal wird vorallem
die Speicherschicht von Fedora Commons 3 benutzt. Fedora\sphinxhyphen{}Commons legt
alle Daten im Dateisystem (auch) ab. Mit den Daten aus dem Dateisystem
lässt sich eine komplette Fedora\sphinxhyphen{}Commons 3 Instanz von grundauf neu
aufbauen.


\section{MySql}
\label{\detokenize{toscience:mysql}}\label{\detokenize{toscience:id44}}
\sphinxAtStartPar
MySQL wir von Fedora, regal\sphinxhyphen{}api und etikett verwendet.


\section{Elasticsearch 1.1}
\label{\detokenize{toscience:elasticsearch-1-1}}\label{\detokenize{toscience:id45}}
\sphinxAtStartPar
Elasticsearch ist eine Suchmaschine und wird von
{\hyperref[\detokenize{toscience:_regal_api_2}]{\emph{regal\sphinxhyphen{}api}}} verwendet. Auch
{\hyperref[\detokenize{toscience:_regal_drupal}]{\emph{regal\sphinxhyphen{}drupal}}} greift auf den Index zu.


\section{Drupal 7}
\label{\detokenize{toscience:drupal-7}}\label{\detokenize{toscience:id46}}
\sphinxAtStartPar
Über Drupal 7


\section{Vagrant}
\label{\detokenize{toscience:vagrant}}\label{\detokenize{toscience:id47}}
\sphinxAtStartPar
Zur Veranschaulichung dieser Dokumentation wird ein Vagrant\sphinxhyphen{}Skript
angeboten, mit dem eine Regal\sphinxhyphen{}Installation innerhalb eines
VirtualBox\sphinxhyphen{}Images erzeugt werden kann.

\sphinxAtStartPar
Zur Installation kannst Du folgende Schritte ausführen. Die Kommandos
beziehen sich auf die Installation auf einem Ubuntu\sphinxhyphen{}System. Für andere
Betriebssyteme ist die Installation ähnlich.

\sphinxAtStartPar
Die VirtualBox hat folgendes Setup
\begin{itemize}
\item {} 
\sphinxAtStartPar
hdd 40GB

\item {} 
\sphinxAtStartPar
cpu 2core

\item {} 
\sphinxAtStartPar
ram 4096M

\end{itemize}


\subsection{VirtualBox installieren}
\label{\detokenize{toscience:virtualbox-installieren}}\label{\detokenize{toscience:id48}}
\begin{sphinxVerbatim}[commandchars=\\\{\}]
\PYG{n}{sudo} \PYG{n}{apt}\PYG{o}{\PYGZhy{}}\PYG{n}{get} \PYG{n}{install} \PYG{n}{virtualbox}
\end{sphinxVerbatim}


\subsection{Vagrant installieren}
\label{\detokenize{toscience:vagrant-installieren}}\label{\detokenize{toscience:id49}}
\begin{sphinxVerbatim}[commandchars=\\\{\}]
\PYG{n}{cd} \PYG{o}{/}\PYG{n}{tmp}
\PYG{n}{wget} \PYG{n}{https}\PYG{p}{:}\PYG{o}{/}\PYG{o}{/}\PYG{n}{releases}\PYG{o}{.}\PYG{n}{hashicorp}\PYG{o}{.}\PYG{n}{com}\PYG{o}{/}\PYG{n}{vagrant}\PYG{o}{/}\PYG{l+m+mf}{2.2}\PYG{l+m+mf}{.3}\PYG{o}{/}\PYG{n}{vagrant\PYGZus{}2}\PYG{l+m+mf}{.2}\PYG{l+m+mf}{.3}\PYG{n}{\PYGZus{}x86\PYGZus{}64}\PYG{o}{.}\PYG{n}{deb}
\PYG{n}{sudo} \PYG{n}{dpkg} \PYG{o}{\PYGZhy{}}\PYG{n}{i} \PYG{n}{vagrant\PYGZus{}2}\PYG{l+m+mf}{.2}\PYG{l+m+mf}{.3}\PYG{n}{\PYGZus{}x86\PYGZus{}64}\PYG{o}{.}\PYG{n}{deb}
\end{sphinxVerbatim}


\subsection{Repository herunterladen}
\label{\detokenize{toscience:repository-herunterladen}}\label{\detokenize{toscience:id50}}
\begin{sphinxVerbatim}[commandchars=\\\{\}]
\PYG{n}{git} \PYG{n}{clone} \PYG{n}{https}\PYG{p}{:}\PYG{o}{/}\PYG{o}{/}\PYG{n}{github}\PYG{o}{.}\PYG{n}{com}\PYG{o}{/}\PYG{n}{jschnasse}\PYG{o}{/}\PYG{n}{Regal}
\PYG{n}{cd} \PYG{n}{Regal}\PYG{o}{/}\PYG{n}{vagrant}\PYG{o}{/}\PYG{n}{ubuntu}\PYG{o}{\PYGZhy{}}\PYG{l+m+mf}{14.04}
\end{sphinxVerbatim}


\subsection{Eine JDK8 bereitstellen}
\label{\detokenize{toscience:eine-jdk8-bereitstellen}}\label{\detokenize{toscience:id51}}
\sphinxAtStartPar
Hierfür bitte ein JDK8\sphinxhyphen{}Tarball herunterladen und unter dem Namen
\sphinxcode{\sphinxupquote{java8.tar.gz}} in einem Verzeichnis \sphinxcode{\sphinxupquote{/bin}} unterhalb des
Vagrant\sphinxhyphen{}Directories bereitstellen.

\begin{sphinxVerbatim}[commandchars=\\\{\}]
\PYG{n}{mkdir} \PYG{n+nb}{bin}
\PYG{n}{mv} \PYG{o}{\PYGZti{}}\PYG{o}{/}\PYG{n}{downloads}\PYG{o}{/}\PYG{n}{jdk}\PYG{o}{.}\PYG{o}{.}\PYG{o}{.}\PYG{o}{.} \PYG{n+nb}{bin}\PYG{o}{/}\PYG{n}{java8}\PYG{o}{.}\PYG{n}{tar}\PYG{o}{.}\PYG{n}{gz}
\end{sphinxVerbatim}


\subsection{Geteiltes Entwicklungsverzeichnis}
\label{\detokenize{toscience:geteiltes-entwicklungsverzeichnis}}\label{\detokenize{toscience:id52}}
\begin{sphinxVerbatim}[commandchars=\\\{\}]
\PYG{n}{mkdir} \PYG{o}{\PYGZti{}}\PYG{o}{/}\PYG{n}{regal}\PYG{o}{\PYGZhy{}}\PYG{n}{dev}
\end{sphinxVerbatim}


\subsection{Vagrant Guest Additions installieren}
\label{\detokenize{toscience:vagrant-guest-additions-installieren}}\label{\detokenize{toscience:id53}}
\begin{sphinxVerbatim}[commandchars=\\\{\}]
\PYG{n}{vagrant} \PYG{n}{plugin} \PYG{n}{install} \PYG{n}{vagrant}\PYG{o}{\PYGZhy{}}\PYG{n}{vbguest} \PYG{o}{\PYGZam{}}\PYG{o}{\PYGZam{}} \PYG{n}{vagrant} \PYG{n}{reload}
\end{sphinxVerbatim}


\subsection{Vagrant Host anlegen}
\label{\detokenize{toscience:vagrant-host-anlegen}}\label{\detokenize{toscience:id54}}
\sphinxAtStartPar
Damit alle Dienste komfortabel erreichbar sind, muss in die lokale HOSTs
Datei ein Eintrag für die Vagrant\sphinxhyphen{}Box erfolgen. Im Vagrantfile ist die
IP \sphinxcode{\sphinxupquote{192.168.50.4}} für die Box konfiguriert. Über die \sphinxcode{\sphinxupquote{FRONTEND}} und
\sphinxcode{\sphinxupquote{BACKEND}} Einträge in der \sphinxcode{\sphinxupquote{variables.conf}} ist der Servername als
\sphinxcode{\sphinxupquote{regal.vagrant}} definiert.

\begin{sphinxVerbatim}[commandchars=\\\{\}]
\PYG{n}{sudo} \PYG{n}{printf} \PYG{l+s+s2}{\PYGZdq{}}\PYG{l+s+s2}{192.168.50.4 regal.vagrant api.regal.vagrant}\PYG{l+s+s2}{\PYGZdq{}} \PYG{o}{\PYGZgt{}\PYGZgt{}} \PYG{o}{/}\PYG{n}{etc}\PYG{o}{/}\PYG{n}{hosts}
\end{sphinxVerbatim}


\subsection{Vagrant starten}
\label{\detokenize{toscience:vagrant-starten}}\label{\detokenize{toscience:id55}}
\begin{sphinxVerbatim}[commandchars=\\\{\}]
\PYG{n}{vagrant} \PYG{n}{up}
\end{sphinxVerbatim}


\subsection{Auf der Maschine einloggen}
\label{\detokenize{toscience:auf-der-maschine-einloggen}}\label{\detokenize{toscience:id56}}
\begin{sphinxVerbatim}[commandchars=\\\{\}]
\PYG{n}{vagrant} \PYG{n}{ssh}
\end{sphinxVerbatim}


\section{Server}
\label{\detokenize{toscience:server}}\label{\detokenize{toscience:id57}}
\sphinxAtStartPar
Die Installation auf einem Server kann mit Hilfe des mitgelieferten
Skriptes
\sphinxhref{https://github.com/jschnasse/Regal/blob/master/vagrant/ubuntu-14.04/regal-install.sh}{regal\sphinxhyphen{}install.sh}
erfolgen. Dazu muss analog zur Vagrant\sphinxhyphen{}Installation zunächst das \sphinxcode{\sphinxupquote{bin}}
Verzeichnis mit einem JDK aufgebaut werden. Danach erfolgt die
Installation unter \sphinxcode{\sphinxupquote{/opt/regal}} und mit einem Benutzer \sphinxcode{\sphinxupquote{regal}} (vgl.
\sphinxcode{\sphinxupquote{variables.conf}})


\subsection{Hardware Empfehlung}
\label{\detokenize{toscience:hardware-empfehlung}}\label{\detokenize{toscience:id58}}\begin{itemize}
\item {} 
\sphinxAtStartPar
hdd \textgreater{}500GB

\item {} 
\sphinxAtStartPar
cpu 8 core

\item {} 
\sphinxAtStartPar
ram 32 G

\end{itemize}


\subsection{Unterschiede zur Vagrant Installation}
\label{\detokenize{toscience:unterschiede-zur-vagrant-installation}}\label{\detokenize{toscience:id59}}
\sphinxAtStartPar
Auf dem Server empfehlen ich den fedora tomcat mit erweiterten
Speichereinstellungen zu betreiben.

\sphinxAtStartPar
Dazu in \sphinxcode{\sphinxupquote{/opt/regal/bin/fedora/tomcat/bin}} eine \sphinxcode{\sphinxupquote{setenv.sh}} anlegen
und folgende Zeilen hinein kopieren.

\begin{sphinxVerbatim}[commandchars=\\\{\}]
\PYG{n}{CATALINA\PYGZus{}OPTS}\PYG{o}{=}\PYG{l+s+s2}{\PYGZdq{}}\PYG{l+s+s2}{ }\PYG{l+s+se}{\PYGZbs{}}
\PYG{l+s+s2}{\PYGZhy{}Xms1536m }\PYG{l+s+se}{\PYGZbs{}}
\PYG{l+s+s2}{\PYGZhy{}Xmx1536m }\PYG{l+s+se}{\PYGZbs{}}
\PYG{l+s+s2}{\PYGZhy{}XX:NewSize=256m }\PYG{l+s+se}{\PYGZbs{}}
\PYG{l+s+s2}{\PYGZhy{}XX:MaxNewSize=256m }\PYG{l+s+se}{\PYGZbs{}}
\PYG{l+s+s2}{\PYGZhy{}XX:PermSize=256m }\PYG{l+s+se}{\PYGZbs{}}
\PYG{l+s+s2}{\PYGZhy{}XX:MaxPermSize=256m }\PYG{l+s+se}{\PYGZbs{}}
\PYG{l+s+s2}{\PYGZhy{}server }\PYG{l+s+se}{\PYGZbs{}}
\PYG{l+s+s2}{\PYGZhy{}Djava.awt.headless=true }\PYG{l+s+se}{\PYGZbs{}}
\PYG{l+s+s2}{\PYGZhy{}Dorg.apache.jasper.runtime.BodyContentImpl.LIMIT\PYGZus{}BUFFER=true}\PYG{l+s+s2}{\PYGZdq{}}

\PYG{n}{export} \PYG{n}{CATALINA\PYGZus{}OPTS}
\end{sphinxVerbatim}


\subsubsection{Entwicklung Java}
\label{\detokenize{toscience:entwicklung-java}}\label{\detokenize{toscience:id60}}

\section{In der VirtualBox}
\label{\detokenize{toscience:in-der-virtualbox}}\label{\detokenize{toscience:id61}}
\sphinxAtStartPar
Hat man über {\hyperref[\detokenize{toscience:_vagrant}]{\emph{Vagrant}}} eine neue VirtualBox erzeugt und
alle Konfigurationen wie beschrieben vorgenommen, kann man die
VirtualBox zur Entwicklung nutzen. Da im Installationsprozess bereits
Eclipse\sphinxhyphen{}Projekte der unter \sphinxcode{\sphinxupquote{/opt/regal/src}} befindlichen
Java\sphinxhyphen{}Applikationen erzeugt wurden, können die Projekte direkt aus dem
“synced folder” unter \sphinxcode{\sphinxupquote{\textasciitilde{}/regal\sphinxhyphen{}dev}} in eine Eclipse\sphinxhyphen{}IDE auf dem
Host\sphinxhyphen{}System importiert werden.

\sphinxAtStartPar
Damit Änderungen am Code in der VirtualBox direkt sichtbar werden,
sollte die Applikation zunächst im Develop\sphinxhyphen{}Mode neu gestartet werden.
Dazu loggt man sich auf der VirtualBox mit \sphinxcode{\sphinxupquote{vagrant ssh}} ein und
stoppt zunächst den entsprechenden Service, z.B.
\sphinxcode{\sphinxupquote{sudo service regal\sphinxhyphen{}api stop}}. Anschließend navigiert man in das
Source\sphinxhyphen{}Verzeichnis, z.B. \sphinxcode{\sphinxupquote{cd /opt/regal/src/regal\sphinxhyphen{}api}}. Hier startet
man die Applikation auf dem korrekten Port (im Zweifel unter
\sphinxcode{\sphinxupquote{/opt/regal/apps/regal\sphinxhyphen{}api/conf/application.conf}} nachschauen). Der
Start im Develop\sphinxhyphen{}Mode erfolgt aus dem Verzeichnis der Applikation, mit
z.B. \sphinxcode{\sphinxupquote{/opt/regal/bin/activator/bin/activator \sphinxhyphen{}Dhttp.port=9100}}. Danach
kann in die Kosole \sphinxcode{\sphinxupquote{run}} eingegegeben werden. Die Applikation sollte
nun unter dem entsprechenden Port (im Beispiel: 9100) antworten.

\sphinxAtStartPar
Leider funktioniert das Reloading zwischen Host\sphinxhyphen{}System und
Guest\sphinxhyphen{}VirtualBox nicht richtig. D.h. nach Code\sphinxhyphen{}Änderungen im Host, muss
auf der Virtualbox zunächst mit \sphinxcode{\sphinxupquote{Ctrl+D}} und \sphinxcode{\sphinxupquote{run}} neu gestartet
werden, damit die Änderungen sichtbar werden.


\section{Auf dem eigenen System}
\label{\detokenize{toscience:auf-dem-eigenen-system}}\label{\detokenize{toscience:id62}}
\sphinxAtStartPar
Die Javakomponenten können problemlos auch auf einem aktuellen
Ubuntusystem entwickelt werden. Leider läuft die
PHP/Drupal\sphinxhyphen{}Implementierung nicht unter neueren Ubuntusystemen. Für die
lokale installation können die entsprechenden Funktionen aus dem
\sphinxcode{\sphinxupquote{regal\sphinxhyphen{}install.sh}} ausgeführt werden. Dazu einfach eine Kopie anlegen,
entsprechend editieren und ausführen.

\begin{sphinxVerbatim}[commandchars=\\\{\}]
\PYG{n}{mkdir} \PYG{n}{regal}\PYG{o}{\PYGZhy{}}\PYG{n}{install}
\PYG{n}{cp} \PYG{o}{\PYGZhy{}}\PYG{n}{r} \PYG{n}{path}\PYG{o}{/}\PYG{n}{to}\PYG{o}{/}\PYG{n}{Regal}\PYG{o}{/}\PYG{n}{vagrant}\PYG{o}{/}\PYG{n}{ubuntu}\PYG{o}{\PYGZhy{}}\PYG{n}{XX}\PYG{o}{/}\PYG{o}{*} \PYG{n}{regal}\PYG{o}{\PYGZhy{}}\PYG{n}{install}
\PYG{n}{cd} \PYG{n}{regal}\PYG{o}{\PYGZhy{}}\PYG{n}{install}
\PYG{c+c1}{\PYGZsh{} Edit system user \PYGZdq{}vagrant\PYGZdq{} \PYGZhy{}\PYGZhy{}\PYGZgt{} \PYGZdq{}your user\PYGZdq{}}
\PYG{n}{editor} \PYG{n}{variables}\PYG{o}{.}\PYG{n}{conf}
\PYG{c+c1}{\PYGZsh{} put drupal stuff in comments}
\PYG{c+c1}{\PYGZsh{}}
\PYG{c+c1}{\PYGZsh{}  \PYGZsh{}installDrush}
\PYG{c+c1}{\PYGZsh{}  \PYGZsh{}installDrupal}
\PYG{c+c1}{\PYGZsh{}  \PYGZsh{}installRegalDrupal}
\PYG{c+c1}{\PYGZsh{}  \PYGZsh{}installDrupalThemes}
\PYG{c+c1}{\PYGZsh{}  \PYGZsh{}configureDrupalLanguages}
\PYG{c+c1}{\PYGZsh{}  \PYGZsh{}configureDrupal}
\PYG{c+c1}{\PYGZsh{}}
\PYG{n}{editor} \PYG{n}{regal}\PYG{o}{\PYGZhy{}}\PYG{n}{install}\PYG{o}{.}\PYG{n}{sh}
\end{sphinxVerbatim}


\section{Aktualisierung}
\label{\detokenize{toscience:aktualisierung}}\label{\detokenize{toscience:id63}}

\subsection{Play\sphinxhyphen{}Applikationen}
\label{\detokenize{toscience:play-applikationen}}\label{\detokenize{toscience:id64}}
\sphinxAtStartPar
Die Aktualisierung der Regal\sphinxhyphen{}Komponenten erfolgt über Skripte. Die
Aktualisierung funktioniert dabei so, dass der Quellcode der zu
aktualisierenden Komponente unter \sphinxcode{\sphinxupquote{/opt/regal/src}} per \sphinxcode{\sphinxupquote{git}} auf den
entsprechenden Branch gestellt wird. Danach wird ein neues Kompilat der
Komponente erzeugt. Die aktuelle Konfiguration wird aus
\sphinxcode{\sphinxupquote{/opt/regal/conf}} genommen und es wird unter \sphinxcode{\sphinxupquote{/opt/regal/apps}} eine
neue lauffähige Version abgelegt.

\sphinxAtStartPar
Neue Versionen werden immer parallel zu alten Versionen gestartet und
über einen Wechsel der Apachekonfiguration aktiviert. Erst danach wird
die alte Version heruntergefahren.

\sphinxAtStartPar
Der komplette Aktualisierungsprozess erfolgt automatisch. Die alte
Version bleibt immer auf dem Server liegen, so dass bei Bedarf wieder
zurück gewechselt werden kann.


\subsection{Tomcat\sphinxhyphen{}Applikation}
\label{\detokenize{toscience:tomcat-applikation}}\label{\detokenize{toscience:id65}}
\sphinxAtStartPar
Es wird ein \sphinxcode{\sphinxupquote{war}}\sphinxhyphen{}Container erzeugt und im Tomcat \sphinxcode{\sphinxupquote{hot}}\sphinxhyphen{}deployed.


\subsection{Drupal\sphinxhyphen{}Module}
\label{\detokenize{toscience:drupal-module}}\label{\detokenize{toscience:id66}}
\sphinxAtStartPar
Beinhaltet die Aktualisierung ein Datenbankupdate, so wird Drupal erst
in den Wartungszustand versetzt (per drush oder über die Oberfläche).
Danach wird die aktualisierte Version einfach per Git geholt. Bei
Datenbankupdates wird noch ein Drupal\sphinxhyphen{}Updateskript ausgeführt.


\subsection{Speicherschicht}
\label{\detokenize{toscience:speicherschicht}}\label{\detokenize{toscience:id67}}
\sphinxAtStartPar
Aktualisierungen von MySQL, Elasticsearch und Fedora gehen mit einer
Downtime einher.


\section{Verzeichnisse}
\label{\detokenize{toscience:verzeichnisse}}\label{\detokenize{toscience:id68}}

\begin{savenotes}\sphinxattablestart
\centering
\sphinxcapstartof{table}
\sphinxthecaptionisattop
\sphinxcaption{Verzeichnisstruktur}\label{\detokenize{toscience:id121}}
\sphinxaftertopcaption
\begin{tabulary}{\linewidth}[t]{|T|T|}
\hline
\sphinxstyletheadfamily 
\sphinxAtStartPar
Verzeichnis
&\sphinxstyletheadfamily 
\sphinxAtStartPar
Beschreibung
\\
\hline
\sphinxAtStartPar
/opt/regal
&
\sphinxAtStartPar
Außer Apache2, Elasticsearch und
MySQL befinden sich alle
Regal\sphinxhyphen{}Komponenten unter diesem
Verzeichnis.
\\
\hline
\sphinxAtStartPar
/opt/regal/apps
&
\sphinxAtStartPar
Die auf \sphinxcode{\sphinxupquote{Play}} beruhenden
Komponenten:
\sphinxcode{\sphinxupquote{etikett  fedora  regal\sphinxhyphen{}a
pi  skos\sphinxhyphen{}lookup  thumby  zettel}}
\\
\hline
\sphinxAtStartPar
/opt/regal/bin
&
\sphinxAtStartPar
Fremdpakete wie activator,
fedora, heritrix, python \sphinxhyphen{}
weitere tomcats.
\\
\hline
\sphinxAtStartPar
/opt/regal/conf
&
\sphinxAtStartPar
Die variables.conf und die
application.conf wird von
verschiedenen Komponenten
verwendet.
\\
\hline
\sphinxAtStartPar
/opt/regal/logs
&
\sphinxAtStartPar
Logfiles der Skripte und Cronjobs
\\
\hline
\sphinxAtStartPar
/opt/regal/src
&
\sphinxAtStartPar
Alle Eigenentwicklungen oder im
Quellcode modifizierten
Komponenten.
\\
\hline
\sphinxAtStartPar
/opt/regal/var
&
\sphinxAtStartPar
drupal und Datenverzeichnisse.
\\
\hline
\end{tabulary}
\par
\sphinxattableend\end{savenotes}


\section{Ports}
\label{\detokenize{toscience:ports}}\label{\detokenize{toscience:id69}}

\begin{savenotes}\sphinxattablestart
\centering
\sphinxcapstartof{table}
\sphinxthecaptionisattop
\sphinxcaption{Ports und Komponenten (typische Belegung)}\label{\detokenize{toscience:id122}}
\sphinxaftertopcaption
\begin{tabulary}{\linewidth}[t]{|T|T|}
\hline
\sphinxstyletheadfamily 
\sphinxAtStartPar
Port
&\sphinxstyletheadfamily 
\sphinxAtStartPar
Komponente
\\
\hline
\sphinxAtStartPar
80 /443
&
\sphinxAtStartPar
Apache 2
\\
\hline
\sphinxAtStartPar
8080
&
\sphinxAtStartPar
fedora tomcat
\\
\hline
\sphinxAtStartPar
9090
&
\sphinxAtStartPar
openwayback tomcat
\\
\hline
\sphinxAtStartPar
9200
&
\sphinxAtStartPar
elasticsearch
\\
\hline
\sphinxAtStartPar
9000/9100
&
\sphinxAtStartPar
regal\sphinxhyphen{}api
\\
\hline
\sphinxAtStartPar
9001/9101
&
\sphinxAtStartPar
thumby
\\
\hline
\sphinxAtStartPar
9002/9102
&
\sphinxAtStartPar
etikett
\\
\hline
\sphinxAtStartPar
9003/9103
&
\sphinxAtStartPar
zettel
\\
\hline
\sphinxAtStartPar
9004/9104
&
\sphinxAtStartPar
skos\sphinxhyphen{}lookup
\\
\hline
\end{tabulary}
\par
\sphinxattableend\end{savenotes}


\section{Logs}
\label{\detokenize{toscience:logs}}\label{\detokenize{toscience:id70}}

\begin{savenotes}\sphinxattablestart
\centering
\sphinxcapstartof{table}
\sphinxthecaptionisattop
\sphinxcaption{Logfiles}\label{\detokenize{toscience:id123}}
\sphinxaftertopcaption
\begin{tabulary}{\linewidth}[t]{|T|T|}
\hline
\sphinxstyletheadfamily 
\sphinxAtStartPar
Komponente
&\sphinxstyletheadfamily 
\sphinxAtStartPar
Pfad
\\
\hline
\sphinxAtStartPar
Apache
&
\sphinxAtStartPar
/var/log/apache2
\\
\hline
\sphinxAtStartPar
Tomcat
&
\sphinxAtStartPar
/opt/regal/bin/fedora/tomcat/logs
\\
\hline
\sphinxAtStartPar
Fedora
&
\sphinxAtStartPar
/opt/regal/bin/fedora/server/logs
\\
\hline
\sphinxAtStartPar
Elasticsearch
&
\sphinxAtStartPar
/var/log/elasticsearch
\\
\hline
\sphinxAtStartPar
regal\sphinxhyphen{}api
&
\sphinxAtStartPar
/opt/regal/apps/regal\sphinxhyphen{}api/logs
\\
\hline
\sphinxAtStartPar
drupal
&
\sphinxAtStartPar
/var/log/apache2 \#otherhosts !
und/var/log/apache2/error.log
(hier ist auch die Debugausgabe)
\\
\hline
\sphinxAtStartPar
MySql
&
\sphinxAtStartPar
/var/log/mysql
\\
\hline
\sphinxAtStartPar
monit
&
\sphinxAtStartPar
/var/log/monit.log
\\
\hline
\sphinxAtStartPar
regal\sphinxhyphen{}scripts
&
\sphinxAtStartPar
/opt/regal/logs
\\
\hline
\end{tabulary}
\par
\sphinxattableend\end{savenotes}


\section{Configs}
\label{\detokenize{toscience:configs}}\label{\detokenize{toscience:id71}}

\begin{savenotes}\sphinxattablestart
\centering
\sphinxcapstartof{table}
\sphinxthecaptionisattop
\sphinxcaption{Configfiles}\label{\detokenize{toscience:id124}}
\sphinxaftertopcaption
\begin{tabulary}{\linewidth}[t]{|T|T|}
\hline
\sphinxstyletheadfamily 
\sphinxAtStartPar
Komponente
&\sphinxstyletheadfamily 
\sphinxAtStartPar
Pfad
\\
\hline
\sphinxAtStartPar
Apache
&
\sphinxAtStartPar
/etc/apache2/sites\sphinxhyphen{}enabled
\\
\hline
\sphinxAtStartPar
Tomcat
&
\sphinxAtStartPar
/opt/regal/bin/fedora/tomcat/conf
\\
\hline
\sphinxAtStartPar
Fedora
&
\sphinxAtStartPar
/opt/regal/bin/fedora/server/conf
\\
\hline
\sphinxAtStartPar
Elasticsearch
&
\sphinxAtStartPar
/etc/elasticsearch
\\
\hline
\sphinxAtStartPar
regal\sphinxhyphen{}api
&
\sphinxAtStartPar
/opt/regal/conf enthält
Konfigurationsvorschläge des
Installers
\\
\hline
\sphinxAtStartPar
regal\sphinxhyphen{}api
&
\sphinxAtStartPar
/opt/regal/apps/regal\sphinxhyphen{}api/conf
\\
\hline
\sphinxAtStartPar
drupal
&
\sphinxAtStartPar
Konfig kann gut mit dem Tool
drush überwacht werden
\\
\hline
\sphinxAtStartPar
Elasticsearch Plugins
&
\sphinxAtStartPar
/etc/elasticsearch
\\
\hline
\sphinxAtStartPar
oai\sphinxhyphen{}pmh
&
\sphinxAtStartPar
/opt/regal/
bin/fedora/tomcat/webapps/dnb\sphinxhyphen{}unr
/WEB\sphinxhyphen{}INF/classes/proai.properties
\\
\hline
\sphinxAtStartPar
monit
&
\sphinxAtStartPar
/etc/monit
\\
\hline
\end{tabulary}
\par
\sphinxattableend\end{savenotes}


\section{Apache2}
\label{\detokenize{toscience:apache2}}\label{\detokenize{toscience:id72}}

\begin{savenotes}\sphinxattablestart
\centering
\sphinxcapstartof{table}
\sphinxthecaptionisattop
\sphinxcaption{Frontend Pfade}\label{\detokenize{toscience:id125}}
\sphinxaftertopcaption
\begin{tabulary}{\linewidth}[t]{|T|T|T|}
\hline
\sphinxstyletheadfamily 
\sphinxAtStartPar
Komponente
&\sphinxstyletheadfamily 
\sphinxAtStartPar
HTTP\sphinxhyphen{}Pfad
&\sphinxstyletheadfamily 
\sphinxAtStartPar
Lokaler Pfad/Proxy
\\
\hline
\sphinxAtStartPar
Drupal
&
\sphinxAtStartPar
/
&
\sphinxAtStartPar
/opt/regal/var/drupal
\\
\hline
\sphinxAtStartPar
Alte Importe von
Webarchiven
&
\sphinxAtStartPar
/webharvests
&
\sphinxAtStartPar
/data/webharvests
\\
\hline
\sphinxAtStartPar
Täglich generierte
Datei mit Kennziffern
&
\sphinxAtStartPar
/crawlreports
&
\sphinxAtStartPar
/o
pt/regal/crawlreports
\\
\hline
\end{tabulary}
\par
\sphinxattableend\end{savenotes}


\begin{savenotes}\sphinxattablestart
\centering
\sphinxcapstartof{table}
\sphinxthecaptionisattop
\sphinxcaption{API Pfade}\label{\detokenize{toscience:id126}}
\sphinxaftertopcaption
\begin{tabulary}{\linewidth}[t]{|T|T|T|}
\hline
\sphinxstyletheadfamily 
\sphinxAtStartPar
Komponente
&\sphinxstyletheadfamily 
\sphinxAtStartPar
HTTP\sphinxhyphen{}Pfad
&\sphinxstyletheadfamily 
\sphinxAtStartPar
Lokaler Pfad/Proxy
\\
\hline
\sphinxAtStartPar
Über wget erstellte
Webarchive
&
\sphinxAtStartPar
/wget\sphinxhyphen{}data
&
\sphinxAtStartPar
/op
t/regal/var/wget\sphinxhyphen{}data
\\
\hline
\sphinxAtStartPar
Über wpull erstellte
Webarchive
&
\sphinxAtStartPar
/wpull\sphinxhyphen{}data
&
\sphinxAtStartPar
/opt
/regal/var/wpull\sphinxhyphen{}data
\\
\hline
\sphinxAtStartPar
Über heritrix
erstellte Webarchive
&
\sphinxAtStartPar
/heritrix\sphinxhyphen{}data
&
\sphinxAtStartPar
/opt/re
gal/var/heritrix\sphinxhyphen{}data
\\
\hline
\sphinxAtStartPar
OAI\sphinxhyphen{}Schnittstelle für
die DNB
&
\sphinxAtStartPar
/dnb\sphinxhyphen{}urn
&
\sphinxAtStartPar
\sphinxurl{http://loc}
alhost:8080/dnb\sphinxhyphen{}urn\$1
\\
\hline
\sphinxAtStartPar
OAI\sphinxhyphen{}Schnittstelle
&
\sphinxAtStartPar
/oai\sphinxhyphen{}pmh
&
\sphinxAtStartPar
\sphinxurl{http://loc}
alhost:8080/oai\sphinxhyphen{}pmh\$1
\\
\hline
\sphinxAtStartPar
Deepzoomer
&
\sphinxAtStartPar
/deepzoom
&
\sphinxAtStartPar
\sphinxurl{http://loca}
lhost:7080/deepzoom\$1
\\
\hline
\sphinxAtStartPar
Openwayback privat
&
\sphinxAtStartPar
/wayback
&
\sphinxAtStartPar
\sphinxurl{http://l}
ocalhost:9080/wayback
\\
\hline
\sphinxAtStartPar
Openwayback
öffentlich
&
\sphinxAtStartPar
/weltweit
&
\sphinxAtStartPar
\sphinxurl{http://lo}
calhost:9080/weltweit
\\
\hline
\sphinxAtStartPar
Thumby
&
\sphinxAtStartPar
/tools/thumby
&
\sphinxAtStartPar
\sphinxurl{http://localh}
ost:9001/tools/thumby
\\
\hline
\sphinxAtStartPar
Etikett
&
\sphinxAtStartPar
/tools/etikett
&
\sphinxAtStartPar
\sphinxurl{http://localho}
st:9002/tools/etikett
\\
\hline
\sphinxAtStartPar
Zettel
&
\sphinxAtStartPar
/tools/zettel
&
\sphinxAtStartPar
\sphinxurl{http://localh}
ost:9004/tools/zettel
\\
\hline
\sphinxAtStartPar
Elasticsearch GET
&
\sphinxAtStartPar
/search
&
\sphinxAtStartPar
\sphinxurl{http://localhost:9200}
\\
\hline
\sphinxAtStartPar
Fedora
&
\sphinxAtStartPar
/fedora
&
\sphinxAtStartPar
\sphinxurl{http://}
localhost:8080/fedora
\\
\hline
\sphinxAtStartPar
JSON\sphinxhyphen{}LD Context
&
\sphinxAtStartPar
/
public/resources.json
&
\sphinxAtStartPar
\sphinxurl{http:/}
/localhost:9002/tools
/etikett/context.json
\\
\hline
\sphinxAtStartPar
regal\sphinxhyphen{}api
&
\sphinxAtStartPar
/
&
\sphinxAtStartPar
h
ttp://localhost:9000/
\\
\hline
\sphinxAtStartPar
heritrix
&
\sphinxAtStartPar
/tools/heritrix
&
\sphinxAtStartPar
\sphinxurl{https://localhos}
t:8443/tools/heritrix
\\
\hline
\end{tabulary}
\par
\sphinxattableend\end{savenotes}


\section{Matomo}
\label{\detokenize{toscience:matomo}}\label{\detokenize{toscience:id73}}
\sphinxAtStartPar
Matomo wird einmal täglich per Cronjob mit Apache\sphinxhyphen{}Logfiles befüllt.
Dabei erfolgt eine Anonymisierung. Die Logfiles verbleiben noch sieben
Tage auf dem Server und werden dann annoynmisiert.


\section{Monit}
\label{\detokenize{toscience:monit}}\label{\detokenize{toscience:id74}}
\sphinxAtStartPar
Das Tool Monit erlaubt es, den Status der Regal\sphinxhyphen{}Komponenten zu
überwachen und Dienste ggfl. neu zu starten. Folgende Einträge können in
/etc/monit/monitrc vorgenommen werden

\begin{sphinxVerbatim}[commandchars=\\\{\}]
\PYG{n}{check} \PYG{n}{process} \PYG{n}{apache} \PYG{k}{with} \PYG{n}{pidfile} \PYG{o}{/}\PYG{n}{var}\PYG{o}{/}\PYG{n}{run}\PYG{o}{/}\PYG{n}{apache2}\PYG{o}{/}\PYG{n}{apache2}\PYG{o}{.}\PYG{n}{pid}
    \PYG{n}{start} \PYG{n}{program} \PYG{o}{=} \PYG{l+s+s2}{\PYGZdq{}}\PYG{l+s+s2}{/etc/init.d/apache2 start}\PYG{l+s+s2}{\PYGZdq{}} \PYG{k}{with} \PYG{n}{timeout} \PYG{l+m+mi}{60} \PYG{n}{seconds}
    \PYG{n}{stop} \PYG{n}{program}  \PYG{o}{=} \PYG{l+s+s2}{\PYGZdq{}}\PYG{l+s+s2}{/etc/init.d/apache2 stop}\PYG{l+s+s2}{\PYGZdq{}}

\PYG{n}{check} \PYG{n}{process} \PYG{n}{regal}\PYG{o}{\PYGZhy{}}\PYG{n}{api} \PYG{k}{with} \PYG{n}{pidfile} \PYG{o}{/}\PYG{n}{opt}\PYG{o}{/}\PYG{n}{regal}\PYG{o}{/}\PYG{n}{apps}\PYG{o}{/}\PYG{n}{regal}\PYG{o}{\PYGZhy{}}\PYG{n}{api}\PYG{o}{/}\PYG{n}{RUNNING\PYGZus{}PID}
     \PYG{n}{start} \PYG{n}{program} \PYG{o}{=} \PYG{l+s+s2}{\PYGZdq{}}\PYG{l+s+s2}{/etc/init.d/regal\PYGZhy{}api start}\PYG{l+s+s2}{\PYGZdq{}} \PYG{k}{with} \PYG{n}{timeout} \PYG{l+m+mi}{60} \PYG{n}{seconds}
     \PYG{n}{stop} \PYG{n}{program} \PYG{o}{=} \PYG{l+s+s2}{\PYGZdq{}}\PYG{l+s+s2}{/etc/init.d/regal\PYGZhy{}api stop}\PYG{l+s+s2}{\PYGZdq{}}

\PYG{n}{check} \PYG{n}{process} \PYG{n}{tomcat6} \PYG{k}{with} \PYG{n}{pidfile} \PYG{o}{/}\PYG{n}{var}\PYG{o}{/}\PYG{n}{run}\PYG{o}{/}\PYG{n}{tomcat6}\PYG{o}{.}\PYG{n}{pid}
     \PYG{n}{start} \PYG{n}{program} \PYG{o}{=} \PYG{l+s+s2}{\PYGZdq{}}\PYG{l+s+s2}{/etc/init.d/tomcat6 start}\PYG{l+s+s2}{\PYGZdq{}} \PYG{k}{with} \PYG{n}{timeout} \PYG{l+m+mi}{60} \PYG{n}{seconds}
     \PYG{n}{stop} \PYG{n}{program} \PYG{o}{=} \PYG{l+s+s2}{\PYGZdq{}}\PYG{l+s+s2}{/etc/init.d/regal\PYGZhy{}api stop}\PYG{l+s+s2}{\PYGZdq{}}

\PYG{n}{check} \PYG{n}{process} \PYG{n}{elasticsearch} \PYG{k}{with} \PYG{n}{pidfile} \PYG{o}{/}\PYG{n}{var}\PYG{o}{/}\PYG{n}{run}\PYG{o}{/}\PYG{n}{elasticsearch}\PYG{o}{.}\PYG{n}{pid}
     \PYG{n}{start} \PYG{n}{program} \PYG{o}{=} \PYG{l+s+s2}{\PYGZdq{}}\PYG{l+s+s2}{/etc/init.d/elasticsearch start}\PYG{l+s+s2}{\PYGZdq{}} \PYG{k}{with} \PYG{n}{timeout} \PYG{l+m+mi}{60} \PYG{n}{seconds}
     \PYG{n}{stop} \PYG{n}{program} \PYG{o}{=} \PYG{l+s+s2}{\PYGZdq{}}\PYG{l+s+s2}{/etc/init.d/elasticsearch stop}\PYG{l+s+s2}{\PYGZdq{}}

\PYG{n}{check} \PYG{n}{process} \PYG{n}{thumby} \PYG{k}{with} \PYG{n}{pidfile} \PYG{o}{/}\PYG{n}{opt}\PYG{o}{/}\PYG{n}{regal}\PYG{o}{/}\PYG{n}{apps}\PYG{o}{/}\PYG{n}{thumby}\PYG{o}{/}\PYG{n}{RUNNING\PYGZus{}PID}
     \PYG{n}{start} \PYG{n}{program} \PYG{o}{=} \PYG{l+s+s2}{\PYGZdq{}}\PYG{l+s+s2}{/etc/init.d/thumby start}\PYG{l+s+s2}{\PYGZdq{}} \PYG{k}{with} \PYG{n}{timeout} \PYG{l+m+mi}{60} \PYG{n}{seconds}
     \PYG{n}{stop} \PYG{n}{program} \PYG{o}{=} \PYG{l+s+s2}{\PYGZdq{}}\PYG{l+s+s2}{/etc/init.d/thumby stop}\PYG{l+s+s2}{\PYGZdq{}}

\PYG{n}{check} \PYG{n}{process} \PYG{n}{etikett} \PYG{k}{with} \PYG{n}{pidfile} \PYG{o}{/}\PYG{n}{opt}\PYG{o}{/}\PYG{n}{regal}\PYG{o}{/}\PYG{n}{apps}\PYG{o}{/}\PYG{n}{etikett}\PYG{o}{/}\PYG{n}{RUNNING\PYGZus{}PID}
     \PYG{n}{start} \PYG{n}{program} \PYG{o}{=} \PYG{l+s+s2}{\PYGZdq{}}\PYG{l+s+s2}{/etc/init.d/etikett start}\PYG{l+s+s2}{\PYGZdq{}} \PYG{k}{with} \PYG{n}{timeout} \PYG{l+m+mi}{60} \PYG{n}{seconds}
     \PYG{n}{stop} \PYG{n}{program} \PYG{o}{=} \PYG{l+s+s2}{\PYGZdq{}}\PYG{l+s+s2}{/etc/init.d/etikett stop}\PYG{l+s+s2}{\PYGZdq{}}

\PYG{n}{check} \PYG{n}{process} \PYG{n}{zettel} \PYG{k}{with} \PYG{n}{pidfile} \PYG{o}{/}\PYG{n}{opt}\PYG{o}{/}\PYG{n}{regal}\PYG{o}{/}\PYG{n}{apps}\PYG{o}{/}\PYG{n}{zettel}\PYG{o}{/}\PYG{n}{RUNNING\PYGZus{}PID}
     \PYG{n}{start} \PYG{n}{program} \PYG{o}{=} \PYG{l+s+s2}{\PYGZdq{}}\PYG{l+s+s2}{/etc/init.d/zettel start}\PYG{l+s+s2}{\PYGZdq{}} \PYG{k}{with} \PYG{n}{timeout} \PYG{l+m+mi}{60} \PYG{n}{seconds}
     \PYG{n}{stop} \PYG{n}{program} \PYG{o}{=} \PYG{l+s+s2}{\PYGZdq{}}\PYG{l+s+s2}{/etc/init.d/zettel stop}\PYG{l+s+s2}{\PYGZdq{}}
\end{sphinxVerbatim}


\section{Scripts und Cronjobs}
\label{\detokenize{toscience:scripts-und-cronjobs}}\label{\detokenize{toscience:id75}}
\sphinxAtStartPar
Für das Funktionieren von Regal sind einige regal\sphinxhyphen{}scripts sinnvoll. Die
Skripte sind sämtlich unter Github zu finden.

\sphinxAtStartPar
\sphinxurl{https://github.com/edoweb/regal-scripts}

\sphinxAtStartPar
Die folgenden Abschnitte zeigen ein typisches Setup.


\subsection{OAI\sphinxhyphen{}Providing}
\label{\detokenize{toscience:oai-providing-2}}\label{\detokenize{toscience:id76}}
\sphinxAtStartPar
Der OAI\sphinxhyphen{}Provider läuft nicht die ganze Zeit mit, da dies Probleme
gemacht hat. Er wird nur für einen bestimmten Zeitraum angestellt und
dann wieder ausgestellt. Auf diese Weise liefert die OAI\sphinxhyphen{}Schnittstelle
tagesaktuelle Daten.

\begin{sphinxVerbatim}[commandchars=\\\{\}]
\PYG{l+m+mi}{0} \PYG{l+m+mi}{2} \PYG{o}{*} \PYG{o}{*} \PYG{o}{*} \PYG{o}{/}\PYG{n}{opt}\PYG{o}{/}\PYG{n}{regal}\PYG{o}{/}\PYG{n}{src}\PYG{o}{/}\PYG{n}{regal}\PYG{o}{\PYGZhy{}}\PYG{n}{scripts}\PYG{o}{/}\PYG{n}{turnOnOaiPmhPolling}\PYG{o}{.}\PYG{n}{sh}
\PYG{l+m+mi}{0} \PYG{l+m+mi}{5} \PYG{o}{*} \PYG{o}{*} \PYG{o}{*} \PYG{o}{/}\PYG{n}{opt}\PYG{o}{/}\PYG{n}{regal}\PYG{o}{/}\PYG{n}{src}\PYG{o}{/}\PYG{n}{regal}\PYG{o}{\PYGZhy{}}\PYG{n}{scripts}\PYG{o}{/}\PYG{n}{turnOffOaiPmhPolling}\PYG{o}{.}\PYG{n}{sh}
\end{sphinxVerbatim}


\subsection{URN\sphinxhyphen{}Registrierung}
\label{\detokenize{toscience:urn-registrierung}}\label{\detokenize{toscience:id77}}
\sphinxAtStartPar
Die URN\sphinxhyphen{}Registrierung erfolgt mit einem gewissen Verzug. Das dafür
zuständige Skript überprüft daher zunächst das Anlagedatum der
Ressource.

\begin{sphinxVerbatim}[commandchars=\\\{\}]
\PYG{l+m+mi}{05} \PYG{l+m+mi}{7} \PYG{o}{*} \PYG{o}{*} \PYG{o}{*} \PYG{o}{/}\PYG{n}{opt}\PYG{o}{/}\PYG{n}{regal}\PYG{o}{/}\PYG{n}{src}\PYG{o}{/}\PYG{n}{regal}\PYG{o}{\PYGZhy{}}\PYG{n}{scripts}\PYG{o}{/}\PYG{n}{register\PYGZus{}urn}\PYG{o}{.}\PYG{n}{sh} \PYG{n}{control}  \PYG{o}{\PYGZgt{}\PYGZgt{}} \PYG{o}{/}\PYG{n}{opt}\PYG{o}{/}\PYG{n}{regal}\PYG{o}{/}\PYG{n}{regal}\PYG{o}{\PYGZhy{}}\PYG{n}{scripts}\PYG{o}{/}\PYG{n}{log}\PYG{o}{/}\PYG{n}{control\PYGZus{}urn\PYGZus{}vergabe}\PYG{o}{.}\PYG{n}{log}
\PYG{l+m+mi}{1} \PYG{l+m+mi}{1} \PYG{o}{*} \PYG{o}{*} \PYG{o}{*} \PYG{o}{/}\PYG{n}{opt}\PYG{o}{/}\PYG{n}{regal}\PYG{o}{/}\PYG{n}{src}\PYG{o}{/}\PYG{n}{regal}\PYG{o}{\PYGZhy{}}\PYG{n}{scripts}\PYG{o}{/}\PYG{n}{register\PYGZus{}urn}\PYG{o}{.}\PYG{n}{sh} \PYG{n}{katalog} \PYG{o}{\PYGZgt{}\PYGZgt{}} \PYG{o}{/}\PYG{n}{opt}\PYG{o}{/}\PYG{n}{regal}\PYG{o}{/}\PYG{n}{regal}\PYG{o}{\PYGZhy{}}\PYG{n}{scripts}\PYG{o}{/}\PYG{n}{log}\PYG{o}{/}\PYG{n}{katalog\PYGZus{}update}\PYG{o}{.}\PYG{n}{log}
\PYG{l+m+mi}{1} \PYG{l+m+mi}{0} \PYG{o}{*} \PYG{o}{*} \PYG{o}{*} \PYG{o}{/}\PYG{n}{opt}\PYG{o}{/}\PYG{n}{regal}\PYG{o}{/}\PYG{n}{src}\PYG{o}{/}\PYG{n}{regal}\PYG{o}{\PYGZhy{}}\PYG{n}{scripts}\PYG{o}{/}\PYG{n}{register\PYGZus{}urn}\PYG{o}{.}\PYG{n}{sh} \PYG{n}{register} \PYG{o}{\PYGZgt{}\PYGZgt{}} \PYG{o}{/}\PYG{n}{opt}\PYG{o}{/}\PYG{n}{regal}\PYG{o}{/}\PYG{n}{regal}\PYG{o}{\PYGZhy{}}\PYG{n}{scripts}\PYG{o}{/}\PYG{n}{log}\PYG{o}{/}\PYG{n}{register\PYGZus{}urn}\PYG{o}{.}\PYG{n}{log}
\end{sphinxVerbatim}


\subsection{Katalog\sphinxhyphen{}Aktualisierung}
\label{\detokenize{toscience:katalog-aktualisierung}}\label{\detokenize{toscience:id78}}
\sphinxAtStartPar
Das System gleicht einmal am Tag Metadaten mit dem hbz\sphinxhyphen{}Verbundkatalog ab
und führt ggf. Aktualisierungen durch.

\begin{sphinxVerbatim}[commandchars=\\\{\}]
\PYG{l+m+mi}{0} \PYG{l+m+mi}{5} \PYG{o}{*} \PYG{o}{*} \PYG{o}{*} \PYG{o}{/}\PYG{n}{opt}\PYG{o}{/}\PYG{n}{regal}\PYG{o}{/}\PYG{n}{src}\PYG{o}{/}\PYG{n}{regal}\PYG{o}{\PYGZhy{}}\PYG{n}{scripts}\PYG{o}{/}\PYG{n}{updateAll}\PYG{o}{.}\PYG{n}{sh} \PYG{o}{\PYGZgt{}} \PYG{o}{/}\PYG{n}{dev}\PYG{o}{/}\PYG{n}{null}
\end{sphinxVerbatim}


\subsection{Matomo}
\label{\detokenize{toscience:matomo-2}}\label{\detokenize{toscience:id79}}
\sphinxAtStartPar
Matomo wird mit Apache\sphinxhyphen{}Logfiles befüllt. Innerhalb von Matomo werden die
Einträge annonymisiert.

\begin{sphinxVerbatim}[commandchars=\\\{\}]
\PYG{l+m+mi}{0} \PYG{l+m+mi}{1} \PYG{o}{*} \PYG{o}{*} \PYG{o}{*} \PYG{o}{/}\PYG{n}{opt}\PYG{o}{/}\PYG{n}{regal}\PYG{o}{/}\PYG{n}{regal}\PYG{o}{\PYGZhy{}}\PYG{n}{scripts}\PYG{o}{/}\PYG{n}{import}\PYG{o}{\PYGZhy{}}\PYG{n}{logfiles}\PYG{o}{.}\PYG{n}{sh} \PYG{o}{\PYGZgt{}}\PYG{o}{/}\PYG{n}{dev}\PYG{o}{/}\PYG{n}{null}
\end{sphinxVerbatim}


\subsection{Logfile Annonymisierung}
\label{\detokenize{toscience:logfile-annonymisierung}}\label{\detokenize{toscience:id80}}
\sphinxAtStartPar
Apache\sphinxhyphen{}Logfiles werden sieben Tage unverändert aufbewahrt. Danach
erfolgt eine Annonymisierung.

\begin{sphinxVerbatim}[commandchars=\\\{\}]
\PYG{l+m+mi}{0} \PYG{l+m+mi}{2} \PYG{o}{*} \PYG{o}{*} \PYG{o}{*} \PYG{o}{/}\PYG{n}{opt}\PYG{o}{/}\PYG{n}{regal}\PYG{o}{/}\PYG{n}{src}\PYG{o}{/}\PYG{n}{regal}\PYG{o}{\PYGZhy{}}\PYG{n}{scripts}\PYG{o}{/}\PYG{n}{depersonalize}\PYG{o}{\PYGZhy{}}\PYG{n}{apache}\PYG{o}{\PYGZhy{}}\PYG{n}{logs}\PYG{o}{.}\PYG{n}{sh}
\end{sphinxVerbatim}


\subsection{Webgatherer}
\label{\detokenize{toscience:webgatherer}}\label{\detokenize{toscience:id81}}
\sphinxAtStartPar
Der Webgatherer prüft Archivierungsintervalle von Webpages und stößt bei
Bedarf die Erzeugung eines neuen Snapshots/Version an.

\begin{sphinxVerbatim}[commandchars=\\\{\}]
\PYG{l+m+mi}{0} \PYG{l+m+mi}{20} \PYG{o}{*} \PYG{o}{*} \PYG{o}{*} \PYG{o}{/}\PYG{n}{opt}\PYG{o}{/}\PYG{n}{regal}\PYG{o}{/}\PYG{n}{src}\PYG{o}{/}\PYG{n}{regal}\PYG{o}{\PYGZhy{}}\PYG{n}{scripts}\PYG{o}{/}\PYG{n}{runGatherer}\PYG{o}{.}\PYG{n}{sh} \PYG{o}{\PYGZgt{}\PYGZgt{}} \PYG{o}{/}\PYG{n}{opt}\PYG{o}{/}\PYG{n}{regal}\PYG{o}{/}\PYG{n}{regal}\PYG{o}{\PYGZhy{}}\PYG{n}{scripts}\PYG{o}{/}\PYG{n}{log}\PYG{o}{/}\PYG{n}{runGatherer}\PYG{o}{.}\PYG{n}{log}
\PYG{c+c1}{\PYGZsh{} Auswertung des letzten Webgatherer\PYGZhy{}Laufs}
\PYG{l+m+mi}{0} \PYG{l+m+mi}{21} \PYG{o}{*} \PYG{o}{*} \PYG{o}{*} \PYG{o}{/}\PYG{n}{opt}\PYG{o}{/}\PYG{n}{regal}\PYG{o}{/}\PYG{n}{src}\PYG{o}{/}\PYG{n}{regal}\PYG{o}{\PYGZhy{}}\PYG{n}{scripts}\PYG{o}{/}\PYG{n}{evalWebgatherer}\PYG{o}{.}\PYG{n}{sh} \PYG{o}{\PYGZgt{}\PYGZgt{}} \PYG{o}{/}\PYG{n}{opt}\PYG{o}{/}\PYG{n}{regal}\PYG{o}{/}\PYG{n}{regal}\PYG{o}{\PYGZhy{}}\PYG{n}{scripts}\PYG{o}{/}\PYG{n}{log}\PYG{o}{/}\PYG{n}{runGatherer}\PYG{o}{.}\PYG{n}{log}
\PYG{c+c1}{\PYGZsh{} Crawl Reports}
\PYG{l+m+mi}{0} \PYG{l+m+mi}{22} \PYG{o}{*} \PYG{o}{*} \PYG{o}{*} \PYG{o}{/}\PYG{n}{opt}\PYG{o}{/}\PYG{n}{regal}\PYG{o}{/}\PYG{n}{src}\PYG{o}{/}\PYG{n}{regal}\PYG{o}{\PYGZhy{}}\PYG{n}{scripts}\PYG{o}{/}\PYG{n}{crawlReport}\PYG{o}{.}\PYG{n}{sh} \PYG{o}{\PYGZgt{}\PYGZgt{}} \PYG{o}{/}\PYG{n}{opt}\PYG{o}{/}\PYG{n}{regal}\PYG{o}{/}\PYG{n}{logs}\PYG{o}{/}\PYG{n}{crawlReport}\PYG{o}{.}\PYG{n}{log}
\end{sphinxVerbatim}


\subsection{Backup}
\label{\detokenize{toscience:backup}}\label{\detokenize{toscience:id82}}
\sphinxAtStartPar
MySQL und Elasticsearch

\sphinxAtStartPar
Der Elasticsearch\sphinxhyphen{}Index und die MySQL\sphinxhyphen{}Datenbanken werden täglich
gesichert. Es werden Backups der letzten 30 Tage aufbewahrt. Ältere
Backups werden von der Platte gelöscht.

\begin{sphinxVerbatim}[commandchars=\\\{\}]
\PYG{l+m+mi}{0} \PYG{l+m+mi}{2} \PYG{o}{*} \PYG{o}{*} \PYG{o}{*} \PYG{o}{/}\PYG{n}{opt}\PYG{o}{/}\PYG{n}{regal}\PYG{o}{/}\PYG{n}{src}\PYG{o}{/}\PYG{n}{regal}\PYG{o}{\PYGZhy{}}\PYG{n}{scripts}\PYG{o}{/}\PYG{n}{backup}\PYG{o}{\PYGZhy{}}\PYG{n}{es}\PYG{o}{.}\PYG{n}{sh} \PYG{o}{\PYGZhy{}}\PYG{n}{c} \PYG{o}{\PYGZgt{}\PYGZgt{}} \PYG{o}{/}\PYG{n}{opt}\PYG{o}{/}\PYG{n}{regal}\PYG{o}{/}\PYG{n}{logs}\PYG{o}{/}\PYG{n}{backup}\PYG{o}{\PYGZhy{}}\PYG{n}{es}\PYG{o}{.}\PYG{n}{log} \PYG{l+m+mi}{2}\PYG{o}{\PYGZgt{}}\PYG{o}{\PYGZam{}}\PYG{l+m+mi}{1}
\PYG{l+m+mi}{30} \PYG{l+m+mi}{2} \PYG{o}{*} \PYG{o}{*} \PYG{o}{*} \PYG{o}{/}\PYG{n}{opt}\PYG{o}{/}\PYG{n}{regal}\PYG{o}{/}\PYG{n}{src}\PYG{o}{/}\PYG{n}{regal}\PYG{o}{\PYGZhy{}}\PYG{n}{scripts}\PYG{o}{/}\PYG{n}{backup}\PYG{o}{\PYGZhy{}}\PYG{n}{es}\PYG{o}{.}\PYG{n}{sh} \PYG{o}{\PYGZhy{}}\PYG{n}{b} \PYG{o}{\PYGZgt{}\PYGZgt{}} \PYG{o}{/}\PYG{n}{opt}\PYG{o}{/}\PYG{n}{regal}\PYG{o}{/}\PYG{n}{logs}\PYG{o}{/}\PYG{n}{backup}\PYG{o}{\PYGZhy{}}\PYG{n}{es}\PYG{o}{.}\PYG{n}{log} \PYG{l+m+mi}{2}\PYG{o}{\PYGZgt{}}\PYG{o}{\PYGZam{}}\PYG{l+m+mi}{1}
\PYG{l+m+mi}{0} \PYG{l+m+mi}{2} \PYG{o}{*} \PYG{o}{*} \PYG{o}{*} \PYG{o}{/}\PYG{n}{opt}\PYG{o}{/}\PYG{n}{regal}\PYG{o}{/}\PYG{n}{src}\PYG{o}{/}\PYG{n}{regal}\PYG{o}{\PYGZhy{}}\PYG{n}{scripts}\PYG{o}{/}\PYG{n}{backup}\PYG{o}{\PYGZhy{}}\PYG{n}{db}\PYG{o}{.}\PYG{n}{sh} \PYG{o}{\PYGZhy{}}\PYG{n}{c} \PYG{o}{\PYGZgt{}\PYGZgt{}} \PYG{o}{/}\PYG{n}{opt}\PYG{o}{/}\PYG{n}{regal}\PYG{o}{/}\PYG{n}{logs}\PYG{o}{/}\PYG{n}{backup}\PYG{o}{\PYGZhy{}}\PYG{n}{db}\PYG{o}{.}\PYG{n}{log} \PYG{l+m+mi}{2}\PYG{o}{\PYGZgt{}}\PYG{o}{\PYGZam{}}\PYG{l+m+mi}{1}
\PYG{l+m+mi}{30} \PYG{l+m+mi}{2} \PYG{o}{*} \PYG{o}{*} \PYG{o}{*} \PYG{o}{/}\PYG{n}{opt}\PYG{o}{/}\PYG{n}{regal}\PYG{o}{/}\PYG{n}{src}\PYG{o}{/}\PYG{n}{regal}\PYG{o}{\PYGZhy{}}\PYG{n}{scripts}\PYG{o}{/}\PYG{n}{backup}\PYG{o}{\PYGZhy{}}\PYG{n}{db}\PYG{o}{.}\PYG{n}{sh} \PYG{o}{\PYGZhy{}}\PYG{n}{b} \PYG{o}{\PYGZgt{}\PYGZgt{}} \PYG{o}{/}\PYG{n}{opt}\PYG{o}{/}\PYG{n}{regal}\PYG{o}{/}\PYG{n}{logs}\PYG{o}{/}\PYG{n}{backup}\PYG{o}{\PYGZhy{}}\PYG{n}{db}\PYG{o}{.}\PYG{n}{log} \PYG{l+m+mi}{2}\PYG{o}{\PYGZgt{}}\PYG{o}{\PYGZam{}}\PYG{l+m+mi}{1}
\end{sphinxVerbatim}


\subsection{Entwicklung}
\label{\detokenize{toscience:entwicklung}}\label{\detokenize{toscience:id83}}
\sphinxAtStartPar
Für die Entwicklung an Regal empfiehlt sich folgende Vorgehensweise…


\chapter{API\sphinxhyphen{}documentation}
\label{\detokenize{api:api-documentation}}\label{\detokenize{api:id1}}\label{\detokenize{api::doc}}

\section{Preface}
\label{\detokenize{api:preface}}\label{\detokenize{api:id2}}
\sphinxAtStartPar
The Regal webservices documented by example \sphinxcode{\sphinxupquote{curl}}\sphinxhyphen{}calls. Examples are
assumed to work in the Vagrant\sphinxhyphen{}Environment that comes with this
document.


\section{Environment}
\label{\detokenize{api:environment}}\label{\detokenize{api:id3}}
\sphinxAtStartPar
Got to your server or to the Vagrant\sphinxhyphen{}Box, that comes with this document.

\sphinxAtStartPar
\sphinxcode{\sphinxupquote{vagrant ssh}}

\sphinxAtStartPar
Prepare your environment to make the following \sphinxcode{\sphinxupquote{curl}}\sphinxhyphen{}Calls work!

\begin{sphinxVerbatim}[commandchars=\\\{\}]
\PYG{n+nb}{source} /opt/regal/conf/variables.conf
\PYG{n+nb}{export} \PYG{n+nv}{REGAL\PYGZus{}API}\PYG{o}{=}http://\PYG{n+nv}{\PYGZdl{}SERVER}
\PYG{n+nb}{export} \PYG{n+nv}{API\PYGZus{}USER}\PYG{o}{=}edoweb\PYGZhy{}admin
\end{sphinxVerbatim}


\subsection{to.science.api}
\label{\detokenize{api-toscience:to-science-api}}\label{\detokenize{api-toscience:id1}}\label{\detokenize{api-toscience::doc}}
\sphinxAtStartPar
\sphinxurl{https://github.com/hbz/to.science.api/blob/master/conf/routes}


\subsubsection{Create}
\label{\detokenize{api-toscience:create}}\label{\detokenize{api-toscience:id2}}

\paragraph{Create a new resource}
\label{\detokenize{api-toscience:create-a-new-resource}}\label{\detokenize{api-toscience:id3}}
\begin{sphinxVerbatim}[commandchars=\\\{\}]
curl \PYGZhy{}i \PYGZhy{}u\PYGZdl{}API\PYGZus{}USER:\PYGZdl{}PASSWORD \PYGZhy{}XPUT \PYGZdl{}REGAL\PYGZus{}API/resource/regal:1234 \PYGZhy{}d\PYGZsq{}\PYGZob{}\PYGZdq{}contentType\PYGZdq{}:\PYGZdq{}monograph\PYGZdq{},\PYGZdq{}accessScheme\PYGZdq{}:\PYGZdq{}public\PYGZdq{}\PYGZcb{}\PYGZsq{} \PYGZhy{}H\PYGZsq{}content\PYGZhy{}type:application/json\PYGZsq{}
\end{sphinxVerbatim}


\paragraph{Create a new hierarchy}
\label{\detokenize{api-toscience:create-a-new-hierarchy}}\label{\detokenize{api-toscience:id4}}
\begin{sphinxVerbatim}[commandchars=\\\{\}]
curl \PYGZhy{}i \PYGZhy{}u\PYGZdl{}API\PYGZus{}USER:\PYGZdl{}PASSWORD \PYGZhy{}XPUT \PYGZdl{}REGAL\PYGZus{}API/resource/regal:1235 \PYGZhy{}d\PYGZsq{}\PYGZob{}\PYGZdq{}parentPid\PYGZdq{}:\PYGZdq{}regal:1234\PYGZdq{},\PYGZdq{}contentType\PYGZdq{}:\PYGZdq{}file\PYGZdq{},\PYGZdq{}accessScheme\PYGZdq{}:\PYGZdq{}public\PYGZdq{}\PYGZcb{}\PYGZsq{} \PYGZhy{}H\PYGZsq{}content\PYGZhy{}type:application/json\PYGZsq{}
\end{sphinxVerbatim}


\paragraph{Upload binary data}
\label{\detokenize{api-toscience:upload-binary-data}}\label{\detokenize{api-toscience:id5}}
\begin{sphinxVerbatim}[commandchars=\\\{\}]
curl \PYGZhy{}u\PYGZdl{}API\PYGZus{}USER:\PYGZdl{}PASSWORD \PYGZhy{}F\PYGZdq{}data=@\PYGZdl{}ARCHIVE\PYGZus{}HOME/src/REGAL\PYGZus{}API/test/resources/test.pdf;type=application/pdf\PYGZdq{} \PYGZhy{}XPUT \PYGZdl{}REGAL\PYGZus{}API/resource/regal:1235/data
\end{sphinxVerbatim}


\paragraph{Create User}
\label{\detokenize{api-toscience:create-user}}\label{\detokenize{api-toscience:id6}}
\begin{sphinxVerbatim}[commandchars=\\\{\}]
curl \PYGZhy{}u\PYGZdl{}API\PYGZus{}USER:\PYGZdl{}PASSWORD \PYGZhy{}d\PYGZsq{}\PYGZob{}\PYGZdq{}username\PYGZdq{}:\PYGZdq{}test\PYGZdq{},\PYGZdq{}password\PYGZdq{}:\PYGZdq{}test\PYGZdq{},\PYGZdq{}email\PYGZdq{}:\PYGZdq{}test@example.org\PYGZdq{},\PYGZdq{}role\PYGZdq{}:\PYGZdq{}EDITOR\PYGZdq{}\PYGZcb{}\PYGZsq{} \PYGZhy{}XPUT \PYGZdl{}REGAL\PYGZus{}API/utils/addUser \PYGZhy{}H\PYGZsq{}content\PYGZhy{}type:application/json\PYGZsq{}
\end{sphinxVerbatim}


\paragraph{Upload metadata}
\label{\detokenize{api-toscience:upload-metadata}}\label{\detokenize{api-toscience:id7}}
\begin{sphinxVerbatim}[commandchars=\\\{\}]
curl \PYGZhy{}XPUT \PYGZhy{}u\PYGZdl{}API\PYGZus{}USER:\PYGZdl{}PASSWORD \PYGZhy{}d\PYGZsq{}\PYGZlt{}regal:1234\PYGZgt{} \PYGZlt{}dc:title\PYGZgt{} \PYGZdq{}Ein Test Titel\PYGZdq{} .\PYGZsq{} \PYGZhy{}H\PYGZdq{}content\PYGZhy{}type:text/plain\PYGZdq{} \PYGZdl{}REGAL\PYGZus{}API/resource/regal:1235/metadata2
\end{sphinxVerbatim}


\paragraph{Order Child Nodes}
\label{\detokenize{api-toscience:order-child-nodes}}
\begin{sphinxVerbatim}[commandchars=\\\{\}]
\PYGZdl{} curl \PYGZhy{}XPUT \PYGZhy{}u\PYG{n+nv}{\PYGZdl{}API\PYGZus{}USER}:\PYG{n+nv}{\PYGZdl{}PASSWORD} \PYGZhy{}d\PYG{l+s+s1}{\PYGZsq{}[\PYGZdq{}regal:2\PYGZdq{},\PYGZdq{}regal:1249\PYGZdq{}]\PYGZsq{}} \PYG{n+nv}{\PYGZdl{}REGAL\PYGZus{}API}/resource/regal:1/parts \PYGZhy{}H\PYG{l+s+s2}{\PYGZdq{}Content\PYGZhy{}Type:application/json\PYGZdq{}}
\end{sphinxVerbatim}


\paragraph{Ingest unmanaged content}
\label{\detokenize{api-toscience:ingest-unmanaged-content}}\label{\detokenize{api-toscience:id8}}
\sphinxAtStartPar
Example address for external stored content, i.e. research data:
\sphinxcode{\sphinxupquote{https://api.example.com/data/regal:1234/first\_set/data.csv}}

\sphinxAtStartPar
The base url and the default collection url are configured in the application.conf.

\sphinxAtStartPar
Currently only one level of subpaths is supported.


\begin{savenotes}\sphinxattablestart
\centering
\sphinxcapstartof{table}
\sphinxthecaptionisattop
\sphinxcaption{URL parameter}\label{\detokenize{api-toscience:id35}}
\sphinxaftertopcaption
\begin{tabulary}{\linewidth}[t]{|T|T|T|}
\hline
\sphinxstyletheadfamily 
\sphinxAtStartPar
parameter
&\sphinxstyletheadfamily 
\sphinxAtStartPar
default
&\sphinxstyletheadfamily 
\sphinxAtStartPar
description
\\
\hline
\sphinxAtStartPar
collectionUrl
&
\sphinxAtStartPar
data
&
\sphinxAtStartPar
Path to the storage folder
\\
\hline
\sphinxAtStartPar
subPath
&
\sphinxAtStartPar
\sphinxhyphen{}
&
\sphinxAtStartPar
optional: path of the subfolder, ‘first\_set’ in above example
\\
\hline
\sphinxAtStartPar
filename
&
\sphinxAtStartPar
\sphinxhyphen{}
&
\sphinxAtStartPar
bare filename, but with extension
\\
\hline
\sphinxAtStartPar
resourcePid
&
\sphinxAtStartPar
\textless{}empty\textgreater{}
&
\sphinxAtStartPar
automatically assigned pid of the external resource
\\
\hline
\end{tabulary}
\par
\sphinxattableend\end{savenotes}

\begin{sphinxVerbatim}[commandchars=\\\{\}]
\PYGZdl{} curl \PYGZhy{}XPOST \PYGZhy{}u\PYG{n+nv}{\PYGZdl{}API\PYGZus{}USER}:\PYG{n+nv}{\PYGZdl{}PASSWORD} \PYG{l+s+s2}{\PYGZdq{}}\PYG{n+nv}{\PYGZdl{}REGAL\PYGZus{}API}\PYG{l+s+s2}{/resource/regal:1234/postResearchData?collectionUrl=data\PYGZam{}subPath=}\PYG{n+nv}{\PYGZdl{}dataDir}\PYG{l+s+s2}{\PYGZam{}filename=}\PYG{n+nv}{\PYGZdl{}dateiname}\PYG{l+s+s2}{\PYGZam{}resourcePid=}\PYG{n+nv}{\PYGZdl{}resourcePid}\PYG{l+s+s2}{\PYGZdq{}} \PYGZhy{}H \PYG{l+s+s2}{\PYGZdq{}UserId=resourceposter\PYGZdq{}} \PYGZhy{}H \PYG{l+s+s2}{\PYGZdq{}Content\PYGZhy{}Type: text/plain; charset=utf\PYGZhy{}8\PYGZdq{}}\PYG{p}{;}
\end{sphinxVerbatim}


\subsubsection{Read}
\label{\detokenize{api-toscience:read}}\label{\detokenize{api-toscience:id9}}

\paragraph{Read resource}
\label{\detokenize{api-toscience:read-resource}}\label{\detokenize{api-toscience:id10}}
\sphinxAtStartPar
\sphinxstylestrong{html}

\begin{sphinxVerbatim}[commandchars=\\\{\}]
curl \PYGZdl{}REGAL\PYGZus{}API/resource/regal:1234.html
\end{sphinxVerbatim}

\sphinxAtStartPar
\sphinxstylestrong{json}

\begin{sphinxVerbatim}[commandchars=\\\{\}]
curl \PYGZdl{}REGAL\PYGZus{}API/resource/regal:1234.json
curl \PYGZdl{}REGAL\PYGZus{}API/resource/regal:1234.json2
\end{sphinxVerbatim}

\sphinxAtStartPar
\sphinxstylestrong{rdf}

\begin{sphinxVerbatim}[commandchars=\\\{\}]
curl \PYGZdl{}REGAL\PYGZus{}API/resource/regal:1234.rdf
\end{sphinxVerbatim}

\sphinxAtStartPar
\sphinxstylestrong{mets}

\begin{sphinxVerbatim}[commandchars=\\\{\}]
curl \PYGZdl{}REGAL\PYGZus{}API/resource/regal:1234.mets
\end{sphinxVerbatim}

\sphinxAtStartPar
\sphinxstylestrong{aleph}

\begin{sphinxVerbatim}[commandchars=\\\{\}]
curl \PYGZdl{}REGAL\PYGZus{}API/resource/regal:1234.aleph
\end{sphinxVerbatim}

\sphinxAtStartPar
\sphinxstylestrong{epicur}

\begin{sphinxVerbatim}[commandchars=\\\{\}]
curl \PYGZdl{}REGAL\PYGZus{}API/resource/regal:1234.epicur
\end{sphinxVerbatim}

\sphinxAtStartPar
\sphinxstylestrong{datacite}

\begin{sphinxVerbatim}[commandchars=\\\{\}]
curl \PYGZdl{}REGAL\PYGZus{}API/resource/regal:1234.datacite
\end{sphinxVerbatim}

\sphinxAtStartPar
\sphinxstylestrong{csv}

\begin{sphinxVerbatim}[commandchars=\\\{\}]
curl \PYGZdl{}REGAL\PYGZus{}API/resource/regal:1234.csv
\end{sphinxVerbatim}

\sphinxAtStartPar
\sphinxstylestrong{wgl}

\begin{sphinxVerbatim}[commandchars=\\\{\}]
curl \PYGZdl{}REGAL\PYGZus{}API/resource/regal:1234.wgl
\end{sphinxVerbatim}

\sphinxAtStartPar
\sphinxstylestrong{oaidc}

\begin{sphinxVerbatim}[commandchars=\\\{\}]
curl \PYGZdl{}REGAL\PYGZus{}API/resource/regal:1234.oaidc
\end{sphinxVerbatim}


\paragraph{Read resource tree}
\label{\detokenize{api-toscience:read-resource-tree}}\label{\detokenize{api-toscience:id11}}
\begin{sphinxVerbatim}[commandchars=\\\{\}]
curl \PYGZdl{}REGAL\PYGZus{}API/resource/regal:1234/all
\end{sphinxVerbatim}

\begin{sphinxVerbatim}[commandchars=\\\{\}]
curl \PYGZdl{}REGAL\PYGZus{}API/resource/regal:1234/parts
\end{sphinxVerbatim}


\paragraph{Read binary data}
\label{\detokenize{api-toscience:read-binary-data}}\label{\detokenize{api-toscience:id12}}
\begin{sphinxVerbatim}[commandchars=\\\{\}]
curl \PYGZdl{}REGAL\PYGZus{}API/resource/regal:1234/data
\end{sphinxVerbatim}


\paragraph{Read Webgatherer Conf}
\label{\detokenize{api-toscience:read-webgatherer-conf}}\label{\detokenize{api-toscience:id13}}
\begin{sphinxVerbatim}[commandchars=\\\{\}]
curl \PYGZdl{}REGAL\PYGZus{}API/resource/regal:1234/conf
\end{sphinxVerbatim}


\paragraph{Read Ordering of Childs}
\label{\detokenize{api-toscience:read-ordering-of-childs}}\label{\detokenize{api-toscience:id14}}
\begin{sphinxVerbatim}[commandchars=\\\{\}]
curl \PYGZdl{}REGAL\PYGZus{}API/resource/regal:1234/seq
\end{sphinxVerbatim}


\paragraph{Read user}
\label{\detokenize{api-toscience:read-user}}\label{\detokenize{api-toscience:id15}}
\begin{sphinxVerbatim}[commandchars=\\\{\}]
\PYG{o+ow}{not} \PYG{n}{implemented}
\end{sphinxVerbatim}


\paragraph{Read Adhoc Linked Data}
\label{\detokenize{api-toscience:read-adhoc-linked-data}}\label{\detokenize{api-toscience:id16}}
\begin{sphinxVerbatim}[commandchars=\\\{\}]
curl \PYGZdl{}REGAL\PYGZus{}API/adhoc/uri/\PYGZdl{}(echo test |base64)
\end{sphinxVerbatim}


\subsubsection{Update}
\label{\detokenize{api-toscience:update}}\label{\detokenize{api-toscience:id17}}

\paragraph{Update Resource}
\label{\detokenize{api-toscience:update-resource}}\label{\detokenize{api-toscience:id18}}

\paragraph{Update Metadata}
\label{\detokenize{api-toscience:update-metadata}}\label{\detokenize{api-toscience:id19}}
\begin{sphinxVerbatim}[commandchars=\\\{\}]
curl \PYGZhy{}s \PYGZhy{}u\PYGZdl{}API\PYGZus{}USER:\PYGZdl{}REGAL\PYGZus{}PASSWORD \PYGZhy{}XPOST \PYGZdl{}REGAL\PYGZus{}API/utils/updateMetadata/regal:1234 \PYGZhy{}H\PYGZdq{}accept: application/json\PYGZdq{}
\end{sphinxVerbatim}


\paragraph{Add URN}
\label{\detokenize{api-toscience:add-urn}}\label{\detokenize{api-toscience:id20}}
\begin{sphinxVerbatim}[commandchars=\\\{\}]
\PYG{n}{POST} \PYG{o}{/}\PYG{n}{utils}\PYG{o}{/}\PYG{n}{lobidify}
\end{sphinxVerbatim}

\begin{sphinxVerbatim}[commandchars=\\\{\}]
\PYG{n}{POST} \PYG{o}{/}\PYG{n}{utils}\PYG{o}{/}\PYG{n}{addUrn}
\end{sphinxVerbatim}

\begin{sphinxVerbatim}[commandchars=\\\{\}]
\PYG{n}{POST} \PYG{o}{/}\PYG{n}{utils}\PYG{o}{/}\PYG{n}{replaceUrn}
\end{sphinxVerbatim}


\paragraph{Enrich}
\label{\detokenize{api-toscience:enrich}}\label{\detokenize{api-toscience:id21}}
\begin{sphinxVerbatim}[commandchars=\\\{\}]
\PYG{n}{POST} \PYG{o}{/}\PYG{n}{resource}\PYG{o}{/}\PYG{p}{:}\PYG{n}{pid}\PYG{o}{/}\PYG{n}{metadata}\PYG{o}{/}\PYG{n}{enrich}
\end{sphinxVerbatim}


\subsubsection{Delete}
\label{\detokenize{api-toscience:delete}}\label{\detokenize{api-toscience:id22}}

\paragraph{Delete resource}
\label{\detokenize{api-toscience:delete-resource}}\label{\detokenize{api-toscience:id23}}
\begin{sphinxVerbatim}[commandchars=\\\{\}]
curl \PYGZhy{}u\PYGZdl{}API\PYGZus{}USER:\PYGZdl{}REGAL\PYGZus{}PASSWORD \PYGZhy{}XDELETE \PYGZdq{}\PYGZdl{}REGAL\PYGZus{}API/resource/regal:1234\PYGZdq{};echo
\end{sphinxVerbatim}


\paragraph{Purge resource}
\label{\detokenize{api-toscience:purge-resource}}\label{\detokenize{api-toscience:id24}}
\begin{sphinxVerbatim}[commandchars=\\\{\}]
curl \PYGZhy{}u\PYGZdl{}API\PYGZus{}USER:\PYGZdl{}REGAL\PYGZus{}PASSWORD \PYGZhy{}XDELETE \PYGZdq{}\PYGZdl{}REGAL\PYGZus{}API/resource/regal:1234?purge=true\PYGZdq{};echo
\end{sphinxVerbatim}


\paragraph{Delete part of resource}
\label{\detokenize{api-toscience:delete-part-of-resource}}\label{\detokenize{api-toscience:id25}}
\begin{sphinxVerbatim}[commandchars=\\\{\}]
curl \PYGZhy{}u\PYGZdl{}API\PYGZus{}USER:\PYGZdl{}REGAL\PYGZus{}PASSWORD \PYGZhy{}XDELETE \PYGZdl{}REGAL\PYGZus{}API/resource/regal:1234/seq
\end{sphinxVerbatim}

\begin{sphinxVerbatim}[commandchars=\\\{\}]
curl \PYGZhy{}u\PYGZdl{}API\PYGZus{}USER:\PYGZdl{}REGAL\PYGZus{}PASSWORD \PYGZhy{}XDELETE \PYGZdl{}REGAL\PYGZus{}API/resource/regal:1234/metadata
\end{sphinxVerbatim}

\begin{sphinxVerbatim}[commandchars=\\\{\}]
curl \PYGZhy{}u\PYGZdl{}API\PYGZus{}USER:\PYGZdl{}REGAL\PYGZus{}PASSWORD \PYGZhy{}XDELETE \PYGZdl{}REGAL\PYGZus{}API/resource/regal:1234/metadata2
\end{sphinxVerbatim}

\begin{sphinxVerbatim}[commandchars=\\\{\}]
curl \PYGZhy{}u\PYGZdl{}API\PYGZus{}USER:\PYGZdl{}REGAL\PYGZus{}PASSWORD \PYGZhy{}XDELETE \PYGZdl{}REGAL\PYGZus{}API/resource/regal:1234/data
\end{sphinxVerbatim}

\begin{sphinxVerbatim}[commandchars=\\\{\}]
curl \PYGZhy{}u\PYGZdl{}API\PYGZus{}USER:\PYGZdl{}REGAL\PYGZus{}PASSWORD \PYGZhy{}XDELETE \PYGZdl{}REGAL\PYGZus{}API/resource/regal:1234/dc
\end{sphinxVerbatim}


\paragraph{Delete user}
\label{\detokenize{api-toscience:delete-user}}\label{\detokenize{api-toscience:id26}}
\begin{sphinxVerbatim}[commandchars=\\\{\}]
\PYG{o+ow}{not} \PYG{n}{implemented}
\end{sphinxVerbatim}


\subsubsection{Search}
\label{\detokenize{api-toscience:search}}\label{\detokenize{api-toscience:id27}}

\paragraph{Simple Search}
\label{\detokenize{api-toscience:simple-search}}\label{\detokenize{api-toscience:id28}}
\begin{sphinxVerbatim}[commandchars=\\\{\}]
\PYG{n}{GET} \PYG{o}{/}\PYG{n}{find}
\end{sphinxVerbatim}

\begin{sphinxVerbatim}[commandchars=\\\{\}]
\PYG{n}{GET} \PYG{o}{/}\PYG{n}{resource}
\end{sphinxVerbatim}


\paragraph{Facetted Search}
\label{\detokenize{api-toscience:facetted-search}}\label{\detokenize{api-toscience:id29}}

\paragraph{Search for field}
\label{\detokenize{api-toscience:search-for-field}}\label{\detokenize{api-toscience:id30}}

\subsubsection{Misc}
\label{\detokenize{api-toscience:misc}}\label{\detokenize{api-toscience:id31}}

\paragraph{Load metadata from Lobid}
\label{\detokenize{api-toscience:load-metadata-from-lobid}}\label{\detokenize{api-toscience:id32}}
\begin{sphinxVerbatim}[commandchars=\\\{\}]
curl \PYGZhy{}u\PYGZdl{}API\PYGZus{}USER:\PYGZdl{}PASSWORD \PYGZhy{}XPOST \PYGZdq{}\PYGZdl{}REGAL\PYGZus{}API/utils/lobidify/regal:1234?alephid=HT018920238\PYGZdq{}
\end{sphinxVerbatim}


\paragraph{Reread Labels from etikett}
\label{\detokenize{api-toscience:reread-labels-from-etikett}}\label{\detokenize{api-toscience:id33}}
\begin{sphinxVerbatim}[commandchars=\\\{\}]
curl \PYGZhy{}u\PYGZdl{}API\PYGZus{}USER:\PYGZdl{}PASSWORD \PYGZhy{}XPOST \PYGZdl{}REGAL\PYGZus{}API/context.json
\end{sphinxVerbatim}


\paragraph{Reindex resource}
\label{\detokenize{api-toscience:reindex-resource}}\label{\detokenize{api-toscience:id34}}
\begin{sphinxVerbatim}[commandchars=\\\{\}]
curl \PYGZhy{}u\PYGZdl{}API\PYGZus{}USER:\PYGZdl{}PASSWORD \PYGZhy{}XPOST \PYGZdl{}REGAL\PYGZus{}API/utils/index/regal:1234 \PYGZhy{}H\PYGZdq{}accept: application/json\PYGZdq{}
\end{sphinxVerbatim}


\subsection{to.science.labels}
\label{\detokenize{api-labels:to-science-labels}}\label{\detokenize{api-labels:etikett}}\label{\detokenize{api-labels::doc}}
\sphinxAtStartPar
\sphinxurl{https://github.com/hbz/to.science.labels/blob/master/conf/routes}


\subsubsection{Create}
\label{\detokenize{api-labels:create}}\label{\detokenize{api-labels:create-2}}

\paragraph{Add Labels to Database}
\label{\detokenize{api-labels:add-labels-to-database}}\label{\detokenize{api-labels:id1}}
\begin{sphinxVerbatim}[commandchars=\\\{\}]
curl \PYGZhy{}u\PYGZdl{}API\PYGZus{}USER:\PYGZdl{}PASSWORD \PYGZhy{}XPOST \PYGZhy{}F\PYGZdq{}data=@\PYGZdl{}ARCHIVE\PYGZus{}HOME/src/REGAL\PYGZus{}API/conf/labels.json\PYGZdq{} \PYGZhy{}F\PYGZdq{}format\PYGZhy{}cb=Json\PYGZdq{} \PYGZdl{}REGAL\PYGZus{}API/tools/etikett \PYGZhy{}i \PYGZhy{}L
\end{sphinxVerbatim}


\paragraph{Add Label}
\label{\detokenize{api-labels:add-label}}\label{\detokenize{api-labels:id2}}

\subsubsection{Read}
\label{\detokenize{api-labels:read}}\label{\detokenize{api-labels:read-2}}
\begin{sphinxVerbatim}[commandchars=\\\{\}]
\PYG{n}{curl} \PYG{l+s+s2}{\PYGZdq{}}\PYG{l+s+s2}{\PYGZdl{}REGAL\PYGZus{}API/tools/etikett}\PYG{l+s+s2}{\PYGZdq{}} \PYG{o}{\PYGZhy{}}\PYG{n}{H}\PYG{l+s+s2}{\PYGZdq{}}\PYG{l+s+s2}{accept: application/json}\PYG{l+s+s2}{\PYGZdq{}}
\end{sphinxVerbatim}


\paragraph{Read Etikett}
\label{\detokenize{api-labels:read-etikett}}\label{\detokenize{api-labels:id3}}
\begin{sphinxVerbatim}[commandchars=\\\{\}]
curl \PYGZdl{}REGAL\PYGZus{}API/tools/etikett?url=http\PYGZpc{}3A\PYGZpc{}2F\PYGZpc{}2Fpurl.orms\PYGZpc{}2Fissued \PYGZhy{}H\PYGZdq{}accept: application/json\PYGZdq{}
\end{sphinxVerbatim}


\subsubsection{Update}
\label{\detokenize{api-labels:update}}\label{\detokenize{api-labels:update-2}}

\subsubsection{Delete}
\label{\detokenize{api-labels:delete}}\label{\detokenize{api-labels:delete-2}}

\paragraph{Delete Cache}
\label{\detokenize{api-labels:delete-cache}}\label{\detokenize{api-labels:id4}}
\begin{sphinxVerbatim}[commandchars=\\\{\}]
curl \PYGZhy{}XDELETE \PYGZhy{}u\PYGZdl{}API\PYGZus{}USER:\PYGZdl{}PASSWORD \PYGZdl{}REGAL\PYGZus{}API/tools/etikett/cache
\end{sphinxVerbatim}


\subsubsection{Misc}
\label{\detokenize{api-labels:misc}}\label{\detokenize{api-labels:misc-2}}

\subsection{to.science.forms}
\label{\detokenize{api-forms:to-science-forms}}\label{\detokenize{api-forms:id1}}\label{\detokenize{api-forms::doc}}
\sphinxAtStartPar
\sphinxurl{https://github.com/hbz/to.science.forms/blob/master/conf/routes}


\subsubsection{Create}
\label{\detokenize{api-forms:create}}\label{\detokenize{api-forms:forms-create}}

\paragraph{Create RDF\sphinxhyphen{}Metadata from Form\sphinxhyphen{}Data}
\label{\detokenize{api-forms:create-rdf-metadata-from-form-data}}\label{\detokenize{api-forms:id2}}

\subsubsection{Read}
\label{\detokenize{api-forms:read}}\label{\detokenize{api-forms:forms-read}}

\paragraph{Read HTML\sphinxhyphen{}Form}
\label{\detokenize{api-forms:read-html-form}}\label{\detokenize{api-forms:id3}}

\subsubsection{Search}
\label{\detokenize{api-forms:search}}\label{\detokenize{api-forms:forms-search}}

\subsection{to.science.thumbs}
\label{\detokenize{api-thumbs:to-science-thumbs}}\label{\detokenize{api-thumbs:id1}}\label{\detokenize{api-thumbs::doc}}
\sphinxAtStartPar
\sphinxurl{https://github.com/hbz/thumby/blob/master/conf/routes}


\subsubsection{Read}
\label{\detokenize{api-thumbs:read}}\label{\detokenize{api-thumbs:read-5}}
\begin{sphinxVerbatim}[commandchars=\\\{\}]
\PYG{n}{curl} \PYG{o}{\PYGZhy{}}\PYG{n}{XGET} \PYG{l+s+s2}{\PYGZdq{}}\PYG{l+s+s2}{\PYGZdl{}REGAL\PYGZus{}API/tools/thumby?url=https://www.gravatar.com/avatar/5fefc19b7875e951c7ea9bfdfc06676d\PYGZam{}size=200}\PYG{l+s+s2}{\PYGZdq{}}
\end{sphinxVerbatim}


\subsection{skos\sphinxhyphen{}lookup}
\label{\detokenize{api-skos:skos-lookup}}\label{\detokenize{api-skos:api-skos-lookup}}\label{\detokenize{api-skos::doc}}
\sphinxAtStartPar
\sphinxurl{https://github.com/hbz/skos-lookup/blob/master/conf/routes}


\subsubsection{Create}
\label{\detokenize{api-skos:create}}\label{\detokenize{api-skos:create-4}}

\paragraph{Create new Index}
\label{\detokenize{api-skos:create-new-index}}\label{\detokenize{api-skos:id1}}
\begin{sphinxVerbatim}[commandchars=\\\{\}]
curl \PYGZhy{}i \PYGZhy{}X POST \PYGZhy{}H \PYGZdq{}Content\PYGZhy{}Type: multipart/form\PYGZhy{}data\PYGZdq{} \PYGZdl{}REGAL\PYGZus{}API/tools/skos\PYGZhy{}lookup/upload \PYGZhy{}F \PYGZdq{}data=@/tmp/skos\PYGZhy{}lookup/test/resources/agrovoc\PYGZus{}2016\PYGZhy{}07\PYGZhy{}15\PYGZus{}lod.nt.gz\PYGZdq{} \PYGZhy{}F\PYGZdq{}index=agrovoc\PYGZus{}test\PYGZdq{} \PYGZhy{}F\PYGZdq{}format=NTRIPLES\PYGZdq{}
\end{sphinxVerbatim}


\subsubsection{Read}
\label{\detokenize{api-skos:read}}\label{\detokenize{api-skos:read-4}}
\begin{sphinxVerbatim}[commandchars=\\\{\}]
\PYG{n}{curl} \PYG{o}{\PYGZhy{}}\PYG{n}{XGET} \PYG{l+s+s1}{\PYGZsq{}}\PYG{l+s+s1}{\PYGZdl{}REGAL\PYGZus{}API/tools/skos\PYGZhy{}lookup/autocomplete?lang=de\PYGZam{}q=Erdnus\PYGZam{}callback=mycallback\PYGZam{}index=agrovoc\PYGZus{}test}\PYG{l+s+s1}{\PYGZsq{}}
\end{sphinxVerbatim}


\subsubsection{Search}
\label{\detokenize{api-skos:search}}\label{\detokenize{api-skos:search-3}}
\begin{sphinxVerbatim}[commandchars=\\\{\}]
curl \PYGZdl{}REGAL\PYGZus{}API/tools/skos\PYGZhy{}lookup/search?q=http\PYGZpc{}3A\PYGZpc{}2F\PYGZpc{}2Faims.fao.org\PYGZpc{}2Faos\PYGZpc{}2Fagrovoc\PYGZpc{}2Fc\PYGZus{}13551\PYGZam{}lang=de\PYGZam{}index=agrovoc
\end{sphinxVerbatim}


\subsection{Complex Example of hierarchical content}
\label{\detokenize{api-complex-example:complex-example-of-hierarchical-content}}\label{\detokenize{api-complex-example:id1}}\label{\detokenize{api-complex-example::doc}}
\sphinxAtStartPar
The newly created resource should meet the following requirements:
\begin{itemize}
\item {} 
\sphinxAtStartPar
publishScheme and accessScheme should be private

\item {} 
\sphinxAtStartPar
Metadata are in LRMI Schema

\item {} 
\sphinxAtStartPar
Resource should be assigned to an existing user in the Drupal frontend

\end{itemize}

\sphinxAtStartPar
Structure of the Resource:

\begin{sphinxVerbatim}[commandchars=\\\{\}]
orca:50
├── lrmiData
└── orca:51
     └── document.pdf
\end{sphinxVerbatim}

\sphinxAtStartPar
For the below curl command to work from your local computer it is convenient to put some often used data
into environment variables. Prepare a simple textfile e.g. \sphinxcode{\sphinxupquote{example}} with the following content. The
\sphinxcode{\sphinxupquote{DRUPAL\_USERID}} is the numeric id which is automatically assigned to the user account by the Drupal CMS.

\begin{sphinxVerbatim}[commandchars=\\\{\}]
\PYG{n+nb}{export} \PYG{n+nv}{TOSCIENCE\PYGZus{}API}\PYG{o}{=}https://api.example.com
\PYG{n+nb}{export} \PYG{n+nv}{DRUPAL\PYGZus{}USERID}\PYG{o}{=}\PYG{l+s+s2}{\PYGZdq{}2\PYGZdq{}}
\PYG{n+nb}{export} \PYG{n+nv}{API\PYGZus{}USER}\PYG{o}{=}toscience\PYGZhy{}admin
\PYG{n+nb}{export} \PYG{n+nv}{PASSWORD}\PYG{o}{=}***********
\end{sphinxVerbatim}

\sphinxAtStartPar
Make the variables available by sourcing the file:

\begin{sphinxVerbatim}[commandchars=\\\{\}]
\PYGZdl{} \PYG{n+nb}{source} example
\end{sphinxVerbatim}


\subsubsection{Creating resource}
\label{\detokenize{api-complex-example:creating-resource}}
\sphinxAtStartPar
Initially we create a yet empty resource with the desired accessScheme, publishSchem and user id:

\begin{sphinxVerbatim}[commandchars=\\\{\}]
\PYGZdl{} curl \PYGZhy{}i \PYGZhy{}u\PYG{n+nv}{\PYGZdl{}API\PYGZus{}USER}:\PYG{n+nv}{\PYGZdl{}PASSWORD} \PYGZhy{}XPUT \PYG{n+nv}{\PYGZdl{}TOSCIENCE\PYGZus{}API}/resource/orca:50 \PYGZhy{}d\PYG{l+s+s1}{\PYGZsq{}\PYGZob{}\PYGZdq{}contentType\PYGZdq{}:\PYGZdq{}researchData\PYGZdq{},\PYGZdq{}accessScheme\PYGZdq{}:\PYGZdq{}private\PYGZdq{}, \PYGZdq{}publishScheme\PYGZdq{}:\PYGZdq{}private\PYGZdq{}, \PYGZdq{}isDescribedBy\PYGZdq{}:\PYGZob{}\PYGZdq{}createdBy\PYGZdq{}:\PYGZdq{}\PYGZsq{}}\PYG{l+s+s2}{\PYGZdq{}}\PYG{n+nv}{\PYGZdl{}DRUPAL\PYGZus{}USERID}\PYG{l+s+s2}{\PYGZdq{}}\PYG{l+s+s1}{\PYGZsq{}\PYGZdq{}\PYGZcb{}\PYGZcb{}\PYGZsq{}} \PYGZhy{}H\PYG{l+s+s1}{\PYGZsq{}Content\PYGZhy{}type:application/json\PYGZsq{}} \PYG{p}{;} \PYG{n+nb}{echo}
\end{sphinxVerbatim}

\sphinxAtStartPar
The Metadata are given in a special LRMI\sphinxhyphen{}Format and passed to a didicated endpoint:

\begin{sphinxVerbatim}[commandchars=\\\{\}]
\PYGZdl{} curl \PYGZhy{}i \PYGZhy{}u\PYG{n+nv}{\PYGZdl{}API\PYGZus{}USER}:\PYG{n+nv}{\PYGZdl{}PASSWORD} \PYGZhy{}XPOST \PYG{n+nv}{\PYGZdl{}TOSCIENCE\PYGZus{}API}/resource/orca:50/lrmiData  \PYGZhy{}\PYGZhy{}data\PYGZhy{}binary \PYG{l+s+s1}{\PYGZsq{}@lrmi.json\PYGZsq{}} \PYGZhy{}H\PYG{l+s+s1}{\PYGZsq{}Content\PYGZhy{}Type:application/json;charset=utf\PYGZhy{}8\PYGZsq{}}\PYG{p}{;} \PYG{n+nb}{echo}
\end{sphinxVerbatim}

\sphinxAtStartPar
The data are stored in a separate resource of contentType \sphinxcode{\sphinxupquote{file}}. At this point there is no relation between the two newly created resources:

\begin{sphinxVerbatim}[commandchars=\\\{\}]
\PYGZdl{} curl \PYGZhy{}i \PYGZhy{}u\PYG{n+nv}{\PYGZdl{}API\PYGZus{}USER}:\PYG{n+nv}{\PYGZdl{}PASSWORD} \PYGZhy{}XPUT \PYG{n+nv}{\PYGZdl{}TOSCIENCE\PYGZus{}API}/resource/orca:51 \PYGZhy{}d\PYG{l+s+s1}{\PYGZsq{}\PYGZob{}\PYGZdq{}contentType\PYGZdq{}:\PYGZdq{}file\PYGZdq{},\PYGZdq{}accessScheme\PYGZdq{}:\PYGZdq{}private\PYGZdq{}, \PYGZdq{}publishScheme\PYGZdq{}:\PYGZdq{}private\PYGZdq{}, \PYGZdq{}isDescribedBy\PYGZdq{}:\PYGZob{}\PYGZdq{}createdBy\PYGZdq{}:\PYGZdq{}\PYGZsq{}}\PYG{l+s+s2}{\PYGZdq{}}\PYG{n+nv}{\PYGZdl{}DRUPAL\PYGZus{}USERID}\PYG{l+s+s2}{\PYGZdq{}}\PYG{l+s+s1}{\PYGZsq{}\PYGZdq{}\PYGZcb{}\PYGZcb{}\PYGZsq{}} \PYGZhy{}H\PYG{l+s+s1}{\PYGZsq{}Content\PYGZhy{}type:application/json\PYGZsq{}} \PYG{p}{;} \PYG{n+nb}{echo}
\end{sphinxVerbatim}

\sphinxAtStartPar
Adding the actual data, a pdf\sphinxhyphen{}file in this case:

\begin{sphinxVerbatim}[commandchars=\\\{\}]
\PYGZdl{} curl \PYGZhy{}i \PYGZhy{}u\PYG{n+nv}{\PYGZdl{}API\PYGZus{}USER}:\PYG{n+nv}{\PYGZdl{}PASSWORD} \PYGZhy{}XPUT \PYG{n+nv}{\PYGZdl{}TOSCIENCE\PYGZus{}API}/resource/orca:51/data \PYGZhy{}F\PYG{l+s+s2}{\PYGZdq{}data=@document.pdf;type=application/pdf\PYGZdq{}} \PYG{p}{;} \PYG{n+nb}{echo}
\end{sphinxVerbatim}

\sphinxAtStartPar
In a final step we tell the data resource about it’s parent resource:

\begin{sphinxVerbatim}[commandchars=\\\{\}]
\PYGZdl{} curl \PYGZhy{}i \PYGZhy{}u\PYG{n+nv}{\PYGZdl{}API\PYGZus{}USER}:\PYG{n+nv}{\PYGZdl{}PASSWORD} \PYGZhy{}XPUT \PYG{n+nv}{\PYGZdl{}TOSCIENCE\PYGZus{}API}/resource/orca:51 \PYGZhy{}H\PYG{l+s+s1}{\PYGZsq{}Content\PYGZhy{}Type:application/json;charset=utf\PYGZhy{}8\PYGZsq{}} \PYGZhy{}d\PYG{l+s+s1}{\PYGZsq{}\PYGZob{}\PYGZdq{}parentPid\PYGZdq{}:\PYGZdq{}orca:50\PYGZdq{},\PYGZdq{}contentType\PYGZdq{}:\PYGZdq{}file\PYGZdq{}\PYGZcb{}\PYGZsq{}} \PYG{p}{;} \PYG{n+nb}{echo}
\end{sphinxVerbatim}


\subsubsection{Retrieving the resource}
\label{\detokenize{api-complex-example:retrieving-the-resource}}
\sphinxAtStartPar
Reading the metadata in standard json format:

\begin{sphinxVerbatim}[commandchars=\\\{\}]
\PYGZdl{} curl \PYGZhy{}i \PYGZhy{}u\PYG{n+nv}{\PYGZdl{}API\PYGZus{}USER}:\PYG{n+nv}{\PYGZdl{}PASSWORD} \PYGZhy{}XGET \PYG{n+nv}{\PYGZdl{}TOSCIENCE\PYGZus{}API}/resource/orca:51.json2 \PYG{p}{;} \PYG{n+nb}{echo}
\end{sphinxVerbatim}

\sphinxAtStartPar
The LRMI Metadata are again available via the dedicated endpoint:

\begin{sphinxVerbatim}[commandchars=\\\{\}]
\PYGZdl{} curl \PYGZhy{}i \PYGZhy{}u\PYG{n+nv}{\PYGZdl{}API\PYGZus{}USER}:\PYG{n+nv}{\PYGZdl{}PASSWORD} \PYGZhy{}XGET \PYG{n+nv}{\PYGZdl{}TOSCIENCE\PYGZus{}API}/resource/orca:51/lrmiData \PYG{p}{;} \PYG{n+nb}{echo}
\end{sphinxVerbatim}

\sphinxAtStartPar
Downloading the data

\begin{sphinxVerbatim}[commandchars=\\\{\}]
\PYG{n+nv}{\PYGZdl{}curl} \PYGZhy{}i \PYGZhy{}u\PYG{n+nv}{\PYGZdl{}API\PYGZus{}USER}:\PYG{n+nv}{\PYGZdl{}PASSWORD} \PYGZhy{}XGET \PYG{n+nv}{\PYGZdl{}TOSCIENCE\PYGZus{}API}/resource/orca:51/data  \PYGZhy{}\PYGZhy{}output data.pdf\PYG{p}{;} \PYG{n+nb}{echo}
\end{sphinxVerbatim}


\chapter{Installation}
\label{\detokenize{installation:installation}}\label{\detokenize{installation:id1}}\label{\detokenize{installation::doc}}

\chapter{Dokumentation}
\label{\detokenize{colophon:dokumentation}}\label{\detokenize{colophon:id1}}\label{\detokenize{colophon::doc}}
\sphinxAtStartPar
Diese Dokumentation ist mit \sphinxhref{https://www.sphinx-doc.org}{sphinx} erstellt.
Die Schritte, um an der Doku zu arbeiten sind folgenden


\section{Dieses Repo herunterladen}
\label{\detokenize{colophon:dieses-repo-herunterladen}}\label{\detokenize{colophon:id2}}
\begin{sphinxVerbatim}[commandchars=\\\{\}]
\PYGZdl{} git clone https://github.com/hbz/to.science
\end{sphinxVerbatim}


\section{Sphinx installieren}
\label{\detokenize{colophon:sphinx-installieren}}\label{\detokenize{colophon:id3}}
\sphinxAtStartPar
Für die Verwendung von Sphinx wird eine virtuelle Pythonumgebung im Verzeichnis \sphinxcode{\sphinxupquote{venv}} eingerichtet. Das Verzeichnis sollte nicht mit
ins git repo committet werden. Das virtuelle Python wird aktiviert und mit pip sphinx und zwei weitere themes installiert.

\begin{sphinxVerbatim}[commandchars=\\\{\}]
\PYGZdl{} \PYG{n+nb}{cd} to.science/docs
\PYGZdl{} python3 \PYGZhy{}m venv ./venv
\PYGZdl{} . venv/bin/activate
\PYGZdl{} pip install \PYGZhy{}U sphinx
\PYGZdl{} pip install \PYGZhy{}U sphinx\PYGZus{}rtd\PYGZus{}theme
\PYGZdl{} pip install \PYGZhy{}U furo
\end{sphinxVerbatim}


\section{Doku modifizieren und in HTML übersetzen}
\label{\detokenize{colophon:doku-modifizieren-und-in-html-ubersetzen}}\label{\detokenize{colophon:id4}}
\sphinxAtStartPar
Die Doku ist in \sphinxhref{https://docutils.sourceforge.io/rst.html}{reStructuredText} geschrieben wird mittels \sphinxcode{\sphinxupquote{make}} in html übersetzt.

\begin{sphinxVerbatim}[commandchars=\\\{\}]
\PYGZdl{} \PYG{n+nb}{cd} to.science/docs
\PYGZdl{} vi source/colophon.rst
\PYGZdl{} make html
\end{sphinxVerbatim}

\sphinxAtStartPar
Das fertige html findet man im Unterverzeichnis \sphinxcode{\sphinxupquote{build/html}}. Man kann eine einfachen Webserver starten und das Ergebnis
unter \sphinxurl{http://localhost:8000} ansehen.

\begin{sphinxVerbatim}[commandchars=\\\{\}]
\PYGZdl{} python3 \PYGZhy{}m http.server \PYGZhy{}\PYGZhy{}directory build/html
\end{sphinxVerbatim}


\chapter{License}
\label{\detokenize{colophon:license}}\label{\detokenize{colophon:id5}}
\noindent\sphinxincludegraphics{{cc-by-nc}.png}

\sphinxAtStartPar
This work is licensed under \sphinxhref{http://creativecommons.org/licenses/by-nc/4.0/}{CC BY\sphinxhyphen{}NC 4.0}.


\chapter{Links}
\label{\detokenize{colophon:links}}\label{\detokenize{colophon:id6}}

\section{Slides}
\label{\detokenize{colophon:slides}}\label{\detokenize{colophon:id7}}\begin{itemize}
\item {} 
\sphinxAtStartPar
Lobid \sphinxhyphen{} \sphinxurl{http://hbz.github.io/slides/}

\item {} 
\sphinxAtStartPar
Skos\sphinxhyphen{}Lookup \sphinxhyphen{} \sphinxurl{http://hbz.github.io/slides/siit-170511/\#/}

\item {} 
\sphinxAtStartPar
Regal \sphinxhyphen{} \sphinxurl{http://hbz.github.io/slides/danrw-20180905/\#/}

\end{itemize}


\section{Internes Wiki}
\label{\detokenize{colophon:internes-wiki}}\label{\detokenize{colophon:id8}}\begin{itemize}
\item {} 
\sphinxAtStartPar
\sphinxurl{https://wiki1.hbz-nrw.de/display/edd/Dokumentation}

\end{itemize}


\section{Github}
\label{\detokenize{colophon:github}}\label{\detokenize{colophon:id9}}\begin{itemize}
\item {} 
\sphinxAtStartPar
\sphinxurl{https://github.com/hbz}

\end{itemize}


\chapter{Indices and tables}
\label{\detokenize{index:indices-and-tables}}\begin{itemize}
\item {} 
\sphinxAtStartPar
\DUrole{xref,std,std-ref}{genindex}

\item {} 
\sphinxAtStartPar
\DUrole{xref,std,std-ref}{modindex}

\item {} 
\sphinxAtStartPar
{\hyperref[\detokenize{api-toscience:search}]{\sphinxcrossref{\DUrole{std,std-ref}{Search}}}}

\end{itemize}



\renewcommand{\indexname}{Index}
\printindex
\end{document}